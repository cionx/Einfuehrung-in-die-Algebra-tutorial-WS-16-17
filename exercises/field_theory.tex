\section{Körpertheorie}


\begin{question}
  Zeigen Sie, dass für einen kommutativen Ring $K$ die folgenden Bedingungen äquivalent sind:
  \begin{enumerate}
    \item
      \label{enum: K is a field}
      $K$ ist ein Körper.
    \item
      \label{enum: K has exactly two ideals}
      $K$ hat genau zwei Ideale.
    \item
      \label{enum: The zero ideal is maximal}
      Das Nullideal in $K$ ist maximal.
  \end{enumerate}
\end{question}


\begin{solution}
  (\ref{enum: K is a field} $\implies$ \ref{enum: K has exactly two ideals})
  Da $K$ ein Körper ist gilt $0 \neq K$, also hat $K$ mindestens zwei Ideale.
  Ist $I \subseteq K$ ein Ideal mit $I \neq 0$, so gibt es ein $x \in I$ mit $x \neq 0$.
  Dann ist $x$ eine Einheit in $K$, somit $K = (x) \subseteq I$ und deshalb $I = K$.
  Also sind $0$ und $K$ die einzigen Ideale in $K$.
  
  (\ref{enum: K has exactly two ideals} $\implies$ \ref{enum: The zero ideal is maximal})
  Es muss $0 \neq K$, denn sonst wäre $0$ das einzige Ideal in $K$.
  Also sind $0$ und $K$ die einzigen beiden Ideale in $K$.
  Ist $I \subseteq K$ ein Ideal mit $0 \subsetneq I$, so muss bereits $I = K$.
  Also ist $0$ ein maximales Ideal.
  
  (\ref{enum: The zero ideal is maximal} $\implies$ \ref{enum: K is a field})
  Da $0 \subseteq K$ maximal ist, ergibt sich, dass $K \cong K/0$ ein Körper ist.
\end{solution}


\begin{question}
  Es sei $K$ ein algebraisch abgeschlossener Körper.
  Zeigen Sie, dass $K$ unendlich ist.
\end{question}


\begin{solution}
  Wäre $K$ endlich, so wäre
  \[
              p(T)
    \coloneqq 1 + \prod_{\lambda \in K} (T - \lambda)
    \in       K[T]
  \]
  ein Polynom positiven Grades ohne Nullstellen (denn $p(x) = 1$ für alle $x \in K$).
  Dies stünde im Widerspruch zur algebraischen Abgeschlossenheit von $K$.
\end{solution}


\begin{question}
  Es seien $p, q \in K[T]$ zwei normierte irreduzible Polynome mit $p \neq q$.
  Zeigen Sie, dass $p$ und $q$ in $\overline{K}$ keine gemeinsamen Nullstellen haben.
\end{question}


\begin{solution}
  Gebe es eine gemeinsame Nullstelle $\alpha \in \overline{K}$ von $p$ und $q$, so wären $p$ und $q$ beide das Minimalpolynom von $\alpha$ über $K$, und somit $p = q$.
\end{solution}


