\section{Modultheorie}


% TODO:
% Calculations:
%   Smith normal form
%   simplification of quotients
%   which of these modules / abelians groups is cyclic?
%   which of these modules is finitely generated?
%   which of these modules is noetherian
% Theorems:
%   different definitions of projective modules
%   equivalent definitions of a short exact sequence being split
%   localization of modules is exact
%   rank of a free module is well-defined for nonzero commutative rings
%   every module is quotient of a free module


\begin{question}[subtitle = Multiple Choice]
  Es sei $R$ ein kommutativer Ring.
  Entscheiden Sie, welche der folgenden Aussagen wahr oder falsch sind.
  \begin{enumerate}
    \item
      Ist $M$ ein freier $R$-Modul und $S \subseteq R$ ein Unterring, so ist $M$ auch als $S$-Modul frei.
    \item
      Ist $M$ ein freier $R$-Modul endlichen Rangs, so ist auch jeder Untermodul $N \subseteq M$ frei.
    \item
      Ist jeder $R$-Modul frei, so ist $R$ ein Körper.
    \item
      Ist $M$ ein $R$-Modul und $N \subseteq M$ ein Untermodul, so gibt es einen Untermodul $P \subseteq M$ mit $M = N \oplus P$.
    \item
      Sind $M$ und $N$ zwei freie $R$-Moduln endlichen Rangs, so ist auch $\Hom_R(M, N)$ ein freier $R$-Modul endlichen Rangs.
    \item
      Ist $M$ ein endlich erzeugter $R$-Modul, so ist auch jeder Untermodul $N \subseteq M$ endlich erzeugt.
    \item
      Ist $M$ ein $R$-Modul, so dass jedes Element $m \in M$ bereits in einem endlichen Untermodul von $M$ enthalten ist, so ist $M$ endlich erzeugt.
    \item
      Ist $M$ ein endlich erzeugter $R$-Modul und $E \subseteq M$ ein minimales Erzeugendensystem, so ist $E$ endlich.
    \item
      Ist $M$ ein freier $R$-Modul endlichen Rangs und sind $E_1, E_2 \subseteq M$ zwei minimale Erzeugendensysteme, so sind $E_1$ und $E_2$ gleichmächtig.
    \item
      Ist $0 \to N \to M \to P \to 0$ eine kurze exakte Sequenz von $R$-Moduln mit $M = N \oplus P$, so spaltet die Sequenz.
    \item
      Ist $M$ ein $R$-Modul mit $M \cong M \oplus M$, so gilt $M = 0$.
  \end{enumerate}
\end{question}


\begin{solution}
  \begin{enumerate}
    \item
      Die Aussage ist falsch:
      Betrachtet man etwa $R = \Rational$ und $S = \Integer$, so ist $M = \Rational$ als $R$-Modul endlich frei vom Rang $1$, aber als $\Integer$-Modul nicht frei.
      
    \item
      Die Aussage ist falsch:
      Man betrachte für eine Körper $K$ den Ring $R = K[X,Y]$.
      Dann ist $M = R$ frei vom Rang $1$.
      Der Untermodul, d.h.\ das Ideal $(X,Y) \subseteq R$ ist aber nicht frei:
      Da $(X,Y)$ kein Hauptideal ist (siehe Übung~\ref{question: examples for non principal and not finitely generated modules}) müsste $(X,Y)$ frei vom Rang $\geq 2$ sein.
      Inbesondere wäre dann $(X,Y) = I \oplus J$ für zwei Ideale $I, J \subseteq R$ mit $I, J \neq 0$.
      (Man wähle $I$ als den Span eines Basiselements, und $J$ als den Span aller anderen.)
      Dann gibt es aber $f \in I$ und $g \in J$ mit $f, g \neq 0$.
      Wegen der Nullteilerfreiheit von $R$ ist dann auch $fg \neq 0$, da aber $fg \in IJ \subseteq I \cap J$ steht dies im Widerspruch zur Direktheit der Summe $I \oplus J$.
      
    \item
      Die Aussage ist falsch:
      Für $R = 0$ ist $0$ (bis auf Isomorphie) der einzige $R$-Modul und somit jeder $R$-Modul frei, aber $0$ ist kein Körper.
      
    \item
      Die Aussage ist falsch:
      Es seien etwa $R = \Integer$, $M = \Integer$ und $N = 2\Integer$.
      Für jeden Untermodul $P \subseteq M$ mit $P \neq 0$ gilt dann $N \cap P \neq 0$;
      für jeden Untermodul $P \subseteq M$ gilt also $P + N \subsetneq M$ oder $P \cap N \neq 0$.
      
    \item
      Die Aussage ist wahr:
      Ist $M$ vom Rang $r$ und $N$ vom Rang $s$, so gilt
      \[
              \Hom_R(M,N)
        \cong \Mat(s \times r, R)
        \cong R^{rs}
      \]
      als $R$-Moduln.
      
    \item
      Die Aussage ist falsch:
      Es sei $R$ ein Ring, so dass es ein Ideal $I \subseteq R$ gibt, das nicht endlich erzeugt ist (man siehe etwa Übung~\ref{question: examples for non principal and not finitely generated modules}).
      Dann ist $R$ ein endlich erzeugter $R$-Modul (denn $R$ ist als $R$-Modul frei vom Rang $1$), aber $I$ ist ein Untermodul von $R$, der nicht endlich erzeugt ist.
      
    \item
      Die Aussage ist falsch:
      Ist etwa $K$ ein endlicher Körper und $V$ ein unendlichdimensionaler $K$-Vektorraum, so ist jedes Element $v \in V$ in dem endlichen Untervektorraum $\generate{v}_K$ enthalten, aber $V$ ist als $K$-Vektorraum nicht endlich erzeugt.
      
    \item
      Die Aussage ist wahr:
      Da $F$ ein endliches Erzeugendensystem besitzt, enthält jedes Erzeugendensystem $E \subseteq M$ bereits ein endliches Erzeugendensystem $E' \subseteq E$ (siehe Übung~\ref{question: in a finitely generated module every generating set contains a finite generatin set}).
      Ist $E$ bereits minimal, so muss dabei $E = E'$ gelten, und $E$ somit bereits endlich sein.
      
    \item
      Die Aussage ist falsch:
      Betrachtet man $R = \Integer$ und $M = \Integer$, so sind $\{1\}, \{2, 3\} \subseteq \Integer$ zwei minimal Erzeugendensysteme, die nicht gleichmächtig sind.
      
    \item
      Die Aussage ist falsch:
      Man betrachte für den Ring $R = \Integer$ die kurze exakte Sequenz
      \[
                            0
        \to                 \Integer
        \xlongrightarrow{f} \Integer \oplus \bigoplus_{n \geq 2} (\Integer/2)
        \xlongrightarrow{g} \bigoplus_{n \geq 1} (\Integer/2)
        \to                 0
      \]
      wobei $f(n) = (2n,0,0,0,\dotsc)$ und $g(n, a_1, a_2, a_3, \dotsc) = (\overline{n}, a_1, a_2, a_3, \dotsc)$.
      Diese Sequenz spaltet nicht:
      Ansonsten gebe es nämlich einen Homomorphismus von $R$-Moduln $s \colon \Integer \oplus \bigoplus_{n \geq 1} (\Integer/2) \to \Integer$ mit $s \circ f = \id_\Integer$.
      Dann würde inbesondere
      \[
          2 s(1,0,0,\dotsc)
        = s(2,0,0,\dotsc)
        = s(f(1))
        = 1
      \]
      gelten, was in $\Integer$ nicht möglich ist.
    \item
      Die Aussage ist falsch:
      Man betrachten etwa den Modul $M = \bigoplus_{n \in \Natural} R$.
  \end{enumerate}
\end{solution}


\begin{question}
  Zeigen Sie, dass es auf jeder abelschen Gruppe genau eine $\Integer$-Modulstruktur gibt.
\end{question}


\begin{solution}
  Es sei $A$ eine abelsche Gruppe.
  Aus der Vorlesung ist die Bijektion
  \begin{align*}
    \{\text{$\Integer$-Modulstrukturen $\Integer \times A \to A$}\}
    &\longleftrightarrow
    \{\text{Ringhomomorphismen $\Integer \to \End(A)$}\},
    \\
                    \mu
    &\longmapsto    (n \mapsto (a \mapsto \mu(n,a))),
    \\
                    ( (n,a) \mapsto \phi(n)(a) )
    &\longmapsfrom  \phi.
  \end{align*}
  bekannt.
  Dabei ist
  \[
      \End(A)
    = \{f \colon A \to A \mid \text{$f$ ist additiv}\}
  \]
  ein Ring unter punktweiser Adddition und Komposition.
  Da es genau einen Ringhomomorphismus $\Integer \to \End(A)$ gibt (siehe Übung~\ref{qst: Z is inital}) folgt die Aussage.
\end{solution}


\begin{question}
  Es sei $R$ ein kommutativer Ring und $M$ ein $R$-Modul.
  Zeigen Sie, dass $\Hom_R(R, M) \cong M$ als $R$-Moduln.
\end{question}


\begin{solution}
  Wir zeigen, dass die Abbildung $\varphi \colon \Hom_R(R, M) \to M$, $f \mapsto f(1)$ ein Isomorphismus von $R$-Moduln ist:
  Dass $\varphi$ ein Homomorphismus von $R$-Moduln ist, ergibt sich direkt daraus, dass die $R$-Modulstruktur auf $\Hom_R(R, M)$ punktweise definiert ist.
  Die Injektivität von $\varphi$ ergibt sich daraus, dass $R = \generate{1}_R$, und somit jeder $R$-Modulhomomorphismus $f \colon R \to M$ durch die Einschränkung $f|_{\{1\}}$ bereits eindeutig bestimmt ist.
  Für jedes $m \in M$ ergibt sich ein Homomorphismus von $R$-Moduln $f_m \colon R \to M$, $r \mapsto rm$;
  für diesen gilt $\varphi(f_m) = f_m(1) = m$, was die Surjektivität von $\varphi$ zeigt.
\end{solution}


\begin{question}
  Es sei $R$ ein Ring und $e \colon M \to M$ ein idempotenter Endomorphismus eines $R$-Moduls $M$, d.h.\ es gilt $e^2 = e$.
  Zeigen Sie, das $M = \im e \oplus \ker e$ und dass $e(m + m') = m$ für alle $m \in \im e$ und $m' \in \ker e$.
\end{question}


\begin{solution}
  Es gilt $M = \im e + \ker e$, denn jedes $m \in M$ lässt sich als $m = e(m) + m - e(m)$ schreiben, wobei $e(m) \in \im e$ und $m - e(m) \in \ker(e)$ (denn $e(m - e(m)) = e(m) - e^2(m) = 0$).
  Für jedes $m \in M$ gilt $e(m) = m$, denn es gibt ein $\tilde{m} \in M$ mit $m = e(\tilde{m})$, und somit gilt $e(m) = e(e(\tilde{m})) = e^2(\tilde{m}) = e(\tilde{m}) = m$.
  Für $m \in \im e \cap \ker e$ folgt, dass $m = e(m) = 0$;
  deshalb gilt $\im e \cap \ker e = 0$.
\end{solution}


\begin{question}
  Es sei $R$ ein kommutativer Ring und $M$ ein $R$-Modul.
  Es sei $I \subseteq R$ ein Ideal und $S \subseteq R$ eine multiplikative Menge.
  \begin{enumerate}
    \item
      Zeigen Sie, dass sich die $R$-Modulstruktur auf $M$ genau dann zu einer $R/I$-Modulstruktur fortsetzen lässt, wenn $IM =  0$ (d.h.\ wenn $am = 0$ für alle $a \in I$ und $m \in M$).
      Entscheiden Sie, ob diese Fortsetzung eindeutig ist.
    \item
      Zeigen Sie, dass sich die $R$-Modulstruktur auf $M$ genau dann zu einer $R_S$-Modulstruktur fortsetzen lässt, wenn für jedes $s \in S$ die Abbildung $\lambda_s \colon M \to M$, $m \mapsto sm$ bijektiv ist.
      Entscheiden Sie, ob diese Fortsetzung eindeutig ist.
  \end{enumerate}
\end{question}

\begin{solution}
  Es sei $\End(M) \coloneqq \{f \colon M \to M \mid \text{$f$ ist additiv}\}$.
  Die $R$-Modulstruktur auf $M$ entspricht dem Ringhomomorphismus $\lambda \colon R \to \End(M)$, $r \mapsto \lambda_r$ mit $\lambda_r(m) = r \cdot m$ für alle $r \in R$, $m \in M$.
  \begin{enumerate}
    \item
      Es sei $\pi \colon R \to R/I$, $r \mapsto \overline{r}$ die kanonische Projektion.
      Eine $R/I$-Modulstruktur auf $M$ entspricht genau einem Ringhomomorphismus $\overline{\lambda} \colon R/I \to \End(M)$.
      Dass es sich um eine Fortsetzung der $R$-Modulstruktur handelt, ist dabei äquivalent dazu, dass $\overline{\lambda}$ eine Forsetzung von $\lambda$ ist, d.h.\ dass das folgende Diagramm kommutiert:
      \[
        \begin{tikzcd}[ampersand replacement = \&]
              R
              \arrow{r}{\lambda}
              \arrow[swap]{d}{\pi}
          \&  \End(M)
          \\
              R/I
              \arrow[swap]{ru}{\overline{\lambda}}
          \&  {}
        \end{tikzcd}
      \]
      Nach der universellen Eigenschaft des Quotienten $R/I$ ist eine solche Fortsetzung $\overline{\lambda}$ eindeutig, und sie existiert genau dann, wenn $I \subseteq \ker \lambda$.
      Es bleibt zu zeigen, dass genau dann $I \subseteq \ker \lambda$, wenn $IM = 0$.
      Dies ergibt sich daraus, dass für alle $r \in R$
      \[
              r \in \ker \lambda
        \iff  \lambda_r = 0
        \iff  \forall m \in M : \lambda_r(m) = 0
        \iff  \forall m \in M : r \cdot m = 0.
      \]
    \item
      Es sei $i \colon R \to R_S$, $r \mapsto r/1$ der kanonische Ringhomomorphismus.
      Eine $R_S$-Mo\-dul\-struk\-tur auf $M$ entspricht einen Ringhomomorphismus $\hat{\lambda} \colon R_S \to \End(M)$.
      Dass es sich dabei um eine Forsetzung der $R$-Modulstruktur handelt, ist äquivalent dazu, dass $\hat{\lambda}$ eine Fortsetzung von $\lambda$ ist, d.h.\ dass das folgende Diagramm kommutiert:
      \[
        \begin{tikzcd}[ampersand replacement = \&]
              R_S
              \arrow{rd}{\hat{\lambda}}
          \&  {}
          \\
              R
              \arrow{u}{i}
              \arrow{r}{\lambda}
          \&  \End(M)
        \end{tikzcd}
      \]
      Nach der universellen Eigenschaft der Lokalisierung $R_S$ ist eine solche Fortsetzung $\hat{\lambda}$ eindeutig, und sie existiert genau dann, wenn $\lambda(s)$ für jedes $s \in S$ eine Einheit in $\End(M)$ ist.
      Da ein Element $f \in \End(M)$ genau dann eine Einheit ist, wenn $f$ bijektiv ist, ist die obige Bedingung äquivalent dazu, dass $\lambda_s$ für alle $s \in S$ bijektiv ist.
  \end{enumerate}
\end{solution}


\begin{question}
  \label{question: in a finitely generated module every generating set contains a finite generatin set}
  Es sei $M$ ein endlich erzeugter $R$-Modul.
  Zeigen Sie, dass jedes Erzeugendensystem $S \subseteq M$ ein endliches Erzeugendensystem enthält.
\end{question}


\begin{solution}
  Es sei $\{m_1, \dotsc, m_s\} \subseteq M$ ein endliches Erzeugendensystem.
  Da $S$ ein Erzeugendensystem ist, lässt sich jedes $m_i$ als $m_i = r_{i,1} s_{i,1} + \dotsb + r_{i,t_i} s_{i,t_i}$ mit $t_i \geq 0$, $s_{i,1}, \dotsc, s_{i,t_i} \in S$ und $r_{i,1}, \dotsc, r_{i,t_i} \in R$ schreiben.
  Für $S' \coloneqq \{s_{i,j} \mid i = 1, \dotsc, s, j = 1, \dotsc, t_i\}$ gilt dann $m_i \in \generate{S}$ für alle $i = 1, \dotsc, s$ und deshalb
  \[
              M
    =         \generate{m_1, \dotsc, m_s}
    \subseteq \generate{S'}
    \subseteq M.
  \]
  Also ist $\generate{S'} = M$ und somit $S'$ ein endliches Erzeugendensystem von $M$.
\end{solution}


\begin{question}
  \label{question: restriction of a short exact sequence to a submodule of its middle term}
  Es sei $R$ ein Ring und $0 \to N \xrightarrow{f} M \xrightarrow{g} P \to 0$ ein kurze exakte Sequenz von $R$-Moduln und $M' \subseteq M$ ein Untermodul.
  Zeigen sie, dass auch
  \[
    0 \to f^{-1}(M') \xrightarrow{f'} M' \xrightarrow{g'} g(M') \to 0
  \]
  eine kurze exakte Sequenz ist, wobei $f' \colon f^{-1}(M') \to M'$, $m \mapsto f(m)$ und $g' \colon M' \to g(M')$, $m \mapsto g(m)$ die entsprechenden Einschränkungen von $f$ und $g$ bezeichnen.
\end{question}


\begin{solution}
  Es ist klar, dass $f'$ und $g'$ wohldefinierte Homomorphismen sind.
  Die Injektivität von $f'$ folgt aus der von $f$, und die Surjektivität von $g'$ aus $\im g' = g'(M') = g(M')$.
  Da $g \circ f = 0$ gilt, gilt auch $g' \circ f' = 0$, also $\im f' \subseteq \ker g'$.
  Ist andererseits $m \in \ker g'$, so gilt $m \in \ker g = \im f$, weshalb es $n \in N$ mit $f(n) = m$ gibt.
  Dabei gilt bereits $n \in f^{-1}(M')$, da ja $f(n) = m \in M'$, und somit $m = f(n) = f'(n) \in \im f'$.
  Das zeigt, dass auch $\ker g' \subseteq \im f'$.
\end{solution}


\begin{question}
  \label{question: submodules of free modules over pid are also free}
  Es sei $R$ ein Hauptidealring und $F$ ein freier $R$-Modul mit endlichen Rang $n \geq 0$.
  Es sei $F' \subseteq F$ ein Untermodul.
  Zeigen Sie, dass $F'$ frei vom Rang $r \leq n$ ist.
\end{question}


\begin{remark*}
  Mithilfe des Auswahlaxioms (in Form des Wohlordnungssatzes) verallgemeinert sich Übung~\ref{question: submodules of free modules over pid are also free} auf freie Moduln beliebigen Rangs.
\end{remark*}


\begin{solution}
  Es genügt den Fall $F = R^n$ für $n \geq 0$ zu betrachten.
  Wir zeigen die Aussage per Induktion über $n$.
  Für $n = 0$ ist die Aussage klar.
  
  Für $n = 1$ sei $\aideal \subseteq R$ ein Untermodul, also ein Ideal.
  Für $\aideal = 0$ ist die Aussage klar, wir beschränken uns also auf den Fall $\aideal \neq 0$.
  Es ist $\aideal$ ein Hauptideal, also $\aideal = (a)$ für ein $a \in \aideal$, und nach Annahme gilt $a \neq 0$.
  Die Teilmenge $\{a\} \subseteq \aideal$ ist linear unabhängig, denn die Abbildung $R \to \aideal$, $r \mapsto ra$ ist injektiv, da $R$ ein Integritätsbereich ist und $a \neq 0$ gilt.
  Also ist $\{a\}$ eine Basis von $\aideal$, und $\aideal$ somit frei vom Rang $1$.
  
  Es sei nun $n \geq 2$ und die Aussage gelte für alle kleineren Ränge.
  Durch die Inklusion $i \colon R^{n-1} \to R^n$, $(x_1, \dotsc, x_{n-1}) \mapsto (x_1, \dotsc, x_{n-1}, 0)$ und die Projektion $p \colon R^n \to R$, $(x_1, \dotsc, x_n) \mapsto x_n$ erhalten wir eine kurze exakte Sequenz $0 \to R^{n-1} \xrightarrow{i} R^n \xrightarrow{p} R \to 0$.
  
  Ist $F' \subseteq F$ ein Untermodul, so schränkt sich diese kurze exakte Sequenz zu einer kurzen exakten Sequenz
  \begin{equation}
    \label{equation: restrition of shor exact sequence}
    0 \to i^{-1}(F') \xrightarrow{i'} F' \xrightarrow{p'} p(F') \to 0
  \end{equation}
  ein, wobei $i'$ und $p'$ die entsprechenden Einschränkungen von $i$ und $p$ bezeichnen (siehe Übung~\ref{question: restriction of a short exact sequence to a submodule of its middle term}).
  Dabei sind $i^{-1}(F') \subseteq R^{n-1}$ und $p(F') \subseteq R$ Untermoduln, und somit nach Induktionsannahme frei vom Rang $\leq n-1$ und $\leq 1$.
  Da $p(F')$ frei ist, spaltet die Sequenz \eqref{equation: restrition of shor exact sequence};
  inbesondere ist deshalb $F' \cong i^{-1}(F') \oplus p(F')$.
  Somit ist $F'$ frei von Rang $\leq n-1 + 1 = n$.
\end{solution}


\begin{question}
  Es sei $R$ ein Hauptidealring und $M$ ein endlich erzeugter $R$-Modul.
  Zeigen Sie, dass auch jeder Untermodul $N \subseteq M$ endlich erzeugt ist.
\end{question}


\begin{solution}
  Es sei $m_1, \dotsc, m_t \in M$ ein endliches Erzeugendensystem und $\varphi \colon R^t \to M$ der eindeutige Homomorphismus von $R$-Moduln mit $\varphi(e_i) = m_i$ für alle $i = 1, \dotsc, t$ (hier bezeichnet $e_1, \dotsc, e_t \in R^t$ die Standardbasis).
  Dann ist $\varphi$ surjektiv, und deshalb $F \coloneqq \varphi^{-1}(N)$ ein Untermodul von $R^t$, für den $\varphi(F) = N$ gilt.
  Der $R$-Modul $R^t$ ist frei vom Rang $t$; da $R$ ein Hauptidealring ist, folgt daraus, dass der Untermodul $F \subseteq R^t$ frei vom Rang $s \leq t$ ist (siehe Übung~\ref{question: submodules of free modules over pid are also free}).
  Insbesondere ist $F$ endlich erzeugt.
  Somit ist auch $N = \varphi(F)$ endlich erzeugt.
\end{solution}


\begin{question}
  \label{question: finitely generated in short exact sequences}
  Es sei $0 \to N \xrightarrow{f} M \xrightarrow{g} P \to 0$ eine kurze exakte Sequenz von $R$-Moduln.
  \begin{enumerate}
    \item
      Zeigen Sie, dass $P$ endlich erzeugt ist, wenn $M$ endlich erzeugt ist.
    \item
      Zeigen Sie, dass $M$ endlich erzeugt ist, wenn $P$ und $N$ endlich erzeugt sind.
  \end{enumerate}
\end{question}


\begin{solution}
  \begin{enumerate}
    \item
      Es seien $m_1, \dotsc, m_t \in M$ mit $M = \generate{m_1, \dotsc, m_t}_R$.
      Wegen der Surjektivität von $g$ gilt dann
      \[
          P
        = g(M)
        = g(\generate{m_1, \dotsc, m_t}_R)
        = \generate{g(m_1), \dotsc, g(m_t)}_R,
      \]
      weshalb $P$ endlich erzeugt ist.
    \item
      Es seien $n_1, \dotsc, n_s \in N$ und $p_{s+1}, \dotsc, p_t \in P$ endliche Erzeugendensysteme.
      Für alle $i = 1, \dotsc, s$ sei $m_i \coloneqq f(n_i) \in M$;
      wegen der Surjektivität gibt es für jedes $i = s+1, \dotsc, t$ ein $m_i \in M$ mit $g(m_i) = p_i$.
      Dann gilt $\generate{m_1, \dotsc, m_s, m_{s+1}, \dotsc, m_t}_R = M$:
      
      Für $m \in M$ ist $g(m) \in P$ und deshalb $g(m) = r_{s+1} p_{s+1} + \dotsb + r_t p_t$ für passende $r_{s+1}, \dotsc, r_t \in R$.
      Es sei $m' \coloneqq r_{s+1} m_{s+1} + \dotsb + r_t m_t \in M$.
      Es gilt
      \[
          g(m')
        = r_{s+1} g(m_{s+1}) + \dotsb + r_t g(m_t)
        = r_{s+1} p_{s+1} + \dotsb + r_t p_t
        = g(m)
      \]
      und somit $m - m' \in \ker g = \im N$.
      Es sei $n \in N$ mit $f(n) = m - m'$.
      Dann gilt $n = r_1 n_1 + \dotsb + r_s n_s$ für passende $r_1, \dotsc, r_s \in R$, und somit
      \[
          m - m'
        = f(n)
        = r_1 f(n_1) + \dotsb + r_s f(n_s)
        = r_1 m_1 + \dotsb + r_s m_s.
      \]
      Ingesamt erhalten wir, dass
      \[
          m
        = m - m' + m'
        = r_1 m_1 + \dotsb + r_s m_s + r_{s+1} m_{s+1} + \dotsb + r_t m_t.
      \]
  \end{enumerate}
\end{solution}


\begin{question}[subtitle=Charakterisierungen noetherscher Moduln]
  Es sei $M$ ein $R$-Modul.
  Zeigen Sie, dass die folgenden Bedingungen äquivalent sind:
  \begin{enumerate}
    \item
      \label{enum: every submodule is finitely generated}
      Jeder $R$-Untermodul von $M$ ist endlich erzeugt.
    \item
      \label{enum: every ascending chain of submodules stabilizes}
      Jede aufsteigende Kette
      \[
        N_0 \subseteq N_1 \subseteq N_2 \subseteq N_3 \subseteq N_4 \subseteq \dotso
      \]
      von Untermoduln von $M$ stabilisiert, i.e.\ es gibt ein $i \geq 0$ mit $N_j = N_i$ für alle $j \geq i$.
    \item
      \label{enum: every non empty collection of submodules has a maximal element}
      Jede nicht-leere Menge $\mathcal{S}$ bestehend aus $R$-Untermoduln von $M$ besitzt ein maximales Element, d.h.\ ein Element $N \in \mathcal{S}$, das in keinem anderen Element von $\mathcal{S}$ echt enthalten ist.
  \end{enumerate}
\end{question}


\begin{solution}
  Der Vollständigkeit halber geben wir mehr Implikationen an, als notwendig sind.
  
  (
  \ref{enum: every submodule is finitely generated}
  $\implies$
  \ref{enum: every ascending chain of submodules stabilizes}
  )
  Es sei
  \begin{equation}
    \label{eqn: an ascending chain of submodules}
              N_0
    \subseteq N_1
    \subseteq N_2
    \subseteq N_3
    \subseteq N_4
    \subseteq \dotso
  \end{equation}
  eine aufsteigende Kette von Untermoduln von $M$.
  Dann ist $N \coloneqq \bigcup_{i \geq 0} N_i$ ein Untermodul von $M$.
  Nach Annahme ist $N$ endlich erzeugt; es sei $n_1, \dotsc, n_t \in N$ ein endliches Erzeugendensystem.
  Da $n_1, \dotsc, n_t \in N = \bigcup_{i \geq 0} N_i$ gibt es für jedes $j = 1, \dotsc, t$ ein $i_j \geq 0$ mit $n_j \in N_{i_j}$;
  da $N_i \subseteq N_{i+1}$ für alle $i \geq 0$ gibt es bereits ein $I \geq 0$ mit $n_1, \dotsc, n_t \subseteq N_I$.
  Damit gilt
  \[
              N
    =         \generate{n_1, \dotsc, n_t}_R
    \subseteq N_I
    \subseteq \bigcup_{i \geq 0} N_i
    =         N
  \]
  und deshalb bereits $N = N_I$.
  Für alle $i \geq I$ git dann $N = N_I \subseteq N_i \subseteq N$ und somit $N_i = N_I$.
  Also stabilisiert die Kette \eqref{eqn: an ascending chain of submodules}.
  
  (
  \ref{enum: every ascending chain of submodules stabilizes}
  $\implies$
  \ref{enum: every submodule is finitely generated}
  )
  Es gebe einen Untermodul $N \subseteq M$, der nicht endlich erzeugt ist.
  Es gilt notwendigerweise $N \neq 0$.
  Wir konstruieren eine nicht-stabilisierende Kette
  \[
                N_0
    \subsetneq  N_1
    \subsetneq  N_2
    \subsetneq  N_3
    \subsetneq  N_4
    \subsetneq \dotso
    \subsetneq  N
    \subseteq   M
  \]
  von endlich erzeugten von $N$ wie folgt:
  Wir beginnen mit $N_0 \coloneqq 0$.
  Ist $N_i$ definiert, so gilt $N_i \subsetneq N$, da $N_i$ endlich erzeugt ist, $N$ aber nicht. 
  Es gibt also $f \in N$ mit $f \notin N_i$.
  Da $N_i$ endlich erzeugt ist, gilt dies auch für $N_{i+1} \coloneqq N_i + \generate{f}_R$, und nach Wahl von $f$ gilt $N_i \subsetneq N_{i+1}$.
  
  
  (
  \ref{enum: every ascending chain of submodules stabilizes}
  $\implies$
  \ref{enum: every non empty collection of submodules has a maximal element}
  )
  Es gebe eine nicht-leere Menge $\mathcal{S}$ von Untermoduln von $M$, die kein maximales Element besitzt.
  Dann gibt es für jedes $N \in \mathcal{S}$ ein $N' \in \mathcal{S}$ mit $N \subsetneq N'$.
  Ausgehend von einem beliebigen $N_0 \in \mathcal{S}$ erhalten wir somit eine Kette
  \[
                N_0
    \subsetneq  N_1
    \subsetneq  N_2
    \subsetneq  N_3
    \subsetneq  N_4
    \subsetneq \dotso
  \]
  von Untermoduln von $M$, die nicht stabilisiert.
  
  
  (
  \ref{enum: every non empty collection of submodules has a maximal element}
  $\implies$
  \ref{enum: every ascending chain of submodules stabilizes}
  )
  Es sei
  \[
              N_0
    \subseteq N_1
    \subseteq N_2
    \subseteq N_3
    \subseteq N_4
    \subseteq \dotso
  \]
  eine aufsteigende Kette von Untermoduln von $M$.
  Dann ist $\mathcal{S} \coloneqq \{N_i \mid i \in I\}$ eine nicht-leere Menge von Untermoduln von $M$.
  Nach Annahme hat $\mathcal{S}$ ein maximales Element, d.h.\ es gibt ein $i \in I$ mit $N_i \subsetneq N_j$ für alle $j \geq 0$.
  Es muss also bereits $N_i = N_j$ für alle $j \geq i$ gelten, weshalb die Kette stabilisiert.
  
  (
  \ref{enum: every non empty collection of submodules has a maximal element}
  $\implies$
  \ref{enum: every submodule is finitely generated}
  )
  Es sei $N \subseteq M$ ein Untermodul von $M$ und
  \[
      \mathcal{S}
    = \{
        P \subseteq N
      \mid
        \text{$P$ ist ein endlich erzeugter Untermodul von $N$}
      \}.
  \]
  Dann ist $\mathcal{S}$ eine nicht-leere ($0 \in \mathcal{S}$) Menge von Untermoduln von $M$, und besitzt daher nach Annahme ein maximales Element $N'$.
  Wäre $N' \subsetneq N$, so gebe es ein $f \in N$ mit $f \notin N'$.
  Dann wäre aber $N'' \coloneqq N' + \generate{f}_R$ ein endlich erzeugter Untermodul von $N$, also ein Element von $\mathcal{S}$, mit $N' \subseteq N''$, was der Maximalität von $N'$ widerspricht.
  Also muss bereits $N = N'$, und $N$ somit endlich erzeugt sein.
\end{solution}


\begin{question}
  Es sei $R$ ein Ring.
  \begin{enumerate}
    \item
      Es sei $0 \to N \to M \to P \to 0$ eine kurze exakte Sequenz von $R$-Moduln.
      Zeigen Sie, dass $M$ genau dann noethersch ist, wenn $N$ und $P$ beide noethersch sind.
    \item
      Folgern Sie, dass für alle noetherschen $R$-Moduln $M_1, \dotsc, M_s$ auch $M_1 \oplus \dotsb \oplus M_s$ noethersch ist.
  \end{enumerate}
  Es sei nun $R$ zusätzlich noethersch.
  \begin{enumerate}[resume]
    \item
      Folgern Sie, dass inbesondere $R^n$ für alle $n \geq 0$ noethersch ist, falls $R$ noethersch ist.
    \item
      Folgern Sie, dass jeder endlich erzeugte $R$-Modul noethersch ist, falls $R$ noethersch ist.
  \end{enumerate}  
\end{question}


\begin{question}
  \begin{enumerate}
    \item
      Wir bezeichnen die Abbildungen mit $0 \to N \xrightarrow{f} M \xrightarrow{g} P \to 0$.
      
      Es sei zunächst $M$ noethersch, d.h.\ jeder Untermodul von $M$ sei endlich erzeugt.
      
      Dann ist auch der Untermodul $f(N) \subseteq M$ noethersch, denn jeder Untermodul von $f(N)$ ist auch ein Untermodul von $M$, und somit endlich erzeugt.
      Wegen der Injektivität von $f$ gilt $N \cong f(N)$, weshalb auch $N$ noethersch ist.
      
      Ist $P' \subseteq P$ ein Untermodul, so ist $g^{-1}(P') \subseteq M$ ein Untermodul, und somit endlich erzeugt.
      Damit ist auch $g(g^{-1}(P'))$ endlich erzeugt, und wegen der Surjektivität von $g$ gilt $g(g^{-1}(P')) = P'$.
      Somit ist jeder Untermodul von $P$ endlich erzeugt, also $P$ noethersch.
      
      Es seien nun $N$ und $P$ noethersch.
      Ist $M' \subseteq M$ ein Untermodul, so schränkt die kurze exakte Sequenz $0 \to N \xrightarrow{f} M \xrightarrow{g} P \to 0$ zu einer kurzen exakten Sequenz
      \begin{equation}
        \label{equation: restriction of short exact sequence to a submodule}
        0 \to f^{-1}(M') \xrightarrow{f'} M' \xrightarrow{g'} g(M') \to 0
      \end{equation}
      ein, wobei $f'$ und $g'$ die entsprechenden Einschränkungen von $f$ und $g$ bezeichnen (siehe Übung~\ref{question: restriction of a short exact sequence to a submodule of its middle term}).
      Nach Annahme sind die Untermoduln $f^{-1}(M') \subseteq N$ und $g(M') \subseteq P$ endlich erzeugt.
      In \eqref{equation: restriction of short exact sequence to a submodule} sind also die beiden äußeren Terme endlich erzeugt;
      somit ist auch der mittlere Term, also $M'$, endlich erzeugt (siehe Übung~\ref{question: finitely generated in short exact sequences}).
    \item
      Dank Induktion genügt es zu zeigen, dass für je zwei noethersche Moduln $M$ und $N$ auch $M \oplus N$ noethersch ist.
      Dies ergibt sich aus dem vorherigen Teil der Aufgabe mithilfe der kurzen exakten Sequenz $0 \to M \xrightarrow{i} M \oplus N \xrightarrow{p} N \to 0$, wobei $i \colon M \to M \oplus N$, $m \mapsto (m,0)$ die kanonische Inklusion bezeichnet und $p \colon M \oplus N \to N$, $(m,n) \mapsto n$ die kanonische Projektion.
    \item
      Da $R$ noethersch ist, ist nach dem vorherigen Aufgabenteil auch $R^n = R \oplus \dotsb \oplus R$ wieder noethersch.
    \item
      Ist $M$ ein $R$-Modul mit endlichen Erzeugendensystem $\{m_1, \dotsc, m_n\} \subseteq M$, so ist der eindeutige Modulhomomorphismus $\varphi \colon R^n \to M$ mit $\varphi(e_i) = m_i$ für alle $i = 1, \dotsc, n$ (also $\varphi(x_1, \dotsc, x_n) = x_1 m_1 + \dotsb + x_n m_n$) bereits surjektiv.
      Wir erhalten somit eine kurze exakte Sequenz $0 \to \ker \varphi \xrightarrow{i} R^n \xrightarrow{\varphi} M \to 0$, wobei $i \colon \ker \varphi \to R^n$ die Inklusion ist.
      Da $R^n$ nach dem vorherigen Aufgabenteil noethersch ist, ist nach dem ersten Aufgabenteil auch $M$ noethersch.
  \end{enumerate}
\end{question}


\begin{question}
  \label{question: surjective endomorphisms of noetherian modules are isomorphisms}
  Es sei $R$ ein Ring und $M$ ein noetherscher $R$-Modul.
  Zeigen Sie, dass jeder surjektive Endomorphismus $f \colon M \to M$ bereits ein Isomorphismus ist.
\end{question}


\begin{remark*}
  Ist $R$ ein kommutativer Ring, so lässt sich Übung~\ref{question: surjective endomorphisms of noetherian modules are isomorphisms} dazu verallgemeinern, dass jeder surjektive Endomorphismus $M \to M$ eines endlich erzeugten $R$-Moduls bereits ein Isomorphismus ist:
  Mithilfe von Lokalisierungen und Nakayamas Lemma lässt sich diese allgemeine Aussage auf den Fall zurückführen, dass $R$ ein Körper ist, und für Körper ist die Aussage aus der Linearen Algebra bekannt.
\end{remark*}


\begin{solution}
  Da $M$ noethersch ist stabilisiert die Kette
  \[
              0
    =         \ker f^0
    \subseteq \ker f
    \subseteq \ker f^2
    \subseteq \ker f^3
    \subseteq \ker f^4
    \subseteq \dotso,
  \]
  d.h.\ es gibt $n \geq 1$ mit $\ker f^n = \ker f^k$ für alle $k \geq n$.
  Inbesondere gilt $\ker f^n = \ker f^{2n}$.
  Für $m \in \ker f^n$ gibt es wegen der Surjektivität von $f$ ein $m' \in M$ mit $m = f^n(m')$.
  Dann gilt $0 = f^n(m) = f^{2n}(m')$, also $m' \in \ker f^{2n} = \ker f^n$.
  Deshalb ist bereits $m = f^n(m) = 0$.
  Das zeigt die Gleichheit $\ker f^n = 0$, und da $\ker f \subseteq \ker f^n$ somit auch, dass $\ker f = 0$ gilt.
  Also ist $f$ injektiv, und somit bereits ein Isomorphismus.
\end{solution}


\begin{question}
  Es sei $R$ ein Ring und $F$ ein freier $R$-Modul mit Basis $(b_i)_{i \in I}$.
  Zeigen Sie, dass es für jeden $R$-Modul $M$ und jede Familie $(m_i)_{i \in I}$ von Elementen $m_i \in M$ einen eindeutigen Homomorphismus von $R$-Moduln $\varphi \colon F \to M$ gibt, so dass $\varphi(b_i) = m_i$ für alle $i \in I$ gilt.
\end{question}


\begin{solution}
  Wir zeigen zunächst die Eindeutigkeit:
  Hierfür sei $\varphi \colon F \to M$ ein Homomorphismen mit $\varphi(b_i) = m_i$ für alle $i \in I$.
  Jedes $x \in F$ lässt sich als Linearkombination $x = \sum_{i \in I} r_i b_i$ mit $r_i = 0$ für fast alle $i \in I$ schreiben, da die Familie $(b_i)_{i \in I}$ ein Erzeugendensystem von $F$ ist.
  Deshalb gilt
  \[
      \varphi(x)
    = \varphi\left( \sum_{i \in I} r_i b_i \right)
    = \sum_{i \in I} r_i \varphi(b_i)
    = \sum_{i \in I} r_i m_i.
  \]
  Also ist $\varphi$ bereits eindeutig bestimmt.
  
  Nun zur Existenz:
  Für jedes $x \in F$ ist die Darstellung $x = \sum_{i \in I} r_i b_i$ mit $r_i = 0$ für fast alle $i \in I$ eindeutig, da die Familie $(b_i)_{i \in I}$ linear unabhängig ist.
  Daher ist der Ausdruck $\varphi(x) \coloneqq \sum_{i \in I} r_i m_i$ wohldefiniert, und liefert eine Funktion $\varphi \colon F \to M$.
  Ist $r \in R$ und sind $x, y \in F$ mit $x = \sum_{i \in I} r_i b_i$ und $y = \sum_{i \in I} s_i b_i$, so gelten $rx = \sum_{i \in I} rr_i b_i$ und $x + y = \sum_{i \in I} (r_i + s_i) b_i$.
  Deshalb gilt
  \begin{gather*}
      \varphi(rx)
    = \varphi\left( \sum_{i \in I} rr_i b_i \right)
    = \sum_{i \in I} rr_i m_i
    = r \sum_{i \in I} r_i m_i
    = r \varphi\left( \sum_{i \in I} r_i b_i \right)
    = r \varphi(x)
  \shortintertext{und}
      \varphi(x + y)
    = \sum_{i \in I} (r_i + s_i) b_i
    = \left( \sum_{i \in I} r_i b_i \right) + \left( \sum_{i \in I} s_i b_i \right)
    = \varphi(x) + \varphi(y).
  \end{gather*}
  Also ist $\varphi$ ein Modulhomomorphismus.
\end{solution}


\begin{question}
  \begin{enumerate}
    \item
      Geben Sie für einen passenden kommutativen Ring $R$ eine kurze exakte Sequenz von $R$-Moduln $0 \to N \to M \to P \to 0$ an, die nicht spaltet.
    \item
      Es sei $R$ ein Ring und $F$ ein freier $R$-Modul.
      Zeigen Sie, dass jede kurze exakte Sequenz von $R$-Moduln $0 \to N \to M \to F \to 0$ spaltet.
  \end{enumerate}
\end{question}


\begin{solution}
  \begin{enumerate}
    \item
      Wir betrachten die folgende kurze exakte Sequenz von $\Integer$-Moduln, d.h.\ von abelschen Gruppen:
      \[
                                              0
        \to                                   \Integer
        \xrightarrow{\cdot 2}                 \Integer
        \xrightarrow{x \mapsto \overline{x}}  \Integer/2
        \to                                   0
      \]
      Würde diese kurze exakte Sequenz spalten, so wäre $\Integer \cong \Integer \oplus \Integer/2$.
      Dies gilt aber nicht, wie man den folgenden Gründen entnehmen kann:
      \begin{itemize}
        \item
          Dies würde dem Hauptsatz über endlich erzeugte abelsche Gruppen widersprechen.
        \item
          $\Integer/2$ wäre isomorph zu einer Untergruppe von $\Integer$ und somit torsionsfrei (denn $\Integer$ ist frei und somit auch torsionsfrei, und Untergruppen von torsionsfreien abelschen Gruppen sind ebenfalls torsionsfrei), aber $2 \cdot \Integer/2 = 0$.
        \item
          $\Integer/2$ wäre isomorph zu einer Untergruppe von $\Integer$, und müsste somit entweder trivial oder unendlich sein, was beides nicht gilt.
      \end{itemize}
    \item
      Es sei $(e_i)_{i \in I}$ eine Basis von $F$.
      Wegen der Surjektivität von $g$ gibt es für jedes $i \in I$ ein $m_i \in M$ mit $g(m_i) = e_i$.
      Es sei $h \colon F \to M$ der eindeutige Homomorphismus von $R$-Moduln mit $h(e_i) = m_i$ für alle $i \in I$.
      Dann gilt $g(h(e_i)) = g(m_i) = e_i$ für alle $i \in I$, und wegen der $R$-Linearität von $g \circ h$ somit bereits $g(h(x)) = x$ für alle $x \in F$.
      Also ist $g \circ h = \id_F$, weshalb die gegebene kurze exakte Sequenz spaltet.
  \end{enumerate}
\end{solution}


\begin{question}
  Es sei $R$ ein Ring und $\{ N_i \xrightarrow{f_i} M_i \xrightarrow{g_i} P_i \}_{i \in I}$ eine Familie von exakten Sequenzen von $R$-Moduln.
  \begin{enumerate}
    \item
      Zeigen Sie, dass auch die induzierte Sequenz
      \[
                            \prod_{i \in I} N_i
        \xlongrightarrow{f} \prod_{i \in I} M_i
        \xlongrightarrow{g} \prod_{i \in I} P_i
      \]
      exakt ist, wobei $f$ und $g$ durch $f((n_i)_{i \in I}) = (f(n_i))_{i \in I}$ für alle $(n_i)_{i \in I} \in \prod_{i \in I} N_i$ und $g((m_i)) = (g(m_i))_{i \in I}$ für alle $(m_i)_{i \in I} \in \prod_{i \in I} M_i$ gegeben sind.
    \item
      Entscheiden Sie, ob die Aussage auch gilt, wenn man das Produkt $\prod_{i \in I}$ jeweils durch die direkte Summe $\bigoplus_{i \in I}$ ersetzt.
  \end{enumerate}
\end{question}


\begin{solution}
  \begin{enumerate}
    \item
      Es gilt $g \circ f = (g_i)_{i \in I} \circ (f_i)_{i \in I} = (g_i \circ f_i)_{i \in I} = 0$, da $g_i \circ f_i = 0$ für alle $i \in I$.
      Also ist $\im f \subseteq \ker g$.
      
      Ist andererseits $(m_i)_{i \in I} \in \ker g$, so gilt $0 = g((m_i)_{i \in i}) = (g(m_i))_{i \in I}$, also $g(m_i) = 0$ für alle $i \in I$.
      Dann ist $m_i \in \ker g_i = \im f_i$ für alle $i \in I$, weshalb es für jedes $i \in I$ ein $n_i \in N_i$ mit $f_i(n_i) = m_i$ gibt.
      Für $n \coloneqq (n_i)_{i \in I} \in \prod_{i \in I} N_i$ gilt dann $f(n) = f((n_i)_{i \in I}) = (f(n_i))_{i \in I} = (m_i)_{i \in I} = m$, weshalb $m \in \im f$.
      Das zeigt, dass $\ker g \subseteq \im f$.
    \item
      Die Aussage gilt auch weiterhin.
      Der obige Beweis muss nur bei der Wahl der $n_i$ etwas angepasst werden:
      Damit $(n_i) \in \bigoplus_{i \in I} N_i$ gilt, muss $n_i = 0$ für fast alle $i \in I$ gelten.
      Da aber ohnehin $m_i = 0$ für fast alle $i \in I$ gilt, lassen sich fast alle $n_i$ als $0$ wählen.
  \end{enumerate}
\end{solution}


\begin{question}
  \label{question: annihilators of quotients}
  Es sei $R$ ein kommutativer Ring.
  \begin{enumerate}
    \item
      Zeigen Sie für jedes Ideal $I \subseteq R$ die Gleichheit $\annihilator(R/I) = I$.
    \item
      Zeigen Sie für jeden freien $R$-Modul $F$ mit $F \neq 0$, dass $\annihilator(F) = 0$.
  \end{enumerate}
  Es sei nun zusätzlich $R \neq 0$.
  \begin{enumerate}[resume]
    \item
      Folgern Sie, dass $R$ genau dann ein Körper ist, wenn jeder endlich erzeugte $R$-Modul frei ist.
  \end{enumerate}
\end{question}


\begin{solution}
  \begin{enumerate}
    \item
      Für alle $x \in I$ und $\overline{y} \in R/I$ gilt $xy \in I$ und somit $\overline{x} \overline{y} = \overline{xy} = 0$.
      Deshalb gilt $I \subseteq \annihilator(R/I)$.
      Für jedes $x \in \annihilator(R/I)$ gilt $0 = x \cdot \overline{1} = \overline{x}$ und somit $x \in I$.
      Deshalb ist auch $\annihilator(R/I) \subseteq I$.
    \item
      Da $F$ frei ist, besitzt $F$ eine Basis; da $F \neq 0$ gilt, ist diese nicht leer.
      Es gibt daher ein Element $m \in F$, so dass $\{m\} \subseteq F$ linear unabhängig ist.
      Dann gilt $xm \neq 0$ für alle $x \in R$ mit $x \neq 0$, und deshalb $x \notin \annihilator(F)$.
    \item
      Ist $R$ ein Körper, so besitzt jeder endlich erzeugte erzugte $R$-Modul, also endlicher erzeugte $R$-Vektorraum, bekanntermaßen eine Basis.
      Ist andererseits $R$ kein Körper, so gibt es ein Ideal $I \subseteq R$ mit $I \neq 0, R$ (siehe Übung~\ref{question: characterization of fields via its ideals}).
      Dann ist $M \coloneqq R/I$ ein endlich erzeugter $R$-Modul mit $M \neq 0$ (da $I \neq R$) sowie $\annihilator(R/I) = I \neq 0$.
      Nach dem vorherigen Aufgabenteil ist $M$ nicht frei.
  \end{enumerate}
\end{solution}


\begin{question}
  Es sei $R$ ein Ring und $0 \to N \to M \to P \to 0$ eine kurze exakte Sequenz von $R$-Moduln.
  \begin{enumerate}
    \item
      Es sei zunächst $R$ ein Hauptidealbereich und $p \in R$ prim.
      Zeigen Sie, dass $M$ genau dann $p$-primär ist, wenn $N$ und $P$ beide $p$-primär sind.
    \item
      Zeigen Sie allgemeiner, dass $\radicalideal{\annihilator(M)} = \radicalideal{\annihilator(N)} \cap \radicalideal{\annihilator(P)}$.
      (Dabei gilt die Gleichheit $\radicalideal{\annihilator(N)} \cap \radicalideal{\annihilator(P)} = \radicalideal{\annihilator(N) \cap \annihilator(P)}$, siehe Übung~\ref{question: radical ideals}.
  \end{enumerate}
\end{question}


\begin{solution}
  Wir bezeichnen die gegebene kurze exakte Sequenz mit $0 \to N \xrightarrow{f} M \xrightarrow{g} P \to 0$.
  \begin{enumerate}
    \item
      Es sei zunächst $M$ $p$-primär.
      Dann ist auch der Untermodul $f(N) \subseteq M$ $p$-primär, und da $f$ eine Isomorphie $N \cong f(N)$ liefert, damit auch $N$ $p$-primär.
      Für $x \in P$ gibt es $x' \in M$ mit $g(x') = x$; da $M$ $p$-primär ist, gibt es ein $k \geq 0$ mit $p^k x' = 0$ und somit auch $p^k x = p^k g(x') = g(p^k x') = g(0) = 0$.
      Also ist auch $P$ $p$-primär.
      
      Es seien nun $N$ und $P$ beide $p$-primär.
      Für $m \in M$ ist $g(m) \in P$, es gibt also $k_2 \geq 0$ mit $p^{k_2} g(m) = 0$.
      Dann gilt $g(p^{k_2} m) = p^{k_2} g(m) = 0$ und somit $p^{k_2} m \in \ker g = \im f$.
      Es gibt also ein $n \in N$ mit $p^{k_2} m = f(n)$.
      Da $N$ $p$-primär ist, gibt es ein $k_1 \geq 0$ mit $p^{k_1} n = 0$.
      Ingesamt erhalten wir, dass
      \[
          p^{k_1 + k_2} m
        = p^{k_1} p^{k_2} m
        = p^{k_1} f(n)
        = f(p^{k_1} n)
        = f(0)
        = 0.
      \]
      Das zeigt, dass $M$ $p$-primär ist.
    \item
      Für $x \in R$ gilt genau dann $x \in \radicalideal{\annihilator(M)}$, wenn es $k \geq 0$ mit $x^k \in \annihilator(M)$ gibt, also genau dann, wenn es $k \geq 0$ mit $x^k M = 0$ gibt.
      Es gilt also genau dann $x \in \radicalideal{\annihilator(M)}$, wenn $M$ „$x$-primär“ ist.
      Ersetzt man in der obigen Argumentation $p$ durch $x$, so erhält man deshalb, dass genau dann $x \in \radicalideal{\annihilator(M)}$ gilt, wenn $x \in \radicalideal{\annihilator(N)}$ und $x \in \radicalideal{\annihilator(P)}$ gelten.
  \end{enumerate}
\end{solution}


\begin{question}
  Es sei $M$ ein $R$-Modul.
  \begin{enumerate}
    \item
      Zeigen Sie, dass $\annihilator( \generate{m}_R ) = \annihilator(m)$ für jedes $m \in M$.
    \item
      Zeigen Sie, dass $\annihilator( \sum_{i \in I} M_i) = \bigcap_{i \in I} \annihilator(M_i)$ für jede Familie $(M_i)_{i \in I}$ von Untermoduln $M_i \subseteq M$.
    \item
      Folgern Sie, dass $\annihilator( \generate{m_i \mid i \in I}_R ) = \bigcap_{i \in I} \annihilator(m_i)$ für jede Familie $(m_i)_{i \in I}$ von Elementen $m_i \in M$, und dass $\annihilator(M) = \bigcap_{m \in M} \annihilator(m)$.
    \item
      Zeigen Sie, dass $\sum_{i \in I} \annihilator(M_i) \subseteq \annihilator( \bigcap_{i \in I} M_i)$ für jede Familie $(M_i)_{i \in I}$ von Untermoduln $M_i \subseteq M$.
    \item
      Geben Sie ein Beispiel an, in dem die obige Inklusion strikt ist.
  \end{enumerate}
\end{question}


\begin{solution}
  \begin{enumerate}
    \item
      Aus der Inklusion $\{m\} \subseteq \generate{m}_R$ folgt die Inklusion $\annihilator( \generate{M}_R ) \subseteq \annihilator(m)$.
      Ist andererseits $x \in \annihilator(m)$, so gilt $xm = 0$, und somit auch $x(rm) = r(xm) = 0$ für alle $r \in R$, also $xm' = 0$ für alle $m' \in \{rm \mid r \in R\} = \generate{m}_R$.
    \item
      Für jedes $j \in J$ folgt sich aus $M_j \subseteq \sum_{i \in I} M_i$ die Inklusion $\annihilator( \sum_{i \in I} M_i ) \subseteq \annihilator(M_j)$, und somit insgesamt die Inklusion $\annihilator( \sum_{i \in I} M_i ) \subseteq \bigcap_{i \in I} \annihilator(M_i)$.
      Gilt andererseits $x \in \bigcap_{i \in I} \annihilator(M_i)$, so gilt $x M_i = 0$ für alle $i \in I$, also $x m_i = 0$ für alle $i \in I$ und $m_i \in M_i$.
      Da jedes $m \in \sum_{i \in I} M_i$ eine endliche Summe solcher $m_i$ ist, gilt bereits $xm = 0$ für alle $m \in \sum_{i \in I} M_i$, und somit $x \in \annihilator( \sum_{i \in I} M_i )$.
      Somit gilt auch $\bigcap_{i \in I} \annihilator( M_i ) \subseteq \annihilator( \sum_{i \in I} M_i )$.
    \item
      Es gilt die Gleichungskette
      \[
          \annihilator( \generate{m_i \mid i \in I}_R )
        = \annihilator\left( \sum_{i \in I} \generate{m_i}_R \right)
        = \bigcap_{i \in I} \annihilator( \generate{m_i}_R )
        = \bigcap_{i \in I} \annihilator(m_i)
      \]
      und somit insbesondere $\annihilator(M) = \annihilator(\sum_{m \in M} \generate{m}_R) = \bigcap_{m \in M} \annihilator(m)$.
    \item
      Es gilt genau dann $\sum_{i \in I} \annihilator(M_i) \subseteq \annihilator( \bigcap_{i \in I} M_i )$, wenn $\annihilator(M_j) \subseteq \annihilator( \bigcap_{i \in I} M_i )$ für alle $j \in I$.
      Für jedes $j \in I$ ergibt sich diese Inklusion aus der Inklusion $\bigcap_{i \in I} M_i \subseteq M_j$.
    \item
      Für $R \neq 0$ betrachte man $M = R \oplus R$ mit den Untermoduln $M_1 = R \oplus 0$ und $M_2 = 0 \oplus R$.
      Dann gilt $M_1 \cong M_2 \cong R$ und somit $\annihilator(M_1) = \annihilator(M_2) = \annihilator(R) = 0$.
      Deshalb gilt $\annihilator(M_1) + \annihilator(M_2) = 0$.
      Andererseits gilt $M_1 \cap M_2 = 0$ und somit $\annihilator(M_1 \cap M_2) = \annihilator(0) = R \neq 0$.
  \end{enumerate}
\end{solution}


\begin{question}
  Es sei $R$ ein kommutativer Ring und $I, J \subseteq R$ seien zwei Ideale, so dass $R/I \cong R/J$ als $R$-Moduln.
  Zeigen Sie, dass bereits $I = J$.
  (\emph{Hinweis}:
   Betrachten Sie Annihilatoren.)
\end{question}


\begin{solution}
  Für jedes Ideal $K \subseteq R$ gilt $\annihilator(R/K) = K$ (siehe Übung~\ref{question: annihilators of quotients}), und somit gilt
  \[
      I
    = \annihilator(R/I)
    = \annihilator(R/J)
    = J.
  \]
\end{solution}


\begin{question}[subtitle = Existenz der Hauptraumzerlegung]
  Es sei $K$ ein Körper, $V$ ein $K$-Vektorraum und $f \colon V \to V$ ein Endomorphismus.
  \begin{enumerate}
    \item
      Zeigen Sie für je zwei teilerfremde Polynome $p, q \in K[T]$ die Gleichheit
      \[
        \ker {(pq)(f)} = \ker p(f) \oplus \ker q(f).
      \]
    \item
      Folgern Sie für alle paarweise teilerfremden Polynome $p_1, \dotsc, p_n \in K[T]$ die Gleichheit
      \[
          \ker {(p_1 \dotsm p_n)(f)}
        = \ker p_1(f) \oplus \dotsb \oplus \ker p_n(f).
      \]
  \end{enumerate}
  Es sei nun $K$ algebraisch abgeschlossen und $V$ endlichdimensional.
  \begin{enumerate}[resume]
    \item
      Zeigen Sie, dass $V$ in die Summe der Haupträume von $f$ ist.
  \end{enumerate}
\end{question}


\begin{solution}
  \begin{enumerate}
    \item
      Da $p$ und $q$ teilerfremd sind, gibt es $a, b \in K[T]$ mit $ap + bq = 1$.
      Durch Einsetzen von $f$ ergibt sich, dass $a(f) p(f) + b(f) g(f) = \id_V$.
      Für $v \in \ker p(f) \cap \ker q(f)$ gilt deshalb
      \[
          0
        =   a(f)( \underbrace{p(f)(v)}_{=0} ) + b(f)( \underbrace{q(f)(v)}_{=0} )
        = (a(f) p(f) + b(f) q(f))(v)
        = \id_V(v)
        = v.
      \]
      Also ist $\ker p(f) \cap \ker q(f) = 0$.
      Für $v \in \ker {(pq)(f)}$ gilt andererseits
      \[
          v
        = \id_V(v)
        = (a(f) p(f) + b(f) q(f))(v)
        = \underbrace{(a(f)p(f))(v)}_{\eqqcolon v_2} + \underbrace{(b(f)q(v))(v)}_{\eqqcolon v_1}
      \]
      Dabei gilt $v_1 \in \ker p(f)$, da
      \begin{align*}
            p(f)(v_1)
        &=  (p(f)b(f)q(f))(v)
         =  (b(f)p(f)q(f))(v)
        \\
        &=  (b(f)(pq)(f))(v)
         =  b(f)( \underbrace{(pq)(f)(v)}_{=0} )
         =  0.
      \end{align*}
      Analog gilt auch $v_2 \in \ker q(f)$.
      Also gilt auch $\ker {(pq)(f)} = \ker p(f) + \ker q(f)$.
      
    \item
      Wir zeigen die Aussage per Induktion über $n$.
      Für $n = 1$ ist die Aussage klar, und für $n = 2$ wurde sie im vorherigen Teil der Aussage gezeigt.
%       
      Es sei also $n \geq 3$; da $f_1, \dotsc, f_n$ paarweise teilerfremd sind, sind auch die beiden Polynome $f_1 \dotsm f_{n-1}$ und $f_n$ teilerfremd.
      Nach Induktionsvoraussetzung gilt deshalb
      \begin{equation}
        \label{equation: decomposition into two subspaces}
          \ker {(p_1 \dotsm p_n)(f)}
        = \ker {(p_1 \dotsm p_{n-1} \cdot p_n)(f)}
        = \ker {(p_1 \dotsm p_{n-1})(f)} \oplus \ker p_n(f).
      \end{equation}
      Da $f_1, \dotsc, f_n$ paarweise teilerfremd sind, sind es auch $f_1, \dotsc, f_{n-1}$.
      Nach Induktionsvoraussetzung gilt daher auch
      \begin{equation}
        \label{equation: decomposition of the first summand into subspaces}
          \ker {(p_1 \dotsm p_{n-1})(f)}
        = \ker p_1(f) \oplus \dotsb \oplus \ker p_{n-1}(f).
      \end{equation}
      Zusammenfügen von \eqref{equation: decomposition into two subspaces} und \eqref{equation: decomposition of the first summand into subspaces} ergibt die Aussage.
    \item
      Es sei $p \in K[T]$ das charakteristische Polynom von $f$.
      Da $K$ algebraisch abgeschlossen ist zerfällt $p$ in Linearfaktoren, also $p(T) = (T - \lambda_1)^{n_1} \dotsm (T - \lambda_s^{n_s})$ für $n_1, \dotsc, n_s \geq 0$ und paarweise verschiedene $\lambda_1, \dotsc, \lambda_s \in K$.
      Die Polynome $(T-\lambda_1)^{n_1}, \dotsc, (T-\lambda_s)^{n_s}$ sind paarweise teilerfremd, und somit gilt nach dem vorherigen Aussagenteil
      \[
          \ker p(f)
        = \ker (f - \lambda_1)^{n_1} \oplus \dotsb \oplus \ker (f - \lambda_s)^{n_s}.
      \]
      Nach dem Satz von Cayley-Hamilton gilt dabei $p(f) = 0$ und somit $\ker p(f) = V$.
  \end{enumerate}
\end{solution}



\begin{question}[subtitle=Torsionsuntermoduln]
  Es sei $R$ ein Integritätsbereich mit Quotientenkörper $K$.
  \begin{enumerate}
    \item
      Definieren Sie den Torsionsuntermodul $\torsion(M)$ eines $R$-Moduls $M$, und zeigen Sie, dass es sich um einen $R$-Untermodul von $M$ handelt.
    \item
      Zeigen Sie, dass $\torsion(M)$ der Kern der kanonischen Abbildung $M \to M_K$, $m \mapsto m/1$ ist.
    \item
      Zeigen Sie für jeden $R$-Moduln $M$, dass $\torsion(M \oplus N) = \torsion(M) \oplus \torsion(N)$ für alle $R$-Moduln $M$ und $N$.
    \item
      Zeigen Sie, dass jeder freie $R$-Modul torsionsfrei ist.
    \item
      Zeigen Sie für jeden $R$-Moduln $M$, dass $M/\torsion(M)$ torsionsfrei ist.
    \item
      Es sei $f \colon M \to N$ ein $R$-Modulhomomorphismus.
      Zeigen Sie, dass $f(\torsion(M)) \subseteq \torsion(N)$.
  \end{enumerate}
  Wir bezeichnen die Einschränkung von $f \colon M \to N$ auf die entsprechenden Torsionsuntermoduln mit $\torsion(f) \colon \torsion(M) \to \torsion(N)$, $m \mapsto f(m)$.
  \begin{enumerate}[resume]
    \item
      Zeigen Sie, dass
      \begin{enumerate}[leftmargin=*]
        \item
          $\torsion(\id_M) = \id_{\torsion(M)}$ für jeden $R$-Modul $M$, und
        \item
          $\torsion(g \circ f) = \torsion(g) \circ \torsion(f)$ für alle $R$-Modulhomomorphismen $N \xrightarrow{f} M \xrightarrow{g} P$.
      \end{enumerate}
    \item
      Zeigen Sie für jede exakte Sequenz von $R$-Moduln $0 \to N \xrightarrow{f} M \xrightarrow{g} P \to 0$ die Exaktheit der induzierten Sequenz
      \[
                                  0
        \to                       \torsion(N)
        \xrightarrow{\torsion(f)} \torsion(M)
        \xrightarrow{\torsion(g)} \torsion(P).
      \]
    \item
      Geben Sie ein Beispiel für einen surjektiven $R$-Modulhomomorphismus $g \colon M \to P$ an, so dass $T(g)$ nicht surjektiv ist.
  \end{enumerate}
\end{question}


\begin{solution}
  \begin{enumerate}
    \item
      Es ist $\torsion(M) = \{m \in M \mid \text{$rm = 0$ für ein $r \in R$ mit $r \neq 0$}\}$.
      Es gilt $0 \in M$, da $1 \cdot 0 = 0$.
      
      Für $m_1, m_2 \in \torsion(M)$ gibt es $r_1, r_2 \in R$ mit $r_1, r_2 \neq 0$, so dass $r_1 m_1 = r_2 m_2 = 0$.
      Dann gilt auch $(r_1 r_2) (m_1 + m_2) = r_2 r_1 m_1 + r_1 r_2 m_2 = 0$, wobei $r_1 r_2 \neq 0$, da $R$ ein Integritätsbereich ist.
      Also gilt auch $m_1, m_2 \in M$.
      
      Für $m \in \torsion(M)$ gibt es $r \in R$ mit $rm = 0$ und $r \neq 0$.
      Für jedes $r' \in R$ ist dann auch $r (r' m) = r' (rm) = 0$, und somit $r' m \in \torsion(M)$.
      
    \item
      Es ist $K$ die Lokalisierung von $R$ an der multiplikativen Menge $S = R \setminus \{0\}$.
      Für die kanonische Abbildung $i \colon M \to M_K$, $m \mapsto m/1$ gilt deshalb
      \[
              m \in \ker i
        \iff  \frac{m}{1} = \frac{0}{1}
        \iff  \exists s \in S : s \cdot m = 0
        \iff  m \in \torsion(M).
      \]

    \item
      Für $(m, n) \in \torsion(M)$ gibt es $r \in R$ mit $r \neq 0$ und $0 = r(m,n) = (rm,rn)$.
      Dann gilt $rm = 0$ un $rn = 0$, und wegen $r \neq 0$ gelten somit $m \in \torsion(M)$ und $n \in \torsion(N)$.
      Also gilt $\torsion(M \oplus N) \subseteq \torsion(M) \oplus \torsion(N)$.
      
      Für $(m, n) \in \torsion(M) \oplus \torsion(N)$ gibt es $r_1, r_2 \in R$ mit $r_1, r_2 \neq 0$ und $r_1 m = 0$ und $r_2 n = 0$.
      Dann gelten $(r_1 r_2) (m, n) = (r_2, r_1 m, r_1 r_2 n) = (0,0)$, und da $R$ ein Integritätsbereich ist, gilt $r_1 r_2 \neq 0$.
      Also ist $(m,n) \in \torsion(M \oplus N)$, und somit $\torsion(M) \oplus \torsion(N) \subseteq \torsion(M \oplus N)$.
      
%       Andererseits gilt ergibt sich für die beiden Untermoduln $M \oplus 0, 0 \oplus N \subseteq M \oplus N$, dass $\torsion(M \oplus 0), \torsion(0 \oplus N) \subseteq \torsion(M \oplus N)$.
%       Der Isomorphismus $M \oplus 0 \to M$, $(m, 0) \mapsto m$ schränkt sich zu einem Isomorphismus $\torsion(M \oplus 0) \to \torsion(M)$ ein;
%       dabei entspricht aber der Untermodul $\torsion(M) \subseteq M$ dem Untermodul $\torsion(M) \oplus 0 \subseteq M \oplus 0$.
%       Es folgt, dass $\torsion(M \oplus 0) = \torsion(M) \oplus 0$;
%       analog ergibt sich, dass $\torsion(0 \oplus N) = 0 \oplus \torsion(N)$ gilt.
%       Somit gelten $\torsion(M) \oplus 0, 0 \oplus \torsion(N) \subseteq \torsion(M \oplus N)$ und deshalb auch
%       \[
%                   \torsion(M) \oplus \torsion(N)
%         =         (\torsion(M) \oplus 0) + (0 \oplus \torsion(N))
%         \subseteq \torsion(M \oplus N)
%       \]
    
    \item
      Es sei $F$ ein freier $R$-Modul mit Basis $(b_i)_{i \in I}$ und $m \in \torsion(F)$.
      Dann gibt es eine (eindeutige) Darstellung $m = \sum_{i \in I} r_i b_i$ mit $r_i = 0$ für fast alle $i \in I$.
      Nach Annahme gibt es $r \in R$ mit $r \neq 0$ und $0 = rm = \sum_{i \in I} r r_i b_i$.
      Wegen der linearen Unabhängigkeit der Familie $(b_i)_{i \in I}$ muss bereits $r r_i = 0$ für alle $i \in I$.
      Da $R$ ein Integritätsbereich ist folgt zusammen mit $r \neq 0$, dass $r_i = 0$ für alle $i \in I$.
      Somit ist bereits $m = \sum_{i \in I} r_i b_i = 0$.
      
    \item
      Es sei $\overline{m} \in \torsion(M/\torsion(M))$.
      Dann gibt es $r_2 \in R$ mit $r_2 \neq 0$ und $0 = r_2 \overline{m} = \overline{r_2 m}$, also $r_1 m \in \torsion(M)$.
      Dann gibt es wiederum $r_1 \in R$ mit $r_1 \neq 0$ und $r_1 r_2 m = 0$. 
      Da $R$ ein Integritätsbereich ist, gilt dabei $r_1 r_2 \neq 0$.
      Deshalb gilt bereits $m \in \torsion(M)$ und somit $\overline{m} = 0$.
      
    \item
      Es sei $m \in \torsion(M)$.
      Dann gibt es $r \in R$ mit $r \neq 0$ und $rm = 0$.
      Dann gilt auch $r f(m) = f(rm) = f(0) = 0$ und somit $f(m) \in \torsion(N)$.
    
    \item
      Dass $\torsion(\id_M) = \id_{\torsion(M)}$ gilt ist klar, und dass $\torsion(g \circ f) = \torsion(g) \circ \torsion(f)$ gilt, folgt aus der Veträglichkeit von Komposition und Restriktion.
    
    \item
      Wegen der Injektivität von $f$ ist auch $\torsion(f)$ injektiv, die Sequenze also exakt an $\torsion(N)$.
      Für die Exaktheit an $\torsion(M)$ bemerken wir zunächst, dass $\torsion(g) \circ \torsion(f) = \torsion(g \circ f) = \torsion(0) = 0$, und somit $\im \torsion(f) \subseteq \ker \torsion(g)$.
      Ist andererseits $m \in \ker \torsion(g) = \ker g \cap \torsion(M)$, so ist $m \in \ker g = \im f$, weshalb es ein $n \in N$ mit $m = f(n)$.
      Da $m \in \torsion(M)$ gilt, gibt es $r \in R$ mit $r \neq 0$ und $0 = rm = r f(n) = f(rn)$.
      Wegen der Injektivität von $f$ gilt bereits $rn = 0$ und somit $n \in \torsion(N)$.
      Also gilt bereits $m = f(n) = \torsion(f)(n) \in \im \torsion(f)$ und somit $\ker \torsion(g) \subseteq \im \torsion(f)$.
    
    \item
      Wir betrachten das folgende Gegenbeispiel von $\Integer$—Moduln:
      Die kanonische Projektion $p \colon \Integer \to \Integer/2$, $n \mapsto \overline{n}$ ist zwar surjektiv, die induzierte Abbildung
      \[
                                  0
        =                         \torsion(\Integer)
        \xrightarrow{\torsion(p)} \torsion(\Integer/2)
        =                         \Integer/2
      \]
      kann es aber nicht sein. (Dass $\torsion(\Integer) = 0$ folgt daraus, dass $\Integer$ als freier $\Integer$-Modul torsionfrei ist.)
  \end{enumerate}
\end{solution}


\begin{question}
  Zeigen Sie, dass für jeden $R$-Moduln $M$ die folgenden Bedingungen äquivalent sind:
  \begin{enumerate}
    \item
      \label{enum: module is cyclic}
      $M$ wird von einem einzelnen Element erzeugt, d.h.\ es gibt $m \in M$ mit $M = \generate{m}_R$.
    \item
      \label{enum: module is quotient by its annihilator}
      Es gilt $M \cong R/\!\annihilator(M)$ als $R$-Moduln.
    \item
      \label{enum: module is quotient by an ideal}
      Es gibt ein Ideal $I \subseteq R$ mit $R/I \cong M$ als $R$-Moduln.
  \end{enumerate}
  Erfüllt $M$ eine (und damit alle) dieser Bedingungen, so heißt $M$ \emph{zyklisch}.
\end{question}


\begin{solution}
  (
    \ref{enum: module is cyclic}
    $\implies$
    \ref{enum: module is quotient by its annihilator}
  )
  Es sei $m \in M$ mit $M = \generate{m}_R$.
  Dann gilt
  \[
      \annihilator(M)
    = \annihilator(\generate{m}_R)
    = \annihilator(m)
    = \{r \in R \mid rm = 0\}.
  \]
  Für den surjektive Homomorphismus von $R$-Moduln
  \[
            \varphi
    \colon  R \to M,
    \quad
            r \mapsto rm
  \]
  gilt deshalb $\ker \varphi = \annihilator(M)$.
  Somit induziert $\varphi$ einen Isomorphismus von $R$-Moduln
  \[
            \bar{\varphi}
    \colon  R/\annihilator(M) \to M,
    \quad
            [r] \mapsto rm.
  \]
  
  (
    \ref{enum: module is quotient by its annihilator}
    $\implies$
    \ref{enum: module is quotient by an ideal}
  )
  Man setze $I = \annihilator(M)$.
  
  (
    \ref{enum: module is quotient by an ideal}
    $\implies$
    \ref{enum: module is cyclic}
  )
  Ist $\varphi \colon R/I \to M$ ein Isomorphismus, so gilt
  \[
      M
    = \varphi(R/I)
    = \varphi(\generate{\overline{1}}_R)
    = \generate{\varphi(\overline{1})}_R.
  \]
\end{solution}


\begin{question}[subtitle = Schurs Lemma]
  Ein $R$-Modul $M$ heißt \emph{einfach}, wenn $M$ genau zwei Untermoduln hat.
  \begin{enumerate}
    \item
      Zeigen Sie, dass $M$ genau dann einfach ist, wenn $M \neq 0$ und $0, M \subseteq M$ die einzigen beiden Untermoduln sind.
    \item
      Zeigen Sie, dass für je zwei einfache $R$-Moduln $M$ und $N$ jeder $R$-Mo\-dul\-ho\-mo\-mor\-phis\-mus $f \colon M \to N$ entweder $0$ oder ein Isomorphismus ist.
  \end{enumerate}
\end{question}


\begin{solution}
  \begin{enumerate}
    \item
      Ist $M$ einfach, so muss $M \neq 0$, da $M$ sonst nur einen Untermodul hätte (nämlich sich selbst).
      Dann sind $0, M \subseteq M$ zwei verschiedene Untermoduln, und nach Annahme gibt es keine weiteren Untermoduln.
      
      Ist $M \neq 0$ und sind $0, M \subseteq M$ die einzigen beiden Untermoduln, so hat $M$ genau zwei Untermoduln.
    \item
      Ist $f \colon M \to N$ ein Homomorphismus von $R$-Moduln mit $f \neq 0$, so sind $\ker f \subseteq M$ und $\im f \subseteq N$ Untermoduln mit $\ker f \neq M$ und $\im f \neq 0$.
      Ist $M$ einfach, so muss bereits $\ker f = 0$ gelten, und $f$ somit bereits injektiv sein.
      Ist $N$ einfach, so muss bereits $\im f = N$ gelten, und $f$ somit bereits surjektiv sein.
      Sind $M$ und $N$ beide einfach, so ist $f$ also bereits ein Isomorphismus.
  \end{enumerate}
\end{solution}


\begin{remark*}
  Das Lemma von Schur besagt insbesondere, dass der Endomorphismenring eines einfachen Moduls ein Schiefkörper ist.
\end{remark*}


% \begin{question}
%   Ein $R$-Modul $M$ heißt \emph{unzerlegbar}, falls es keine Zerlegung $M = N_1 \oplus N_2$ in zwei echten Untermoduln $N_1, N_2 \subsetneq M$ gibt.
%   \begin{enumerate}
%     \item
%       Es sei $R$ ein Integritätsbereich.
%       Zeigen Sie, dass $R$ als $R$-Modul unzerlegbar ist.
%       
%       (\emph{Hinweis}: Zeigen Sie, dass $I \cap J \neq 0$ für alle Ideale $I, J \subseteq R$ mit $I, J \neq 0$.)
%     \item
%       Geben Sie ein Beispiel für einen Ring $R$, der zwar nicht nullteilerfrei ist, so dass aber $R$ als $R$-Modul dennoch unzerlegbar ist.
%     \item
%       Geben Sie ein Beispiel für einen Ring $R$, so dass $R$ als $R$-Modul nicht unzerlegbar ist.
%   \end{enumerate}
% \end{question}


\begin{question}[subtitle = Kürzungsregeln bis auf Isomorphie]
  Geben Sie einen jeweils passenden kommutativen Ring $R$ Beispiele für $R$-Moduln $M_1$ und $M_2$, sowie Untermoduln $N_1 \subseteq M_1$ und $N_2 \subseteq M_2$, so dass die folgenden Bedingungen erfüllt sind:
  \begin{enumerate}
    \item
      Es gilt $M_1 \cong M_2$ und $N_1 \cong N_2$, aber $M_1/N_1 \ncong M_2/N_2$.
    \item
      Es gilt $M_1 \cong M_2$ und $M_1/N_1 \cong M_2/N_2$, aber $N_1 \ncong N_2$.
    \item
      Es gilt $M_1/N_1 \cong M_2/N_2$ und $N_1 \cong N_2$, aber $M_1 \ncong M_2$.
  \end{enumerate}
  (\emph{Hinweis}:
   Betrachten Sie Moduln der Form $\bigoplus_{n \in \Natural} R$.)
\end{question}


\begin{solution}
  Für die ersten beiden Beispiel sei $R$ ein beliebiger kommutativer Ring mit $R \neq 0$.
  \begin{enumerate}
    \item
      Wir betrachten $M_1 = M_2 = \bigoplus_{n \geq 0} R = R \oplus R \oplus R \oplus \dotsb$ und die Untermoduln $N_1 = 0 \oplus \bigoplus_{n \geq 1} R = 0 \oplus R \oplus R \oplus \dotsb$ und $N_2 = 0 \oplus 0 \oplus \bigoplus_{n \geq 2} R = 0 \oplus 0 \oplus R \oplus \dotsb$;
      dann gilt $M_1 = M_2 \cong N_1 \cong N_2$, aber
      \[
        M_1/N_1 \cong R \ncong R^2 \cong M_2/N_2.
      \]
      (Hier nutzen wir, dass wegen $R \neq 0$ und der Kommutativität von $R$ der Rang eines freien $R$-Moduln wohldefiniert ist.)
    
    \item
      Wir betrachten erneut $M_1 = M_2 = \bigoplus_{n \geq 0} R$, dieses Mal mit den jeweiligen Untermoduln $N_1 = R \oplus \bigoplus_{n \geq 1} 0 = R \oplus 0 \oplus 0 \oplus \dotsb$ und $N_2 = R \oplus R \oplus \bigoplus_{n \geq 2} 0 = R \oplus R \oplus 0 \oplus 0 \oplus \dotsb$;
      dann gelten $M_1 = M_2$ und
      \[
              M_1/N_1
        \cong \bigoplus_{n \geq 1} R
        \cong \bigoplus_{n \geq 2} R
        \cong M_2/N_2,
      \]
      aber $N_1 \cong R \ncong R^2 \cong N_2$.
  \end{enumerate}
  Für das dritte Gegenbeispiel muss $R$ weiter eingeschränkt werden;
  ist etwa $R$ ein Körper, so folgt aus $N_1 \cong N_2$ und $M_1/N_1 \cong M_2/N_2$, dass bereits
  \[
    \dim M_1 = \dim M_1/N_1 + \dim N_1 = \dim M_2/N_2 + \dim N_2 = \dim M_2,
  \]
  und somit $M_1 \cong M_2$.
  Wir betrachten daher den Fall $R = \Integer$.
  \begin{enumerate}[resume]
    \item
      Es seien $M_1 = \Integer/4$, $M_2 = \Integer/2 \oplus \Integer/2$, $N_1 = \{0,2\} = 2 M_1$ und $N_2 = \Integer/2 \oplus 0$.
      Dann gelten $M_1/N_1 \cong \Integer/2 \cong M_2/N_2$ und $N_1 \cong \Integer/2 \cong N_2$ aber $M_1 \ncong M_2$.
  \end{enumerate}
\end{solution}


\begin{question}[subtitle = Isomorphie von kurzen exakten Sequenzen]
  \label{question: isomorpism of short exact sequences}
  Es sei $R$ ein Ring.
  Zwei kurze exakte Sequenzen
  \[
    0 \to N  \xrightarrow{f}  M  \xrightarrow{g}  P \to 0
    \quad\text{und}\quad
    0 \to N' \xrightarrow{f'} M' \xrightarrow{g'} P' \to 0
  \]
  von $R$-Moduln heißen \emph{isomorph}, wenn es Isomorphismen $\varphi_N \colon N \to N'$, $\varphi_M \colon M \to M'$ und $\varphi_P \colon P \to P'$ gibt, die das folgende Diagramm zu kommutieren bringen:
  \[
    \begin{tikzcd}[ampersand replacement = \&]
          0
          \arrow{r}
      \&  N
          \arrow{r}{f}
          \arrow{d}{\varphi_N}
      \&  M
          \arrow{r}{g}
          \arrow{d}{\varphi_M}
      \&  P
          \arrow{r}
          \arrow{d}{\varphi_P}
      \&  0
      \\
          0
          \arrow{r}
      \&  N'
          \arrow{r}{f'}
      \&  M'
          \arrow{r}{g'}
      \&  P'
          \arrow{r}
      \&  0
    \end{tikzcd}
  \]
  \begin{enumerate}
    \item
      Zeigen Sie, dass Isomorphie von kurzen exakten Sequenzen eine Äquivalenzrelation auf der Klasse der kurzen exakten Sequenzen von $R$-Moduln ist.
    \item
      Es sei $0 \to N \xrightarrow{f} M \xrightarrow{g} P \to 0$ eine kurze exakte Sequenz von $R$-Moduln.
      Zeigen Sie, dass sich das Diagramm
      \[
        \begin{tikzcd}[ampersand replacement = \&]
              0
              \arrow{r}
          \&  N
              \arrow{r}{f}
          \&  M
              \arrow{r}{g}
              \arrow[equal]{d}
          \&  P
              \arrow{r}
          \&  0
          \\
              0
              \arrow{r}
          \&  \im f
              \arrow{r}{i}
          \&  M
              \arrow{r}{p}
          \&  P
              \arrow{r}
          \&  0
        \end{tikzcd}
      \]
      zu einem Isomorphismus von kurzen exakten Sequenzen ergänzen lässt.
      Dabei bezeichnet $i \colon \im f \to M$, $x \mapsto x$ die Inklusion und $p \colon M \to M / \im f$, $x \mapsto \overline{x}$ die kanonische Projektion.
      
      (Mit anderen Worten:
       Die beiden Zeilen im obigen Diagramm sind isomorphe kurze exakte Sequenzen, und es gibt einen Isomorphismus, dessen mittlerer vertikaler Pfeil die Identität ist.)
  \end{enumerate}
\end{question}


% TODO: Adding a solution.


\begin{remark*}
  Übung~\ref{question: isomorpism of short exact sequences} gibt eine formale Begründung der informalen Aussage, dass jede kurze exakte Sequenz von $R$-Moduln von der Form $0 \to N \to M \to M/N \to 0$ für einen $R$-Modul $M$ und Untermodul $N \subseteq M$ ist (wobei die Pfeile $N \to M$ und $M \to M/N$ die jeweils kanonischen Homomorphismen sind).
  % TODO: Extending this remark when the different definitions of split exact sequences are discussed.
\end{remark*}


\begin{question}[subtitle = Zwei Vierer- und ein Fünferlemma]
  \label{question: the five lemma}
  Es sei $R$ ein Ring und
  \[
    \begin{tikzcd}[ampersand replacement=\&]
          M_1
          \arrow{r}{f_1}
          \arrow{d}{h_1}
      \&  M_2
          \arrow{r}{f_2}
          \arrow{d}{h_2}
      \&  M_3
          \arrow{r}{f_3}
          \arrow{d}{h_3}
      \&  M_4
          \arrow{r}{f_4}
          \arrow{d}{h_4}
      \&  M_5
          \arrow{d}{h_5}
      \\
          N_1
          \arrow{r}{g_1}
      \&  N_2
          \arrow{r}{g_2}
      \&  N_3
          \arrow{r}{g_3}
      \&  N_4
          \arrow{r}{g_4}
      \&  N_5
    \end{tikzcd}
  \]
  ein kommutatives Diagramm von $R$-Moduln mit exakten Ze
  \begin{enumerate}
    \item
      Es sei $h_1$ surjektiv, und $h_2$ und $h_4$ seien injektiv.
      Zeigen Sie, dass auch $h_3$ injektiv ist.
    \item
      Es sei $h_5$ injektiv, und $h_2$ und $h_4$ seien surjektiv.
      Zeigen Sie, dass auch $h_3$ surjektiv ist.
    \item
      Folgern Sie:
      Sind $h_1$, $h_2$, $h_4$ und $h_5$ Isomorphismen, so ist auch $h_3$ ein Isomorphismus.
  \end{enumerate}
\end{question}


% TODO: Adding a solution.


% \begin{question}[subtitle = Anwendungen des Fünferlemmas]
%   Dies ist eine Fortsetzung von Übung~\ref{question: the five lemma}.
%   Es sei $R$ ein Ring.
%   \begin{enumerate}
%     \item
%       Es seien nun $N$ und $P$ zwei $R$-Moduln.
%       Eine \emph{Erweiterung von $N$ um $P$} ist eine kurze exakte Sequenz der Form $0 \to N \to M \to P \to 0$.
%       Zwei Erweiterungen
%       \[
%         0 \to N \xrightarrow{f} M \xrightarrow{g} P \to 0
%         \quad\text{und}\quad
%         0 \to N \xrightarrow{f'} M' \xrightarrow{g'} P \to 0
%       \]
%       heißen \emph{Yoneda-äquivalent}, falls es einen Homomorphismus $h \colon M \to M'$ gibt, der das folgende Diagram zum Kommutieren bringt:
%       \[
%         \begin{tikzcd}[ampersand replacement = \&]
%               N
%               \arrow{r}{f}
%               \arrow[equal]{d}
%           \&  M
%               \arrow{r}{g}
%               \arrow{d}{h}
%           \&  P
%               \arrow[equal]{d}
%           \\
%               N
%               \arrow{r}{f'}
%           \&  M'
%               \arrow{r}{g'}
%           \&  P
%         \end{tikzcd}
%       \]
% %       \[
% %         \begin{tikzcd}[ampersand replacement = \&]
% %               0
% %               \arrow{r}
% %           \&  N
% %               \arrow{r}{f}
% %               \arrow[equal]{d}
% %           \&  M
% %               \arrow{r}{g}
% %               \arrow{d}{h}
% %           \&  P
% %               \arrow{r}
% %               \arrow[equal]{d}
% %           \&  0
% %           \\
% %               0
% %               \arrow{r}
% %           \&  N
% %               \arrow{r}{f'}
% %           \&  M'
% %               \arrow{r}{g'}
% %           \&  P
% %               \arrow{r}
% %           \&  0
% %         \end{tikzcd}
% %       \]
%       Zeigen Sie, dass Yoneda-Äquivalenz eine Äquivalenzrelation auf der Klasse der Erweiterungen von $N$ um $P$ ist.
%   \end{enumerate}
% \end{question}


% TODO: Adding a solution.