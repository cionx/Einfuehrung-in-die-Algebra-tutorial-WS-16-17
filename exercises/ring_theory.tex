\section{Ringtheorie}


\begin{question}[subtitle = Initialobjekte in der Kategorie der Ringe]
  \begin{enumerate}
    \item
      Zeigen Sie, dass es für jeden Ring $R$ einen eindeutigen Ringhomomorphismus $\Integer \to R$ gibt.
      (Dies bedeutet, dass der Ring $\Integer$ ein Initialobjekt in der Kategorie der Ringe ist.)
    \item
      Es sei $Z$ ein Ring, so dass es für jeden Ring $R$ einen eindeutigen Ringhomomorphismus $Z \to R$ gibt.
      Zeigen Sie, dass $Z \cong \Integer$.
  \end{enumerate}
\end{question}


\begin{solution}
  \begin{enumerate}
    \item
      Ist $\phi \colon \Integer \to R$ ein Ringhomomorphismus, so ist $\phi(1_\Integer) = 1_R$.
      Für alle $n \in \Integer$ ist damit
      \[
          \phi(n)
        = \phi(n \cdot 1_\Integer)
        = n \cdot \phi(1_\Integer)
        = n \cdot 1_R.
      \]
      Also ist $\phi$ eindeutig.
      Durch direktes Nachrechnen ergibt sich auch, dass $\psi \colon \Integer \to R$ mit
      \[
        \psi(n) \coloneqq n \cdot 1_R
        \quad
        \text{für alle $n \in \Integer$}
      \]
      ein Ringhomomorphismus ist.
    \item
      Es gibt einen eindeutigen Ringhomomorphismus $\phi \colon \Integer \to Z$ sowie einen eindeutigen Ringhomomorphismus $\psi \colon Z \to \Integer$.
      Es ist auch $\psi \circ \phi \colon \Integer \to \Integer$ ein Ringhomomorphismus.
      Die Identität $\id_\Integer \colon \Integer \to \Integer$ ebenfalls ein Ringhomomorphismus ist.
      Da es genau einen Ringhomomorphismus $\Integer \to \Integer$ gibt, muss sowohl $\psi \circ \phi$ als auch $\id_\Integer$  dieser eindeutige Ringhomomorphismus $\Integer \to \Integer$ sein.
      Folglich ist $\psi \circ \phi = \id_\Integer$.
      Analog ergibt sich, dass $\phi \circ \psi = \id_Z$.
  \end{enumerate}
\end{solution}


\begin{question}
  Es sei $R$ ein Ring.
  Konstruieren Sie eine Bijektion zwischen der Menge der Ringhomomorphismen $\Integer[T] \to R$ und $R$.
\end{question}


\begin{solution}
  Aus der Vorlesung ist bekannt, dass die Abbildung
  \begin{align*}
              \{ \text{Ringhomomorphismen $\Integer[T] \to R$} \}
    &\to      \{ \text{Ringhomomorphismen $\Integer \to R$} \} \times R,
    \\
              \phi
    &\mapsto  (\phi|_\Integer, \phi(T))
  \end{align*}
  eine Bijektion ist.
  Da es genau einen Ringhomomorphismus $\Integer \to R$ gibt, ergibt sich ferner, dass die Abbildung
  \[
            \{ \text{Ringhomomorphismen $\Integer \to R$} \} \times R
    \to     R,
    \quad
            (\psi, r)
    \mapsto r
  \]
  eine Bijektion ist.
  Damit ergibt sich insgesamt eine Bijektion
  \[
            \{ \text{Ringhomomorphismen $\Integer[T] \to R$} \}
    \to     R,
    \quad
            \phi
    \mapsto \phi(T).
  \]
\end{solution}



\begin{question}
  Es sei $R$ ein kommutativer Ring.
  \begin{enumerate}
    \item
      Zeigen Sie, dass ein Ideal $\pideal \subseteq R$ genau dann prim ist, wenn $R/\pideal$ ein Integritätsbereich ist.
    \item
      Zeigen Sie, dass ein Ideal $\mideal \subseteq R$ genau dann maximal ist, wenn $R/\mideal$ ein Körper ist. 
  \end{enumerate}
\end{question}


\begin{solution}
  Dies ist eine Standardaussage, deren Beweis sich in jedem Algebra-Buch findet.
\end{solution}


\begin{question}
  Es sei $R$ ein kommutativer Ring und $I \subseteq R$ ein Ideal.
  \begin{enumerate}
    \item
      Definieren Sie das Radikal $\radicalideal{I}$ und zeigen Sie, dass $\radicalideal{I}$ ebenfalls ein Ideal ist.
    \item
      Zeigen Sie, dass $\radicalideal{\radicalideal{I}} = \radicalideal{I}$.
    \item
      Zeigen Sie, dass $\radicalideal{I}$ genau dann ein echtes Ideal ist, wenn $I$ ein echtes Ideal ist.
  \end{enumerate}
  Ein Ring $S$ heißt \emph{reduziert}, falls $0$ das einzige nilpotente Element von $S$ ist.
  \begin{enumerate}[resume]
    \item
      Zeigen Sie, dass $R/I$ genau dann reduziert ist, wenn $I$ ein Radikalideal ist.
    \item
      Zeigen Sie, dass jedes Primideal ein Radikalideal ist.
  \end{enumerate}
\end{question}


% TODO: Adding a solution.


\begin{question}
  Es sei $R$ ein kommutativer Ring und $\pideal \subseteq R$ ein Ideal.
  Zeigen Sie, dass $\pideal$ genau dann ein Primideal ist, wenn es einen Körper $K$ und einen Ringhomomorphismus $\phi \colon R \to K$ mit $\ker \phi = \pideal$ gibt.
\end{question}


\begin{solution}
  Ist $\pideal$ ein Primideal, so ist der Quotient $R/\pideal$ ein Integritätsbereich.
  Da die kanonische Inklusion $R/\pideal \to Q(R/\pideal)$ ein injektiver Ringhomomorphismus ist, folgt für die Komposition
  \[
    \phi \colon R \xrightarrow{\pi} R/\pideal \to Q(R/\pideal),
  \]
  dass $\ker \phi = \ker \pi = \pideal$.
  (Hier bezeichnet $\pi \colon R \to R/\pideal$ die kanonische Projektion.)
  Da $Q(R/\pideal)$ ein Körper ist, zeigt dies eine Implikation.
  
  Gibt es andererseits einen Körper $K$ und einen Ringhomomorphismus $\phi \colon R \to K$ mit $\pideal = \ker \phi$, so ist $R/\pideal \cong \im \phi \subseteq K$.
  Der Körper $K$ ist insbesondere ein Integritätsbereich, weshalb auch der Unterring $\im \phi$ ein Integritätsbereich ist.
  Der Quotient $R/\pideal$ ist also ein Integritätsbereich und $\pideal$ somit eine Primideal.
\end{solution}


\begin{question}
  Es sei $K$ ein Körper.
  \begin{enumerate}
    \item
      Zeigen Sie, dass es für jedes Polynom $f \in K[X]$ einen eindeutigen $K$-linearen Ringhomomorphismus $\phi_f \colon K[X] \to K[X]$ gibt, so dass $\phi_f(X) = f$.
      
      (\emph{Hinweis}:
       Zeigen Sie zunächst, dass $\phi_f|_K = \id_K$ gilt.)
    \item
      Zeigen Sie, dass $\phi_f$ genau dann ein Ringisomorphismus ist, wenn $\deg f = 1$.
  \end{enumerate}
\end{question}


% TODO: Adding a solution.


\begin{question}[subtitle = Funktorialität der Einheitengruppe]
  Ist $R$ ein Ring, so ist
  \[
              R^\times
    \coloneqq \{x \in R \mid \text{$x$ ist eine Einheit}\}
  \]
  die \emph{Einheitengruppe} von $R$.
  Zeigen Sie:
  \begin{enumerate}
    \item
      Ist $R$ ein Ring, so bildet $R^\times$ bezüglich der Multiplikation aus $R$ eine Gruppe.
    \item
      Sind $R$ und $S$ zwei Ringe und ist $\phi \colon R \to S$ ein Ringhomomorphismus, so induziert $\phi$ per Einschränkung einen Gruppenhomomorphismus
      \[
        \phi^\times \colon R^\times \to S^\times,
        \quad
        x \mapsto \phi(x).
      \]
    \item
      Für jeden Ring $R$ gilt $\id_R^\times = \id_{R^\times}$, und für alle Ringhomomorphismen $\phi \colon R_1 \to R_2$ und $\psi \colon R_2 \to R_3$ gilt $(\psi \phi)^\times = \psi^\times \phi^\times$.
  \end{enumerate}
  (\emph{Hinweis}:
   Zum Verständnis genügt es kommutative Ringe zu betrachten.
   Die Aussage ist aber auch für nicht-kommutative Ringe von Bedeutung.)
\end{question}


% TODO: Lösung hinzufügen


\begin{question}[subtitle = Urbilder von Idealen]
  Es seien $R$ und $S$ zwei kommutative Ringe und $\phi \colon R \to S$ ein Ringhomomorphismus.
  \begin{enumerate}
    \item
      Zeigen Sie, dass für jedes Ideal $\aideal \subseteq S$ das Urbild $\phi^{-1}(\aideal)$ ein Ideal in $R$ ist.
    \item
      Entscheiden Sie, ob $\phi^{-1}(\pideal)$ ein Primideal ist, wenn $\pideal \subseteq S$ ein Primideal ist.
    \item
      Entscheiden Sie, ob $\phi^{-1}(\mideal)$ ein maximales Ideal ist, wenn $\mideal \subseteq S$ ein maximales Ideal ist.
  \end{enumerate}
\end{question}


\begin{solution}
  \begin{enumerate}
    \item
      Es sei $\pi \colon S \to S/\aideal$, $s \mapsto \overline{s}$ die kanonische Projektion.
      Dann ist $\pi \phi$ ein Ringhomomorphismus und somit $\ker (\pi \phi) = \phi^{-1}(\ker \pi) = \phi^{-1}(\aideal)$ ein Ideal in $R$.
    \item
      Die Aussage gilt:
      Es sei $\pi \colon S \to S/\pideal$, $s \mapsto \overline{s}$ die kanonische Projektion und $\qideal \coloneqq \phi^{-1}(\pideal)$.
      Der Quotient $S/\pideal$ ist ein Integritätsbereich, da $\pideal$ ein Primideal ist.
      Nach dem vorherigen Aufgabenteil ist $\qideal$ ein Ideal in $R$, und da $\ker (\pi \phi) = \phi^{-1}(\ker \pi) = \phi^{-1}(\pideal) = \qideal$ induziert $\pi \phi$ einen injektiven Ringhomomorphismus
      \[
        \psi \colon R/\qideal \to S/\pideal
        \quad
        \overline{r} \mapsto \overline{\phi(r)}.
      \]
      Der Ring $\im (\pi \phi) \subseteq S/\pideal$ ist als Unterring eines Integritätsbereichs ebenfalls ein Integritätsbereich.
      Somit ist $R/\qideal \cong \im(\pi \phi)$ ein Integritätsbereich, also $\qideal$ ein Primideal.
    \item
      Die Aussage gilt nicht:
      Es sei etwa $\phi \colon \Integer \to \Rational$ die kanonische Inklusion.
      Dann ist $\mideal \coloneqq 0$ ein maximales Ideal in $\Rational$, aber $\phi^{-1}(0) = 0$ ist kein maximales Ideal in $\Integer$, da $\Integer/\mideal \cong \Integer$ kein Körper ist.
  \end{enumerate}
\end{solution}


\begin{question}
  Es sei $R$ ein kommutativer Ring.
  Es seien $\aideal, \bideal \subseteq R$ zwei Ideale mit $\aideal = (x_i \mid i \in I)$ und $\bideal = (y_j \mid j \in J)$.
  Zeigen Sie, dass
  \[
    \aideal \bideal = (x_i y_j \mid i \in I, j \in J).
  \]
\end{question}


\begin{solution}
  Für alle $i \in I$ und $j \in J$ folgt aus $x_i \in \aideal$ und $y_j \in \bideal$, dass $x_i y_j \in \aideal \bideal$.
  Daraus folgt, dass $(x_i y_j \mid i \in I, j \in J) \subseteq \aideal \bideal$.
  Sind andererseits $a \in \aideal$ und $b \in \bideal$, so ist $a = \sum_{i \in I} r_i x_i$ und $b = \sum_{j \in J} s_j y_j$ mit $r_i, s_j \in R$, wobei $r_i = 0$ für fast alle $i \in I$ und $s_j = 0$ für fast alle $j \in J$.
  Deshalb ist
  \[
        ab
    =   \sum_{\substack{i \in I \\ j \in J}} r_i s_j x_i y_j
    \in (x_i y_j \mid i \in I, j \in J).
  \]
  Da jedes Element aus $\aideal \bideal$ von der Form $\sum_{k=1}^n a_k b_k$ mit $a_k \in \aideal$ und $b_k \in \bideal$ ist, folgt daraus, dass $\aideal \bideal \subseteq (x_i y_j \mid i \in I, j \in J)$.
\end{solution}


\begin{question}[subtitle = Zur Definition von Unterringen]
  Geben Sie ein Beispiel für einen kommutativen Ring $R$ und eine Teilmenge $S \subseteq R$ mit den folgenden Eigenschaften:
  \begin{itemize}
    \item
      $S$ ist abgeschlossen unter der Addition und Multiplikation von $R$, d.h.\ für alle $s_1, s_2 \in S$ ist auch $s_1 + s_2 \in S$ und $s_1 s_2 \in S$.
    \item
      Zusammen mit der Einschränkung der Addition und Multiplikation aus $R$ ist $S$ ebenfalls ein (notwendigerweise kommutativer) Ring.
    \item
      $S$ ist kein Unterring von $R$.
  \end{itemize}
\end{question}


\begin{solution}
  Es sei $R = \Integer \times \Integer$ und $S = \Integer \times 0 = \{(n,0) \mid n \in \Integer\}$.
  Offenbar ist $S$ unter der Addition und Multiplikation abgeschlossen.
  Zusammen mit der Einschränkung dieser Operationen bildet $S$ einen kommutativen Ring, für den $S \cong \Integer$ gilt.
  Da $1_R = (1,1) \notin S$ ist $S$ allerdings kein Unterring von $R$.
\end{solution}


\begin{question}
  Es sei $R$ ein kommutativer Ring.
  \begin{enumerate}
    \item
      Definieren Sie, wann zwei Elemente von $R$ assoziiert sind.
    \item
      Es sei nun $R$ ein Integritätsbereich.
      Zeigen Sie, dass zwei Elemente $a, b \in R$ genau dann assoziiert sind, wenn $(a) = (b)$.
  \end{enumerate}
\end{question}


% TODO: Adding a solution.


\begin{question}
  Es sei $R$ ein kommutativer Ring und $S \subseteq R$ eine multiplikative Teilmenge.
  \begin{enumerate}
    \item
      Zeigen Sie, dass $R_S$ noethersch ist, wenn $R$ noethersch ist.
    \item
      Zeigen oder widerlegen Sie, dass $R_S$ ein Hauptidealring ist, wenn $R$ ein Hauptidealring ist.
  \end{enumerate}
\end{question}


% TODO: Adding a solution.


\begin{question}
  Es sei $R$ ein Ring und $I \subseteq R$ ein Ideal.
  \begin{enumerate}
    \item
      Zeigen Sie, dass $R/I$ noethersch ist, wenn $R$ noethersch ist.
    \item
      Zeigen Sie widerlegen, dass $R/I$ ein Hauptidealring ist, wenn $R$ ein Hauptidealring ist.
  \end{enumerate}
\end{question}


% TODO: Adding a solution.


\begin{question}
  Für jedes $d \in \Natural$ sei
  \[
              \Integer[\sqrt{-d}]
    \coloneqq \Integer[i\sqrt{d}]
    =         \{a + i \sqrt{d} b \mid a, b \in \Integer \}
    \subseteq \Complex.
  \]
  Es darf im Folgenden ohne Beweis genutzt werden, dass $\Integer[\sqrt{-d}]$ ein Unterring von $\Complex$ ist.
  \begin{enumerate}
    \item
      Zeigen Sie, dass $\Integer[\sqrt{-1}]$ ein euklidischer Ring ist.
    \item
      Zeigen Sie, dass $\Integer[\sqrt{-2}]$ ein euklidischer Ring ist.
    \item
      Zeigen Sie, dass $\Integer[\sqrt{-5}]$ kein euklidischer Ring ist.
  \end{enumerate}
\end{question}


% TODO: Adding a solution.


\begin{question}
  Es sei $R$ ein euklidischer Ring.
  Zeigen Sie, dass $R$ ein Hauptidealring ist.
\end{question}


% TODO: Adding a solution


\begin{question}
  Es sei $R$ ein kommutativer Ring, so dass $R[X]$ ein Hauptidealring ist.
  Zeigen Sie, dass $R$ bereits ein Körper ist.
\end{question}


% TODO: Adding a solution.


\begin{question}
  Es sei $K$ ein Körper.
  Zeigen Sie, dass es in $K[X]$ unendlich viele irreduzible, normierte Polynome gibt.
\end{question}


% TODO: Adding a solution.


\begin{question}
  Es seien $R$ und $R'$ zwei kommutative Ringe, $S \subseteq R$ eine multiplikative Teilmenge und $f \colon R \to R'$ ein Ringhomomorphismus.
  \begin{enumerate}
    \item
      Zeigen Sie, dass $S' \coloneqq f(S)$ eine multiplikative Teilmenge von $R'$ ist.
    \item
      Zeigen Sie, dass $f$ einen Ringhomomorphismus $f_S \colon R_S \to R'_{S'}$ induziert.
  \end{enumerate}
\end{question}



% TODO: Adding a solution.


\begin{question}
  Zeigen Sie, dass $\Integer[i] \cong \Integer[X]/(X^2 + 1)$.
\end{question}


% TODO: Adding a solution.


\begin{question}
  Es sei $R$ ein kommutativer Ring und $f \in R$.
  Zeigen Sie, dass $R_f \cong R[X]/(fX-1)$.
\end{question}


% TODO: Adding a solution.


\begin{question}
  Bestimmen Sie die Einheitengruppe $\Integer[i]^\times$.
\end{question}


\begin{solution}
  Ein Element $z \in \Integer[i]$ ist genau dann eine Einheit in $\Integer[i]$, wenn $z \neq 0$ und $z^{-1} \in \Integer[i]$ (hier bezeichnet $z^{-1} = 1/z$ das Inverse von $z$ in $\Complex$).
  Für die Elemente $1, -1, i, -i \in \Integer[i]$ ist dies erfüllt.
  Ist $z \in \Integer[i]$ mit $z \neq 0$ und $z^{-1} \in \Integer[i]$, so ist
  \begin{equation}
    \label{eqn: units in the gaussian integers}
      1
    = |1|^2
    = |z z^{-1}|^2
    = |z|^2 |z^{-1}|.
  \end{equation}
  Für alle $w \in \Integer[i]$ mit $w = a + ib$ gilt $a, b \in \Integer$ und deshalb $|w|^2 = a^2 + b^2 \in \Integer$.
  In \eqref{eqn: units in the gaussian integers} gilt deshalb, dass $|z|^2, |z^{-1}|^2 \in \Integer$, und somit $|z|^2 \in \Integer^\times = \{1,-1\}$.
  Also gilt $|z|^2 = 1$.
  Ist $z = a + ib$ mit $a,b \in \Integer$ so ist also $a^2 + b^2 = 1$ und somit entweder $a = 0$ und $b = \pm 1$, oder $a = \pm 1$ und $b = 0$.
  Es ist also $z \in \{1, -1, i, -i\}$.
  Insgesamt zeigt dies, dass $\Integer[i]^\times = \{1, -1, i, -i\}$.
\end{solution}


