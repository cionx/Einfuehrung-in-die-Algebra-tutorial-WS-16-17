\section{Ringtheorie}


\begin{question}[subtitle = Initialobjekte in der Kategorie der Ringe]
  \label{qst: Z is inital}
  \begin{enumerate}
    \item
      Überzeugen Sie sich davon, dass es für jeden Ring $R$ genau einen Ringhomomorphismus $\Integer \to R$ gibt.
      (Dies bedeutet, dass $\Integer$ ein Initialobjekt in der Kategorie der Ringe ist.)
    \item
      Es sei $Z$ ein Ring, so dass es für jeden Ring $R$ einen eindeutigen Ringhomomorphismus $Z \to R$ gibt.
      Zeigen Sie, dass $Z \cong \Integer$.
  \end{enumerate}
\end{question}


\begin{solution}
  \begin{enumerate}
    \item
      Ist $\phi \colon \Integer \to R$ ein Ringhomomorphismus, so ist $\phi(1_\Integer) = 1_R$.
      Für alle $n \in \Integer$ ist damit
      \[
          \phi(n)
        = \phi(n \cdot 1_\Integer)
        = n \cdot \phi(1_\Integer)
        = n \cdot 1_R.
      \]
      Also ist $\phi$ eindeutig.
      Durch direktes Nachrechnen ergibt sich auch, dass $\psi \colon \Integer \to R$ mit
      \[
        \psi(n) \coloneqq n \cdot 1_R
        \quad
        \text{für alle $n \in \Integer$}
      \]
      ein Ringhomomorphismus ist.
    \item
      Es gibt einen eindeutigen Ringhomomorphismus $\phi \colon \Integer \to Z$ sowie einen eindeutigen Ringhomomorphismus $\psi \colon Z \to \Integer$.
      Es ist auch $\psi \circ \phi \colon \Integer \to \Integer$ ein Ringhomomorphismus.
      Die Identität $\id_\Integer \colon \Integer \to \Integer$ ist ebenfalls ein Ringhomomorphismus2.
      Da es genau einen Ringhomomorphismus $\Integer \to \Integer$ gibt, muss sowohl $\psi \circ \phi$ als auch $\id_\Integer$  dieser eindeutige Ringhomomorphismus $\Integer \to \Integer$ sein.
      Folglich gilt $\psi \circ \phi = \id_\Integer$.
      Analog ergibt sich, dass auch $\phi \circ \psi = \id_Z$ gilt.
  \end{enumerate}
\end{solution}


\begin{question}
  Es sei $R$ ein Ring.
  Konstruieren Sie eine Bijektion zwischen der Menge der Ringhomomorphismen $\Integer[T] \to R$ und $R$.
\end{question}


\begin{solution}
  Aus der Vorlesung ist bekannt, dass die Abbildung
  \begin{align*}
              \{ \text{Ringhomomorphismen $\Integer[T] \to R$} \}
    &\to      \{ \text{Ringhomomorphismen $\Integer \to R$} \} \times R,
    \\
              \phi
    &\mapsto  (\phi|_\Integer, \phi(T))
  \end{align*}
  eine Bijektion ist.
  Da es genau einen Ringhomomorphismus $\Integer \to R$ gibt, ergibt sich ferner, dass die Abbildung
  \[
            \{ \text{Ringhomomorphismen $\Integer \to R$} \} \times R
    \to     R,
    \quad
            (\psi, r)
    \mapsto r
  \]
  eine Bijektion ist.
  Damit ergibt sich insgesamt eine Bijektion
  \[
            \{ \text{Ringhomomorphismen $\Integer[T] \to R$} \}
    \to     R,
    \quad
            \phi
    \mapsto \phi(T).
  \]
\end{solution}



\begin{question}
  Es sei $R$ ein kommutativer Ring.
  \begin{enumerate}
    \item
      Zeigen Sie, dass ein Ideal $\pideal \subseteq R$ genau dann prim ist, wenn $R/\pideal$ ein Integritätsbereich ist.
    \item
      Zeigen Sie, dass ein Ideal $\mideal \subseteq R$ genau dann maximal ist, wenn $R/\mideal$ ein Körper ist. 
  \end{enumerate}
\end{question}


\begin{solution}
  \begin{enumerate}
    \item
      Für alle $x \in R$ sei $\overline{x} \in R/\pideal$ die entsprechende Äquivalenzklasse.
      Das Ideal $\pideal$ ist genau dann prim, wenn die Aussage
      \begin{equation}
        \label{eqn: quotient is integral definition}
        \forall x, y \in R
        :
        \overline{x} \cdot \overline{y} = 0
        \implies
        \text{$\overline{x} = 0$ oder $\overline{y} = 0$}
      \end{equation}
      gilt. Da $\overline{x} \cdot \overline{y} = \overline{xy}$ für alle $x, y \in R$ gilt, ist die Aussage~\eqref{eqn: quotient is integral definition} äquivalent dazu, dass
      \begin{equation}
        \label{eqn: quotient is integral nearly definition}
        \forall x, y \in R
        :
        \overline{xy} = 0
        \implies
        \text{$\overline{x} = 0$ oder $\overline{y} = 0$}.
      \end{equation}
      Für alle $x \in R$ gilt genau dann $\overline{x} = 0$, wenn $x \in \pideal$.
      Deshalb ist die Aussage~\eqref{eqn: quotient is integral nearly definition} äquivalent dazu, dass
      \begin{equation}
        \label{eqn: quotient is integral via prime }
        \forall x, y \in R
        :
        xy \in \pideal
        \implies
        \text{$x \in \pideal$ oder $y \in \pideal$}.
      \end{equation}
      Dies ist genau die Aussage, dass $\pideal$ ein Primideal ist.
    \item
      Es sei $\pi \colon R \to R/\mideal$, $x \mapsto \overline{x}$ die kanonische Projektion.
      Wie aus der Vorlesung bekannt ist die Abbildung
      \[
            \{ \text{Ideale $I \subseteq R/\mideal$} \}
        \to \{ \text{Ideale $J \subseteq R$ mit $J \supseteq \mideal$} \},
        \quad
        I \mapsto \pi^{-1}(I)
      \]
      eine wohldefinierte Bijektion.
      Der Ring $R/\mideal$ ist genau dann ein Körper, wenn $R/\mideal$ genau zwei Ideale enthält (man siehe Übung~\ref{question: characterization of fields via its ideals}); das Ideal $\mideal$ ist genau dann ein maximales Ideal in $R$, wenn es genau zwei Ideale $J \subseteq R$ mit $J \supseteq \mideal$ gibt.
      Wegen der Existenz der obigen Bijektion sind beide Aussagen äquivalent.
  \end{enumerate}
\end{solution}


\begin{question}
  Es sei $R$ ein kommutativer Ring und $I \subseteq R$ ein Ideal.
  \begin{enumerate}
    \item
      Definieren Sie das Radikal $\radicalideal{I}$ und zeigen Sie, dass $\radicalideal{I}$ ein Ideal mit $I \subseteq \sqrt{I}$ ist.
    \item
      Zeigen Sie, dass $\radicalideal{\radicalideal{I}} = \radicalideal{I}$.
    \item
      Zeigen Sie, dass $\radicalideal{I}$ genau dann ein echtes Ideal ist, wenn $I$ ein echtes Ideal ist.
  \end{enumerate}
  Ein Ideal $I$ ist ein \emph{Radikalideal}, wenn $I = \radicalideal{J}$ für ein Ideal $J \subseteq I$.
  \begin{enumerate}[resume]
    \item
      Zeigen Sie, dass $I$ genau dann ein Radikalideal ist, wenn $\radicalideal{I} = I$.
  \end{enumerate}
  Ein Ring $S$ heißt \emph{reduziert}, falls $0$ das einzige nilpotente Element von $S$ ist.
  \begin{enumerate}[resume]
    \item
      Zeigen Sie, dass $R/I$ genau dann reduziert ist, wenn $I$ ein Radikalideal ist.
    \item
      Zeigen Sie, dass jedes Primideal ein Radikalideal ist.
  \end{enumerate}
\end{question}


\begin{solution}
  \begin{enumerate}
    \item
      Das Radikal $\radicalideal{I}$ ist als
      \[
          \radicalideal{I}
        = \{ r \in R \mid \text{es gibt $n \in \Natural$ mit $r^n \in I$} \}
      \]
      definiert.
      Für alle $x \in I$ gilt $x^1 = x \in I$, weshalb $I \subseteq \radicalideal{I}$.
      
      Insbesondere ist somit $0 \in \radicalideal{I}$, da $0 \in I$.
      Für $x, y \in \radicalideal{I}$ gibt es $n, m \in \Natural$ mit $x^n, y^m \in I$.
      Für alle $k = 0, \dotsc, n+m$ gilt deshalb $x^k \in I$ oder $y^{n+m-k} \in I$, und somit auch
      \[
          (x + y)^{n+m}
        = \sum_{k=0}^{n+m} \binom{n+m}{k} x^k y^{n+m-k} \in I.
      \]
      Deshalb ist auch $x + y \in \radicalideal{I}$.
      Für $r \in R$ und $x \in I$ gibt es $n \in \Natural$ mit $x^n \in I$, weshalb auch
      \[
        (r x)^n = r^n x^n \in I.
      \]
      Somit ist auch $r x \in \radicalideal{I}$.
      
    \item
      Wir wissen bereits, dass $\radicalideal{I} \subseteq \radicalideal{\radicalideal{I}}$.
      Für $x \in \radicalideal{\radicalideal{I}}$ gibt es $n \in \Natural$ mit $x^n \in \radicalideal{I}$, und somit auch noch $m \in \Natural$ mit $(x^n)^m \in I$.
      Damit ist $x^{nm} \in I$, weshalb auch $\radicalideal{\radicalideal{I}} \subseteq \radicalideal{I}$.
    
    \item
      $I$ ist genau dann ein echtes Ideal, wenn $1 \notin I$.
      Da $1^n = 1$ für alle $n \in \Natural$ ist genau dann $1 \notin I$, wenn $1 \notin \radicalideal{I}$.
      Dies ist wiederum äquivalent dazu, dass $\radicalideal{I}$ ein echtes Ideal ist.
    
    \item
      Gilt $I = \radicalideal{I}$ so erfüllt $I$ die definierende Eigenschaft eines Radikalideals (mit $J = I$).
      Ist andererseits $I = \radicalideal{J}$ für ein Ideal $J \subseteq R$, so gilt
      \[
          \radicalideal{I}
        = \radicalideal{\radicalideal{J}}
        = \radicalideal{J}
        = I.
      \]

    \item
      Der Quotient $R/I$ ist genau reduziert, wenn
      \begin{equation}
        \label{eqn: reduced first equation}
        \text{es gibt $n \in \Natural$ mit $\overline{x}^n = 0$}
        \implies
        \overline{x} = 0
        \qquad
        \text{für alle $x \in R$}.
      \end{equation}
      Dabei gilt $\overline{x}^n = \overline{x^n}$ für alle $x \in R$ und $n \in \Natural$, und für alle $y \in R$ gilt genau dann $\overline{y} = 0$, wenn $y \in I$.
      Daher ist \eqref{eqn: reduced first equation} äquivalent dazu, dass
      \begin{equation}
        \label{eqn: reduced second equation}
        \text{es gibt $n \in \Natural$ mit $x^n \in I$}
        \implies
        x \in I
        \qquad
        \text{für alle $x \in R$}.
      \end{equation}
      Durch Einsetzen der Definition von $\radicalideal{I}$ ergibt sich aus \eqref{eqn: reduced second equation} die äquivalente Bedingung
      \[
        x \in \radicalideal{I}
        \implies
        x \in I
        \qquad
        \text{für alle $x \in R$}.
      \]
      Dies bedeutet gerade, dass $\radicalideal{I} \subseteq I$.
      Da $I \subseteq \radicalideal{I}$ ist dies äquivalent dazu, dass $I = \sqrt{I}$, dass also $I$ ein Radikalideal ist.
      
    \item
      Der Quotient $R/\pideal$ ist ein Integritatsbereich, da $\pideal$ ein Primideal ist.
      Nach dem vorherigen Aufgabenteil genügt es zu zeigen, dass jeder Integritätsbereich $S$ reduziert ist.
      Dies folgt direkt daraus, dass für jedes $x \in S$ mit $x^n = 0$ aus der Nullteilerfreiheit von $S$ folgt, dass $x = 0$.
  \end{enumerate}
\end{solution}


\begin{question}
  Es sei $R$ ein kommutativer Ring und $\pideal \subseteq R$ ein Ideal.
  Zeigen Sie, dass $\pideal$ genau dann ein Primideal ist, wenn es einen Körper $K$ und einen Ringhomomorphismus $\phi \colon R \to K$ mit $\ker \phi = \pideal$ gibt.
\end{question}


\begin{solution}
  Ist $\pideal$ ein Primideal, so ist der Quotient $R/\pideal$ ein Integritätsbereich.
  Da die kanonische Inklusion $R/\pideal \to Q(R/\pideal)$ ein injektiver Ringhomomorphismus ist, folgt für die Komposition
  \[
    \phi \colon R \xrightarrow{\pi} R/\pideal \to Q(R/\pideal),
  \]
  dass $\ker \phi = \ker \pi = \pideal$.
  (Hier bezeichnet $\pi \colon R \to R/\pideal$ die kanonische Projektion.)
  Da $Q(R/\pideal)$ ein Körper ist, zeigt dies eine Implikation.
  
  Gibt es andererseits einen Körper $K$ und einen Ringhomomorphismus $\phi \colon R \to K$ mit $\pideal = \ker \phi$, so ist $R/\pideal \cong \im \phi \subseteq K$.
  Der Körper $K$ ist insbesondere ein Integritätsbereich, weshalb auch der Unterring $\im \phi$ ein Integritätsbereich ist.
  Der Quotient $R/\pideal$ ist also ein Integritätsbereich und $\pideal$ somit eine Primideal.
\end{solution}


% \begin{question}
%   Es sei $K$ ein Körper.
%   \begin{enumerate}
%     \item
%       Zeigen Sie, dass es für jedes Polynom $f \in K[X]$ einen eindeutigen $K$-linearen Ringhomomorphismus $\phi_f \colon K[X] \to K[X]$ gibt, so dass $\phi_f(X) = f$.
%       
%       (\emph{Hinweis}:
%        Überlegen Sie sich zunächst, dass ein Ringhomomorphismus $\phi \colon K[X] \to K[X]$ genau dann $K$-linear ist, wenn $\phi|_K = \id_K$.)
%     \item
%       Zeigen Sie, dass $\phi_f(p) = p(f)$ für alle $p \in K[X]$.
%     \item
%       Zeigen Sie, dass $\phi_f$ genau dann ein Ringisomorphismus ist, wenn $\deg f = 1$.
%   \end{enumerate}
% \end{question}
% 
% 
% \begin{solution}
%   \begin{enumerate}
%     \item
%       Ist $\phi \colon K[X] \to K[X]$ ein $K$-linearer Ringhomomorphismus, so muss
%       \[
%           \phi(\lambda)
%         = \phi(\lambda \cdot 1)
%         = \lambda \phi(1)
%         = \lambda \cdot 1
%         = \lambda
%         \qquad
%         \text{für alle $\lambda \in K$},
%       \]
%       weshalb $\phi|_K = \id_K$.
%       Ist andererseits $\phi \colon K[X] \to K[X]$ ein Ringhomomorphismus mit $\phi|_K = \id_K$, so gilt
%       \[
%           \phi(\lambda \cdot f)
%         = \phi(\lambda) \cdot \phi(f)
%         = \lambda \cdot \phi(f)
%         \qquad
%         \text{für alle $\lambda \in K$, $f \in K[X]$}.
%       \]
%       Also ist ein Ringhomomorphismus $\phi \colon K[X] \to K[X]$ genau dann $K$-linear, wenn $\phi|_K = \id_K$.
%       
%       Es gilt also zu zeigen, dass es für jedes $f \in K[X]$ genau einen Ringhomomorphismus $\phi_f \colon K[X] \to K[X]$ mit $\phi_f|_K = \id_K$ und $\phi_f(T) = f$ gibt.
%       Dies gilt nach der universellen Eigenschaft des Polynomrings.
%     \item
%       
%   \end{enumerate}
% \end{solution}


\begin{question}[subtitle = Funktorialität der Einheitengruppe]
  Ist $R$ ein kommutativer Ring, so ist
  \[
              R^\times
    \coloneqq \{x \in R \mid \text{$x$ ist eine Einheit}\}
  \]
  die \emph{Einheitengruppe} von $R$.
  Zeigen Sie:
  \begin{enumerate}
    \item
      Ist $R$ ein kommutativer Ring, so bildet $R^\times$ mit der Multiplikation aus $R$ eine abelsche Gruppe.
    \item
      Sind $R$ und $S$ zwei kommutativer Ringe und ist $\phi \colon R \to S$ ein Ringhomomorphismus, so induziert $\phi$ per Einschränkung einen Gruppenhomomorphismus
      \[
        \phi^\times \colon R^\times \to S^\times,
        \quad
        x \mapsto \phi(x).
      \]
    \item
      Für jeden Ring kommutativen $R$ gilt $\id_R^\times = \id_{R^\times}$, und für alle kommutativen Ringe $R_1$, $R_2$ und $R_3$ und Ringhomomorphismen $\phi \colon R_1 \to R_2$ und $\psi \colon R_2 \to R_3$ gilt $(\psi \phi)^\times = \psi^\times \phi^\times$.
    \item
      Ist $R$ ein kommutativer Ring und $\phi \colon R \to S$ ein Isomorphismus von Ringen, so ist $\phi^\times \colon R^\times \to S^\times$ ein Isomorphismus von Gruppen.
  \end{enumerate}
  (Die Aussagen gelten auch für nichtkommutative Ringe, wobei $R^\times$ dann im Allgemeinen nicht abelsch ist.
   Dabei ist ein Element $r \in R$ eines nichtkommutativen Rings $R$ eine Einheit, wenn es $s \in R$ mit $rs = 1 = sr$ gibt.
   Es genügt auch, dass es $s, t \in R$ mit $rs = 1 = tr$ gibt; dann gilt bereits $s = t$.)
\end{question}


\begin{solution}
  \begin{enumerate}
    \item
      Die Multiplikation in $R^\times$ ist assoziativ, da sie es in $R$ ist.
      Dass $R^\times$ abelsch ist ergibt sich aus der Kommutativität von $R$.
      Es gilt $1 \in R^\times$, und da $1$ in ganz $R$ neutral bezüglich der Multiplikation ist, gilt dies auch in $R^\times$.
      Für jedes $x \in R^\times$ gibt es ein $y \in R$ mit $xy = 1$.
      Dann gilt auch $y \in R^\times$ und $y$ ist auch in $R^\times$ invers zu $x$.
    \item
      Für $x \in R^\times$ gilt
      \[
          1
        = \phi(1)
        = \phi(x x^{-1})
        = \phi(x) \phi(x^{-1}).
      \]
      Deshalb ist $\phi(x)$ eine Einheit in $S$ (mit $\phi(x)^{-1} = \phi(x^{-1})$), und somit $\phi(x) \in S^\times$.
      Das zeigt, dass die Einschränkung $\phi^\times$ wohldefiniert ist.
      Da $\phi$ mulitplikativ ist, gilt dies auch für $\phi^\times$, weshalb $\phi^\times$ ein Gruppenhomomorphismus ist.
    \item
      Da $\id_R^\times(x) = \id_R(x) = x = \id_{R^\times}(x)$ für alle $x \in X$ gilt, ist $\id_R^\times = \id_{R^\times}$.
      Für alle $x \in R_1$ gilt
      \[
          (\psi^\times \phi^\times)(x)
        = \psi^\times(\phi^\times(x))
        = \psi(\phi(x))
        = (\psi \phi)(x)
        = (\psi \phi)^\times(x).
      \]
      Deshalb ist $(\psi^\times \phi^\times) = (\psi \phi)^\times$.
    \item
      Es sei $\psi \coloneqq \phi^{-1} \colon S \to R$.
      Es gilt
      \[
          \phi^\times \psi^\times
        = (\phi \psi)^\times
        = \left( \phi \phi^{-1} \right)^\times
        = \id_S^\times
        = \id_{S^\times}
      \]
      und analog auch $\psi^\times \phi^\times = \id_{R^\times}$.
      Also ist der Gruppenhomomorphismus $\phi^\times$ bijektiv mit $(\phi^\times)^{-1} = (\phi^{-1})^\times$, und somit ein Gruppenisomorphismus.
  \end{enumerate}
\end{solution}


\begin{question}
  Die Eulersche Phi-Funktion ist definiert als
  \[
            \varphi
    \colon  \Natural_{>1} \to \Natural,
    \quad   n \mapsto |\{ k \in \{1, \dotsc, n\} \mid \text{$k$ und $n$ sind teilerfremd} \}|.
  \]
  \begin{enumerate}
    \item
      Zeigen Sie, dass $\varphi(n) = |(\Integer/n)^\times|$ für alle $n \geq 1$.
    \item
      Folgern Sie, dass $\varphi(n_1 n_2) = \varphi(n_1) \varphi(n_2)$ für je zwei teilerfremde $n_1, n_2 \geq 1$.
    \item
      Zeigen Sie, dass $\varphi(p^r) = p^r - p^{r-1} = p^{r-1} (p - 1)$ für alle Primzahlen $p \in \Natural$ und $r \geq 1$.
    \item
      Berechnen Sie $\varphi(42)$, $\varphi(57)$ und $\varphi(144)$.
  \end{enumerate}
\end{question}


% TODO: Adding a solution.


\begin{question}[subtitle = Urbilder von Idealen]
  Es seien $R$ und $S$ zwei kommutative Ringe und $\phi \colon R \to S$ ein Ringhomomorphismus.
  \begin{enumerate}
    \item
      Zeigen Sie, dass für jedes Ideal $\aideal \subseteq S$ das Urbild $\phi^{-1}(\aideal)$ ein Ideal in $R$ ist.
    \item
      Entscheiden Sie, ob $\phi^{-1}(\pideal)$ ein Primideal ist, wenn $\pideal \subseteq S$ ein Primideal ist.
    \item
      Entscheiden Sie, ob $\phi^{-1}(\mideal)$ ein maximales Ideal ist, wenn $\mideal \subseteq S$ ein maximales Ideal ist.
  \end{enumerate}
\end{question}


\begin{solution}
  \begin{enumerate}
    \item
      Es sei $\pi \colon S \to S/\aideal$, $s \mapsto \overline{s}$ die kanonische Projektion.
      Dann ist $\pi \phi$ ein Ringhomomorphismus und somit $\ker (\pi \phi) = \phi^{-1}(\ker \pi) = \phi^{-1}(\aideal)$ ein Ideal in $R$.
    \item
      Die Aussage gilt:
      Es sei $\pi \colon S \to S/\pideal$, $s \mapsto \overline{s}$ die kanonische Projektion und $\qideal \coloneqq \phi^{-1}(\pideal)$.
      Der Quotient $S/\pideal$ ist ein Integritätsbereich, da $\pideal$ ein Primideal ist.
      Nach dem vorherigen Aufgabenteil ist $\qideal$ ein Ideal in $R$, und da $\ker (\pi \phi) = \phi^{-1}(\ker \pi) = \phi^{-1}(\pideal) = \qideal$ induziert $\pi \phi$ einen injektiven Ringhomomorphismus
      \[
        \psi \colon R/\qideal \to S/\pideal
        \quad
        \overline{r} \mapsto \overline{\phi(r)}.
      \]
      Der Ring $\im (\pi \phi) \subseteq S/\pideal$ ist als Unterring eines Integritätsbereichs ebenfalls ein Integritätsbereich.
      Somit ist $R/\qideal \cong \im(\pi \phi)$ ein Integritätsbereich, also $\qideal$ ein Primideal.
    \item
      Die Aussage gilt nicht:
      Es sei etwa $\phi \colon \Integer \to \Rational$ die kanonische Inklusion.
      Dann ist $\mideal \coloneqq 0$ ein maximales Ideal in $\Rational$, aber $\phi^{-1}(0) = 0$ ist kein maximales Ideal in $\Integer$, da $\Integer/\mideal \cong \Integer$ kein Körper ist.
  \end{enumerate}
\end{solution}


\begin{question}
  \label{question: examples for non principal and not finitely generated modules}
  Es sei $K$ ein Körper.
  \begin{enumerate}
    \item
      Zeigen Sie, dass $(X,Y) \subseteq K[X,Y]$ kein Hauptideal ist.
    \item
      Zeigen Sie, dass das Ideal $(X_i \mid i \in \Natural) \subseteq K[X_i \mid i \in \Natural]$ nicht endlich erzeugt ist.
      
      (\emph{Hinweis}:
       In jedem Polynom $f \in K[X_i \mid i \in \Natural]$ kommen nur endlich viele Variablen vor.)
  \end{enumerate}
\end{question}


% TODO: Adding solutions.


% \begin{question}
%   Es sei $R$ ein kommutativer Ring.
%   Es seien $\aideal, \bideal \subseteq R$ zwei Ideale mit $\aideal = (x_i \mid i \in I)$ und $\bideal = (y_j \mid j \in J)$.
%   Zeigen Sie, dass
%   \[
%     \aideal \bideal = (x_i y_j \mid i \in I, j \in J).
%   \]
% \end{question}
% 
% 
% \begin{solution}
%   Für alle $i \in I$ und $j \in J$ folgt aus $x_i \in \aideal$ und $y_j \in \bideal$, dass $x_i y_j \in \aideal \bideal$.
%   Daraus folgt, dass $(x_i y_j \mid i \in I, j \in J) \subseteq \aideal \bideal$.
%   Sind andererseits $a \in \aideal$ und $b \in \bideal$, so ist $a = \sum_{i \in I} r_i x_i$ und $b = \sum_{j \in J} s_j y_j$ mit $r_i, s_j \in R$, wobei $r_i = 0$ für fast alle $i \in I$ und $s_j = 0$ für fast alle $j \in J$.
%   Deshalb ist
%   \[
%         ab
%     =   \sum_{\substack{i \in I \\ j \in J}} r_i s_j x_i y_j
%     \in (x_i y_j \mid i \in I, j \in J).
%   \]
%   Da jedes Element aus $\aideal \bideal$ von der Form $\sum_{k=1}^n a_k b_k$ mit $a_k \in \aideal$ und $b_k \in \bideal$ ist, folgt daraus, dass $\aideal \bideal \subseteq (x_i y_j \mid i \in I, j \in J)$.
% \end{solution}


\begin{question}[subtitle = Zur Definition von Unterringen]
  Geben Sie ein Beispiel für einen kommutativen Ring $R$ und eine Teilmenge $S \subseteq R$ mit den folgenden Eigenschaften:
  \begin{itemize}
    \item
      $S$ ist abgeschlossen unter der Addition und Multiplikation von $R$, d.h.\ für alle $s_1, s_2 \in S$ ist auch $s_1 + s_2 \in S$ und $s_1 s_2 \in S$.
    \item
      Zusammen mit der Einschränkung der Addition und Multiplikation aus $R$ ist $S$ ebenfalls ein (notwendigerweise kommutativer) Ring.
    \item
      $S$ ist kein Unterring von $R$.
  \end{itemize}
\end{question}


\begin{solution}
  Es sei $R = \Integer \times \Integer$ und $S = \Integer \times 0 = \{(n,0) \mid n \in \Integer\}$.
  Offenbar ist $S$ unter der Addition und Multiplikation abgeschlossen.
  Zusammen mit der Einschränkung dieser Operationen bildet $S$ einen kommutativen Ring, für den $S \cong \Integer$ gilt.
  Da $1_R = (1,1) \notin S$ ist $S$ allerdings kein Unterring von $R$.
\end{solution}


\begin{question}
  Es sei $R$ ein kommutativer Ring.
  \begin{enumerate}
    \item
      Definieren Sie, wann zwei Elemente von $R$ assoziiert sind.
    \item
      Zeigen Sie, dass Assoziiertheit eine Äquivalenzrelation ist.
    \item
      Es sei nun $R$ ein Integritätsbereich.
      Zeigen Sie, dass zwei Elemente $a, b \in R$ genau dann assoziiert sind, wenn $(a) = (b)$.
  \end{enumerate}
\end{question}


\begin{solution}
  \begin{enumerate}
    \item
      Ein Element $y \in R$ ist assoziiert zu einem Element $x \in R$, wenn es eine Einheit $\varepsilon \in R^\times$ mit $y = \varepsilon x$ gibt.
  \end{enumerate}
  Für $x, y \in R$ schreiben wir im Folgenden $x \sim y$, wenn $y$ assoziiert zu $x$ ist.
  \begin{enumerate}[resume]
    \item
      Für jedes $x \in R$ ist $x \sim x$ da $x = 1 \cdot x$ mit $1 \in R^\times$.
      Für $x, y \in R$ mit $x \sim y$ gibt es $\varepsilon \in R^\times$ mit $y = \varepsilon x$;
      dann ist $\varepsilon^{-1} \in R^\times$ mit $x = \varepsilon^{-1} y$ und deshalb $y \sim x$.
      Für $x, y, z \in R$ mit $x \sim y$ und $y \sim z$ gibt es $\varepsilon_1, \varepsilon_2 \in R^\times$ mit $y = \varepsilon_1 x$ und $z = \varepsilon_2 y$;
      dann ist $\varepsilon_2 \varepsilon_1 \in R^\times$ mit $z = \varepsilon_2 y = \varepsilon_2 \varepsilon_1 x$ und somit $x \sim z$.
    \item
      Für $x, y \in R$ mit $x \sim y$ gibt es $\varepsilon \in R^\times$ mit $x = \varepsilon y$.
      Dann ist $R \varepsilon = R$ und deshalb
      \[
          (x)
        = \{r x \mid r \in R\}
        = \{r \varepsilon y \mid r \in R\}
        = \{r' y \mid r' \in R \varepsilon\}
        = \{r' y \mid r' \in R\}
        = (y).
      \]
      Ist andererseits $(x) = (y)$ so ist $x \in (y)$ und $y \in (x)$, also gibt es $\varepsilon_1, \varepsilon_2 \in R$ mit $y = \varepsilon_1 x$ und $x = \varepsilon_2 y$.
      Dann ist $y = \varepsilon_1 x = \varepsilon_1 \varepsilon_2 y$, und da $R$ ein Integritätsbereich ist, somit $\varepsilon_1 \varepsilon_2 = 1$.
      Also ist $\varepsilon_1$ eine Einheit mit $\varepsilon_1^{-1} = \varepsilon_2$.
      Da $y = \varepsilon_1 x$ ist $x \sim y$.
  \end{enumerate}
\end{solution}


\begin{question}
  Es sei $R$ ein kommutativer Ring.
  \begin{enumerate}
    \item
      Zeigen Sie, dass für nilpotentes $n \in R$ das Element $1 - n$ eine Einheit ist, und geben Sie $(1 - n)^{-1}$ an.
    \item
      Zeigen Sie, dass für nilpotentes $n \in R$ das Element $1 + n$ eine Einheit ist, und geben Sie $(1 + n)^{-1}$ an.
    \item
      Zeigen Sie, dass für nilpotentes $n \in R$ und jede Einheit $e \in R^\times$ das Element $e + n$ eine Einheit ist, und geben Sie $(e + n)^{-1}$ an.
  \end{enumerate}
\end{question}


\begin{solution}
  \begin{enumerate}
    \item
      Für $k \geq 0$ mit $n^k = 0$ gilt $(1 - n)(1 + n + \dotsb + n^{k-1}) = 1 - n^k = 1$.
      Also ist $1 - n$ eine Einheit mit $(1-n)^{-1} = \sum_{p=0}^{k-1} n^p = \sum_{p=0}^\infty n^p$.
    \item
      Da $n$ nilpotent ist, gilt dies auch für $-n$.
      Nach dem vorherigen Aufgabenteil ist deshalb $1 + n = 1 - (-n)$ eine Einheit mit $(1 + n)^{-1} = (1 - (-n))^{-1} = \sum_{p=0}^\infty (-1)^p n^p$.
    \item
      Es gilt $e + n = e(1 + e^{-1} n)$, und da $n$ nilpotent ist, gilt dies auch für $e^{-1} n$.
      Nach dem vorherigen Teil ist $1 + e^{-1} n$ eine Einheit, und somit $e + n$ als Produkt zweier Einheiten ebenfalls eine Einheit; ferner gilt
      \[
          (e + n)^{-1}
        = e^{-1} (1 + e^{-1} n)^{-1}
        = e^{-1} \sum_{p=0}^\infty (-1)^p (e^{-1} n)^p
        = \sum_{p=0}^\infty (-1)^p e^{-1-p} n^p.
      \]
  \end{enumerate}
\end{solution}


\begin{question}
  Es sei $R$ ein kommutativer Ring und $S \subseteq R$ eine multiplikative Teilmenge.
  \begin{enumerate}
    \item
      Zeigen Sie, dass $R_S$ noethersch ist, wenn $R$ noethersch ist.
    \item
      Zeigen oder widerlegen Sie, dass $R_S$ ein Hauptidealring ist, wenn $R$ ein Hauptidealring ist.
  \end{enumerate}
\end{question}


% TODO: Adding a solution.


\begin{question}
  Es sei $R$ ein Ring und $I \subseteq R$ ein Ideal.
  \begin{enumerate}
    \item
      Zeigen Sie, dass $R/I$ noethersch ist, wenn $R$ noethersch ist.
    \item
      Zeigen Sie widerlegen, dass $R/I$ ein Hauptidealring ist, wenn $R$ ein Hauptidealring ist.
  \end{enumerate}
\end{question}


% TODO: Adding a solution.


\begin{question}
  Für jedes $d \in \Natural$ sei
  \[
              \Integer[\sqrt{-d}]
    \coloneqq \Integer[i\sqrt{d}]
    =         \{a + i \sqrt{d} b \mid a, b \in \Integer \}
    \subseteq \Complex.
  \]
  Es darf im Folgenden ohne Beweis genutzt werden, dass $\Integer[\sqrt{-d}]$ ein Unterring von $\Complex$ ist.
  \begin{enumerate}
    \item
      Zeigen Sie, dass $\Integer[\sqrt{-1}]$ ein euklidischer Ring ist.
    \item
      Zeigen Sie, dass $\Integer[\sqrt{-2}]$ ein euklidischer Ring ist.
    \item
      Zeigen Sie, dass $\Integer[\sqrt{-5}]$ kein euklidischer Ring ist.
  \end{enumerate}
\end{question}


% TODO: Adding a solution.


\begin{question}
  Es sei $R$ ein euklidischer Ring.
  Zeigen Sie, dass $R$ ein Hauptidealring ist.
\end{question}


\begin{solution}
  Als euklidischer Ring ist $R$ insbesondere ein Integritätsbereich.
  Es sei $g \colon R \to \Natural$ die Gradabbildung und $I \subseteq R$ ein Ideal.
  Ist $I = 0$ so ist $I = (0)$, wir betrachten daher den Fall $I \neq 0$.
  Dann gibt es ein bezüglich $g$ minimales $a \in I$, d.h.\ $a \in I$ mit $a \neq 0$ und $g(a) \leq g(x)$ für alle $x \in I$ mit $x \neq 0$.
  Es gilt $(a) \subseteq I$ und es handelt sich bereits um Gleichheit:
  Ist $x \in I$ so gibt es $b, r \in R$ mit $x = ab + r$, und entweder $r = 0$ oder $g(r) < g(a)$.
  Da $r = x - ab \in I$ kann $g(r) < g(a)$ wegen der Minimalität von $a$ nicht eintretten.
  Also ist $r = 0$ und somit $x = ab \in (a)$.
\end{solution}


\begin{question}
  Es sei $K$ ein kommutativer Ring, so dass $K[X]$ ein Hauptidealring ist.
  Zeigen Sie, dass $K$ bereits ein Körper ist.
\end{question}


\begin{solution}
  Wir geben zwei mögliche Beweise:  
  \begin{enumerate}
    \item
      Es sei $a \in K$ mit $a \neq 0$.
      Das Ideal $(a, X)$ ist nach Annahme ein Hauptideal.
      Also gibt es ein Polynom $f \in K[X]$ mit
      \begin{equation}
        \label{eqn: ideal is principal}
        (a, X) = (f).
      \end{equation}
      Wegen Gleichung \eqref{eqn: ideal is principal} gilt $f \mid a$, d.h.\ es gibt $g \in K[X]$ mit $fg = a$.
      Entscheident ist nun die folgende Beobachtung:
      
      \begin{claim}
        \label{claim: degree is additive}
        Die übliche Gradabbildung $\deg \colon K[X] \to \Natural$ ist additiv.
      \end{claim}
      \begin{proof}
        As Hauptidealring ist $K[X]$ inbesondere ein Integritätsbereich.
        Also ist auch der Unterring $K \subseteq K[X]$ ein Integritätsbereich, woraus die Aussage folgt.
      \end{proof}
      Aus Behauptung~\ref{claim: degree is additive} erhalten wir, dass
      \[
          0
        = \deg(a)
        = \deg(fg)
        = \deg(f) + \deg(g).
      \]
      Es muss $\deg(f) = \deg(g) = 0$ gelten und somit bereits $f, g \in K$.
      
      Da $f \in (a, X)$ gibt es $p, q \in K[X]$ mit $f = a p + X q$.
      Da $f \in K$ und $\deg(X q) \geq 1$ ergibt sich durch Vergleich des $0$-ten Koeffizienten, dass $f = f_0 = a_0 p_0 = a p_0$.
      Deshalb gilt bereits $f = a p_0 \in (a)$.
      Wir haben also
      \[
                  (a, X)
        =         (f)
        \subseteq (a)
        \subseteq (a, X)
      \]
      und somit $(a, X) = (a)$.
      
      Es gibt deshalb $h \in K[X]$ mit $X = a h$.
      Durch Gradvergleich erhalten wir, dass
      \[
          1
        = \deg(X)
        = \deg(a h)
        = \deg(a) + \deg(h)
        = 0 + \deg(h)
        = \deg(h)
      \]
      und deshalb $h(X) = b_1 X + b_0$ für $b_1, b_0 \in K$.
      Durch Koeffizientenvergleich erhalten wir aus der Gleichung
      \[
          X
        = a h(X)
        = a (b_1 X + b_0)
        = a b_1 X + a b_0,
      \]
      dass $a b_1 = 1$.
      Das zeigt, dass $a \in A$ eine Einheit ist.
      
    \item
      Der obige Beweis lässt sich leicht ändern.
      Wir zeigen, dass das Ideal $(X)$ maximal ist.
      Ansonsonsten gebe es $a \in K[X]$, so dass $(X) \subsetneq (a, X) \subsetneq K[X]$.
      Da $(a, X) = (a_0, X)$ können o.B.d.A.\ davon ausgehen, dass $a \in K$.
      Wie zuvor ergibt sich, dass $(a, X) = (X)$, was $(X) \subsetneq (a,X)$ widerspricht.
      Also ist $(X)$ maximal, und $K \cong K[X]/(X)$ somit ein Körper.
  \end{enumerate}
  Der erste Beweis hat den Vorteil, dass er für einen beliebigen kommutativen Ring $R$ zeigt, dass $(a, X)$ für $a \in R$ genau dann ein Hauptidealring ist, wenn $a \in R^\times$.
  Somit ist beispielsweise $(2, X) \subseteq \Integer[X]$ kein Hauptideal.
\end{solution}



\begin{question}[subtitle = Euklid]
  Es sei $K$ ein Körper.
  Zeigen Sie, dass es in $K[X]$ unendlich viele normierte, irreduzible Polynome gibt.
\end{question}


\begin{solution}
  Wir nehmen an, dass es nur endlich viele normierte, irreduzible Polynome in $K[X]$ gibt, nämlich $p_1, \dotsc, p_n \in K[X]$.
  Man bemerke, dass $n \geq 1$, da die Polynome $X - a$ für $a \in K$ irreduzibel und normiert sind.
  Für das Element
  \[
    q \coloneqq 1 + p_1 \dotsm p_n \in K[X]
  \]
  gilt dann $\deg q \geq n \geq 1$.
  Es gilt $q \equiv 1 \pmod{p_i}$ für alle $i = 1, \dotsc, n$, und somit $p_i \nmid q$ für alle $i = 1, \dotsc, n$.
  Da die $p_i$ ein Repräsentantensystem der Primelemente von $K[X]$ sind, widerspricht dies der Existenz einer Primfaktorzerlegung von $q$.
\end{solution}



\begin{question}
  Es sei $R$ ein Ring und $I \subseteq R$ ein echtes Ideal.
  Zeigen Sie, dass es ein maximales Ideal $\mideal \subseteq R$ gibt, so dass $I \subseteq \mideal$.
\end{question}


% TODO: Adding a solution.


\begin{question}
  Es seien $R$ und $R'$ zwei kommutative Ringe, $S \subseteq R$ eine multiplikative Teilmenge und $f \colon R \to R'$ ein Ringhomomorphismus.
  \begin{enumerate}
    \item
      Zeigen Sie, dass $S' \coloneqq f(S)$ eine multiplikative Teilmenge von $R'$ ist.
    \item
      Zeigen Sie, dass $f$ einen Ringhomomorphismus $f_S \colon R_S \to R'_{S'}$ induziert.
  \end{enumerate}
\end{question}


\begin{solution}
  \begin{enumerate}
    \item
      Da $1 \in S$ ist $1 = f(1) \in f(S) = S'$.
      Für $s'_1, s'_2 \in S'$ gibt es $s_1, s_2 \in S$ mit $s'_1 = f(s_1)$ und $s'_2 = f(s_2)$, und damit ist auch $s'_1 s'_2 = f(s_1) f(s_2) = f(s_1 s_2) \in f(S) = S'$.
    \item
      Es seien $i \colon R \to R_S$, $r \mapsto r/1$ und $i' \colon R' \to R'_{S'}$, $r' \mapsto r'/1$ die kanonischen Ringhomomorphismen.
      Die Komposition $i' \circ f \colon R \mapsto R'_{S'}$ bildet $s \in S$ auf die Einheit $f(s)/1 \in R'_{S'}$
      ab.
      Nach der universellen Eigenschaft der Lokalisierung induziert $i' \circ f$ einen eindeutigen Ringhomomorphismus $f_S \colon R_S \to R'_{S'}$ mit $f_S i = i' f$, d.h.\ so dass das folgende Diagram kommutiert:
      \[
        \begin{tikzcd}[ampersand replacement = \&]
              R
              \arrow{r}{f}
              \arrow{d}{i}
          \&  R'
              \arrow{d}{i'}
          \\
              R_S
              \arrow{r}{f_S}
          \&  R'_{S'}
        \end{tikzcd}
      \]
  \end{enumerate}
\end{solution}


\begin{question}
  Es sei $R$ ein kommutativer Ring.
  \begin{enumerate}
    \item
      Zeigen Sie, dass für jedes Ideal $\aideal \subseteq R$ die Teilmenge
      \[
                  \aideal[X]
        \coloneqq \left\{
                    \sum_i f_i X^i \in R[X]
                  \,\middle|\,
                    \text{$f_i \in \aideal$ für alle $i$}
                  \right\}
      \]
      ein Ideal in $R[X]$ ist.
    \item
      Zeigen Sie, dass $\pideal[X]$ ein Primideal in $R[X]$, wenn $\pideal \subseteq R$ ein Primideal ist.
    \item
      Zeigen oder widerlegen Sie, dass $\mideal[X]$ notwendigerweise ein maximales Ideal in $R[X]$ ist, wenn $\mideal \subseteq R$ ein maximales Ideal ist.
  \end{enumerate}
\end{question}


\begin{solution}
  \begin{enumerate}
    \item
      Die kanonische Projektion $\pi \colon R \to R/\aideal$, $x \mapsto \overline{x}$ induziert nach der universellen Eigenschaft des Polynomrings $R[X]$ einen Ringhomomorphismus $\varphi \colon R[X] \to (R/\aideal)[X]$ mit $\varphi|_R = \pi$ und $\varphi(X) = \pi(X)$, und dieser ist gegeben durch
      \[
          \varphi\left( \sum_i f_i X^i \right)
        = \sum_i \pi(f_i) X^i
        = \sum_i \overline{f_i} X^i.
      \]
      Für $f = \sum_i f_i X^i \in R[X]$ ist genau dann $f \in \ker \varphi$, wenn $\overline{f_i} = 0$ für alle $i$, also genau dann, wenn $f_i \in \ker \pi = \aideal$ für alle $i$.
      Somit ist $\ker \varphi = \aideal[X]$ ein Ideal in $R[X]$.
    \item
      Es seien $\pi$ und $\varphi$ wie zuvor.
      Wegen der Surjektivität von $\pi$ ist auch $\varphi$ surjektiv.
      Somit induziert $\varphi$ einen Ringisomorphismus
      \[
        \psi \colon R[X]/\pideal[X] \to (R/\pideal)[X],
        \quad
        \overline{\sum_i f_i X^i} \mapsto \sum_i \overline{f_i} X^i.
      \]
      Der Quotient $R/\pideal$ ist ein Integritätsbereich, da $\pideal$ ein Primideal in $R$ ist.
      Somit ist auch $(R/\pideal)[X]$ ein Integritätsbereich.
      Da der Quotient $R[X]/\aideal[X]$ ein Integritätsbereich ist, folgt, dass $\pideal[X]$ ein Primideal in $R[X]$ ist.
    \item
      Ist $K$ ein Körper, so ist $0 \subseteq K$ ein maximales Ideal, und es gilt $\mideal[X] = 0$.
      Der Quotient $K[X]/\mideal[X] \cong (K/0)[X] \cong K[X]$ ist kein Körper, da $0 \neq X \in K[X]$ keine Einheit ist.
      Also ist $\mideal[X]$ nicht maximal in $K[X]$.
      
      Tatsächlich kann $\mideal[X]$ nicht maximal in $R[X]$ sein, da $R[X]/\mideal[X] \cong (R/\mideal)[X]$, aber es keinen Ring $R'$ gibt, so dass $R'[X]$ ein Körper ist (siehe Übung~\ref{qst: polynomial rings are not fields}).
  \end{enumerate}
\end{solution}


\begin{question}
  \label{qst: polynomial rings are not fields}
  Zeigen Sie, dass es keinen Ring $R$ gibt, so dass $R[X]$ ein Körper ist.
\end{question}


\begin{solution}
  Gebe es einen solchen Ring $R$, so wäre $R$ kommutativ, da $R \subseteq R[X]$ ein Unterring ist.
  Es wäre auch $R \neq 0$ da $0[X] = 0$ kein Körper ist.
  Dann wäre aber $0 \neq X \in R[X]$ keine Einheit und $R[X]$ somit kein Körper.
\end{solution}


\begin{question}
  Zeigen Sie, dass $\Integer[i] \cong \Integer[X]/(X^2 + 1)$.
\end{question}


% TODO: Adding a solution.


\begin{question}
  Es sei $R$ ein kommutativer Ring und $f \in R$.
  Zeigen Sie, dass $R_f \cong R[X]/(fX-1)$.
\end{question}


\begin{solution}
  Das Element $\overline{f} \in R[X]/(fX-1)$ ist eine Einheit mit $\overline{f}^{-1} = \overline{X}$ da
  \[
      \overline{f} \, \overline{X}
    = \overline{fX}
    = \overline{1}
    = 1.
  \]
  Nach der universellen Eigenschaft der Lokalisierung $R_f$ induziert der Ringhomomorphismus $R \to R[X] \to R[X]/(fX-1)$ einen Ringhomomorphismus $\varphi \colon R_f \to R[X]/(fX-1)$ mit
  \[
      \varphi\left( \frac{r}{f^k} \right)
    = \frac{\overline{r}}{\overline{f}^k}
    = \overline{r} \overline{X}^k
    = \overline{r X^k}.
  \]
  
  Andererseits induziert der kanonische Ringhomomorphismus $i \colon R \to R_f$, $r \mapsto r/1$ nach der universellen Eigenschaft des Polynomrings $R[X]$ einen eindeutigen Ringhomomorphismus $\tilde{\psi} \colon R[X] \to R_f$ mit $\tilde{\psi}|_R = i$ und $\tilde{\psi}(X) = 1/f$, und dieser ist gegeben durch
  \[
      \tilde{\psi}\left( \sum_i r_i X^i \right)
    = \sum_i \frac{r_i}{f^i}.
  \]
  Dann gilt insbesondere
  \[
      \tilde{\psi}(fX-1)
    = \tilde{\psi}(f) \tilde{\psi}(X) - \tilde{\psi}(1)
    = \frac{f}{1} \frac{1}{f} - \frac{1}{1}
    = 0.
  \]
  Also faktorisiert $\tilde{\psi}$ über einen eindeutigen Ringhomomorphismus $\psi \colon R[X]/(fX-1) \to R_f$ mit $\psi(\overline{p}) = \tilde{\psi}(p)$ für alle $p \in R[X]$, d.h.\ es ist
  \[
      \psi\left( \overline{ \sum_i r_i X^i } \right)
    = \sum_i \frac{r_i}{f^i}
    \qquad
    \text{für alle $\sum_i r_i X^i \in R[X]$}.
  \]
  
  Die beiden Ringhomomorphismen $\varphi$ und $\psi$ sind invers zueinander:
  Für alle $r/f^k \in R_f$ gilt
  \[
      \psi\left( \varphi\left( \frac{r}{f^k} \right) \right)
    = \psi\left( \overline{r X^k} \right)
    = \frac{r}{f^k},
  \]
  und für alle $\sum_i r_i X^i \in R[X]$ gilt
  \[
      \varphi\left( \psi\left( \overline{\sum_i r_i X^i} \right) \right)
    = \varphi\left( \sum_i \frac{r_i}{f^i} \right)
    = \sum_i \varphi\left( \frac{r_i}{f^i} \right)
    = \overline{\sum_i r_i X^i}.
  \]
  Also ist $\varphi$ ein Isomorphismus mit $\varphi^{-1} = \psi$.
\end{solution}


\begin{question}
  Bestimmen Sie die Einheitengruppe $\Integer[i]^\times$.
\end{question}


\begin{solution}
  Ein Element $z \in \Integer[i]$ ist genau dann eine Einheit in $\Integer[i]$, wenn $z \neq 0$ und $z^{-1} \in \Integer[i]$ (hier bezeichnet $z^{-1} = 1/z$ das Inverse von $z$ in $\Complex$).
  Für die Elemente $1, -1, i, -i \in \Integer[i]$ ist dies erfüllt.
  Ist $z \in \Integer[i]$ mit $z \neq 0$ und $z^{-1} \in \Integer[i]$, so ist
  \begin{equation}
    \label{eqn: units in the gaussian integers}
      1
    = |1|^2
    = |z z^{-1}|^2
    = |z|^2 |z^{-1}|.
  \end{equation}
  Für alle $w \in \Integer[i]$ mit $w = a + ib$ gilt $a, b \in \Integer$ und deshalb $|w|^2 = a^2 + b^2 \in \Integer$.
  In \eqref{eqn: units in the gaussian integers} gilt deshalb, dass $|z|^2, |z^{-1}|^2 \in \Integer$, und somit $|z|^2 \in \Integer^\times = \{1,-1\}$.
  Also gilt $|z|^2 = 1$.
  Ist $z = a + ib$ mit $a,b \in \Integer$ so ist also $a^2 + b^2 = 1$ und somit entweder $a = 0$ und $b = \pm 1$, oder $a = \pm 1$ und $b = 0$.
  Es ist also $z \in \{1, -1, i, -i\}$.
  Insgesamt zeigt dies, dass $\Integer[i]^\times = \{1, -1, i, -i\}$.
\end{solution}


\begin{question}
  Formulieren und beweisen Sie den Hilbertschen Basissatz.
\end{question}


% TODO: Adding a solution.


\begin{question}[subtitle = Multiple Choice]
  Entscheiden Sie, welche der folgenden Aussagen wahr oder falsch sind.
  \begin{enumerate}
    \item
      Jeder faktorielle Ring ist unendlich.
  \end{enumerate}
\end{question}


\begin{solution}
  \begin{enumerate}
    \item
      Die Aussage ist falsch.
      Jeder Körper ist ein faktorieller Ring, aber es gibt endliche Körper.
  \end{enumerate}
\end{solution}


\begin{question}
  Es sei $K$ ein Körper.
  Zeigen Sie, dass der Ring $\powerseries{K}{X}$ ein eindeutiges maximales Ideal besitzt.
\end{question}


\begin{solution}
  Wir geben zwei mögliche Beweise an:
  \begin{enumerate}
    \item
      Der Ring $\powerseries{K}{X}$ ein ein euklidisch mit der üblichen Gradabbildung $\Deg$ und somit ein Hauptidealring.
      Also ist jedes Ideal in $\powerseries{K}{X}$ von der Form $(f)$ für ein Element $f \in \powerseries{K}{X}$.
      Ist $f \in \powerseries{K}{X}$ mit $f \neq 0$, so ist $f = \sum_{i=n}^\infty f_i X^i$ mit $f_n \neq 0$ für ein $n \geq 0$.
      Dann ist
      \[
          f
        = \sum_{i=n}^\infty f_i X^i
        = X^n \cdot \sum_{j=0}^\infty f_{n+j} X^j
        = X^n \cdot g.
      \]
      für das Element $g \coloneqq \sum_{j=0}^\infty f_{n+j} X^j$.
      Es gilt $g_0 \neq 0$, also $g_0 \in K^\times$, und somit $g \in \powerseries{K}{X}^\times$.
      Also sind $f$ und $X^n$ assoziiert, und somit $(f) = (X^n)$.
      
      Damit ist gezeigt, dass $0$ und die Ideale $(X^n)$ für $n \geq 0$ die einzigen Ideale in $\powerseries{K}{X}$ sind.
      Falls $(X^n) = (X^m)$ mit $n \leq m$, so sind $X^n$ und $X^m$ assoziiert zueinander, und es gibt $g \in \powerseries{K}{X}^\times$ mit $X^n = g X^m$.
      Dann gilt $g_0 \neq 0$ und somit $\Deg g X^m = m$, weshalb $n = \Deg X^n = \Deg g X^m = m$.
      Die Ideale in $\powerseries{K}{X}$ bilden also eine echte absteigende Kette
      \[
                    (1)
        =           (X^0)
        \supsetneq  (X)
        \supsetneq  (X^2)
        \supsetneq  (X^3)
        \supsetneq  (X^4)
        \supsetneq \dotsb
        \supsetneq  0.
      \]
      Inbesondere ist $(X)$ das eindeutige maximale Ideal in $\powerseries{R}{X}$.
    \item
      Es sei $\mideal \coloneqq \{f \in \powerseries{K}{X} \mid f_0 = 0\}$.
      Die Abbildung $\varphi \colon \powerseries{K}{X} \to K$, $f \mapsto f_0$ ist ein Ringhomomorphismus mit $\ker \varphi = \mideal$, weshalb $\mideal$ ein Ideal in $\powerseries{K}{X}$ ist.
      Da
      \[
              K
        =     \im \varphi
        \cong \powerseries{K}{X}/\ker \varphi
        =     \powerseries{K}{X}/\mideal
      \]
      ein Körper ist, ist $\mideal$ bereits ein maximales Ideal.

      Gebe es ein maximales Ideal $\mideal' \subseteq \powerseries{K}{X}$ mit $\mideal' \neq \mideal$, so würde wegen der Maximalität von $\mideal'$ inbesondere $\mideal' \nsubseteq \mideal$ gelten.
      Dann gebe es $f \in \mideal'$ mit $f \notin \mideal$, also $f_0 \neq 0$ und somit $f \in K^\times$.
      Dann würde aber $f \in \powerseries{K}{X}^\times$ gelten, und somit $(1) = (f) \subseteq \mideal'$, was im Widerspruch dazu stünde, dass $\mideal'$ ein echtes Ideal in $\powerseries{K}{X}$ ist.
  \end{enumerate}
\end{solution}
