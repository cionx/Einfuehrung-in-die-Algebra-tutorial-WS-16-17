\section{Ringtheorie}


\begin{question}
  Es sei $R$ ein Ring.
  Zeigen Sie, dass es einen eindeutigen Ringhomomorphismus $\Integer \to R$ gibt.
  (Dies bedeutet, dass der Ring $\Integer$ ein Initialobjekt in der Kategorie der Ringe ist.)
\end{question}


\begin{solution}
  Ist $\phi \colon \Integer \to R$ ein Ringhomomorphismus, so ist $\phi(1_\Integer) = 1_R$.
  Für alle $n \in \Integer$ ist damit
  \[
      \phi(n)
    = \phi(n \cdot 1_\Integer)
    = n \cdot \phi(1_\Integer)
    = n \cdot 1_R.
  \]
  Also ist $\phi$ eindeutig.
  Durch direktes Nachrechnen ergibt sich auch, dass $\psi \colon \Integer \to R$ mit
  \[
    \psi(n) \coloneqq n \cdot 1_R
    \quad
    \text{für alle $n \in \Integer$}
  \]
  ein Ringhomomorphismus ist.
\end{solution}


\begin{question}
  Es sei $R$ ein kommutativer Ring.
  \begin{enumerate}
    \item
      Zeigen Sie, dass ein Ideal $\pideal \subseteq R$ genau dann prim ist, wenn $R/\pideal$ ein Integritätsbereich ist.
    \item
      Zeigen Sie, dass ein Ideal $\mideal \subseteq R$ genau dann maximal ist, wenn $R/\mideal$ ein Körper ist. 
  \end{enumerate}
\end{question}


\begin{solution}
  Dies ist eine Standardaussage, deren Beweis sich in jedem Algebra-Buch findet.
\end{solution}


\begin{question}
  Ist $R$ ein Ring, so ist
  \[
              R^\times
    \coloneqq \{x \in R \mid \text{$x$ ist eine Einheit}\}
  \]
  die \emph{Einheitengruppe} von $R$.
  Zeigen Sie:
  \begin{enumerate}
    \item
      Ist $R$ ein Ring, so bildet $R^\times$ bezüglich der Multiplikation aus $R$ eine Gruppe.
    \item
      Sind $R$ und $S$ zwei Ringe und ist $\phi \colon R \to S$ ein Ringhomomorphismus, so induziert $\phi$ per Einschränkung einen Gruppenhomomorphismus
      \[
        \phi^\times \colon R^\times \to S^\times,
        \quad
        x \mapsto \phi(x).
      \]
    \item
      Für jeden Ring $R$ gilt $\id_R^\times = \id_{R^\times}$, und für alle Ringhomomorphismen $\phi \colon R_1 \to R_2$ und $\psi \colon R_2 \to R_3$ gilt $(\psi \phi)^\times = \psi^\times \phi^\times$.
  \end{enumerate}
\end{question}


% TODO: Lösung hinzufügen


\begin{question}[subtitle = Urbilder von Idealen]
  Es seien $R$ und $S$ zwei kommutative Ringe und $\phi \colon R \to S$ ein Ringhomomorphismus.
  \begin{enumerate}
    \item
      Zeigen Sie, dass für jedes Ideal $\aideal \subseteq S$ das Urbild $\phi^{-1}(\aideal)$ ein Ideal in $R$ ist.
    \item
      Entscheiden Sie, ob $\phi^{-1}(\pideal)$ ein Primideal ist, wenn $\pideal \subseteq S$ ein Primideal ist.
    \item
      Entscheiden Sie, ob $\phi^{-1}(\mideal)$ ein maximales Ideal ist, wenn $\mideal \subseteq S$ ein maximales Ideal ist.
  \end{enumerate}
\end{question}


\begin{solution}
  \begin{enumerate}
    \item
      Es sei $\pi \colon S \to S/\aideal$, $s \mapsto \overline{s}$ die kanonische Projektion.
      Dann ist $\pi \phi$ ein Ringhomomorphismus und somit $\ker (\pi \phi) = \phi^{-1}(\ker \pi) = \phi^{-1}(\aideal)$ ein Ideal in $R$.
    \item
      Die Aussage gilt:
      Es sei $\pi \colon S \to S/\pideal$, $s \mapsto \overline{s}$ die kanonische Projektion und $\qideal \coloneqq \phi^{-1}(\pideal)$.
      Der Quotient $S/\pideal$ ist ein Integritätsbereich, da $\pideal$ ein Primideal ist.
      Nach dem vorherigen Aufgabenteil ist $\qideal$ ein Ideal in $R$, und da $\ker (\pi \phi) = \phi^{-1}(\ker \pi) = \phi^{-1}(\pideal) = \qideal$ induziert $\pi \phi$ einen injektiven Ringhomomorphismus
      \[
        \psi \colon R/\qideal \to S/\pideal
        \quad
        \overline{r} \mapsto \overline{\phi(r)}.
      \]
      Der Ring $\im (\pi \phi) \subseteq S/\pideal$ ist als Unterring eines Integritätsbereichs ebenfalls ein Integritätsbereich.
      Somit ist $R/\qideal \cong \im(\pi \phi)$ ein Integritätsbereich, also $\qideal$ ein Primideal.
    \item
      Die Aussage gilt nicht:
      Es sei etwa $\phi \colon \Integer \to \Rational$ die kanonische Inklusion.
      Dann ist $\mideal \coloneqq 0$ ein maximales Ideal in $\Rational$, aber $\phi^{-1}(0) = 0$ ist kein maximales Ideal in $\Integer$, da $\Integer/\mideal \cong \Integer$ kein Körper ist.
  \end{enumerate}
\end{solution}


\begin{question}
  Geben Sie ein Beispiel für einen kommutativen Ring $R$ und eine Teilmenge $S \subseteq R$ mit den folgenden Eigenschaften:
  \begin{itemize}
    \item
      $S$ ist abgeschlossen unter der Addition und Multiplikation von $R$, d.h.\ für alle $s_1, s_2 \in S$ ist auch $s_1 + s_2 \in S$ und $s_1 s_2 \in S$.
    \item
      Zusammen mit der Einschränkung der Addition und Multiplikation aus $R$ ist $S$ ebenfalls ein (notwendigerweise kommutativer) Ring.
    \item
      $S$ ist kein Unterring von $R$.
  \end{itemize}
\end{question}


\begin{solution}
  Es sei $R = \Integer \times \Integer$ und $S = \Integer \times 0 = \{(n,0) \mid n \in \Integer\}$.
  Offenbar ist $S$ unter der Addition und Multiplikation abgeschlossen.
  Zusammen mit der Einschränkung dieser Operationen bildet $S$ einen kommutativen Ring, für den $S \cong \Integer$ gilt.
  Da $1_R = (1,1) \notin S$ ist $S$ allerdings kein Unterring von $R$.
\end{solution}
