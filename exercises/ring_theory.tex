\section{Ringtheorie}


\begin{question}[subtitle = Urbilder von Idealen]
  Es seien $R$ und $S$ zwei kommutative Ringe und $\phi \colon R \to S$ ein Ringhomomorphismus.
  \begin{enumerate}
    \item
      Zeigen Sie, dass für jedes Ideal $\aideal \subseteq S$ das Urbild $\phi^{-1}(\aideal)$ ein Ideal in $R$ ist.
    \item
      Entscheiden Sie, ob $\phi^{-1}(\pideal)$ ein Primideal ist, wenn $\pideal \subseteq S$ ein Primideal ist.
    \item
      Entscheiden Sie, ob $\phi^{-1}(\mideal)$ ein maximales Ideal ist, wenn $\mideal \subseteq S$ ein maximales Ideal ist.
  \end{enumerate}
\end{question}


\begin{solution}
  \begin{enumerate}
    \item
      Es sei $\pi \colon S \to S/\aideal$, $s \mapsto \overline{s}$ die kanonische Projektion.
      Dann ist $\pi \phi$ ein Ringhomomorphismus und somit $\ker (\pi \phi) = \phi^{-1}(\ker \pi) = \phi^{-1}(\aideal)$ ein Ideal in $R$.
    \item
      Die Aussage gilt:
      Es sei $\pi \colon S \to S/\pideal$, $s \mapsto \overline{s}$ die kanonische Projektion und $\qideal \coloneqq \phi^{-1}(\pideal)$.
      Der Quotient $S/\pideal$ ist ein Integritätsbereich, da $\pideal$ ein Primideal ist.
      Nach dem vorherigen Aufgabenteil ist $\qideal$ ein Ideal in $R$, und da $\ker (\pi \phi) = \phi^{-1}(\ker \pi) = \phi^{-1}(\pideal) = \qideal$ induziert $\pi \phi$ einen injektiven Ringhomomorphismus
      \[
        \psi \colon R/\qideal \to S/\pideal
        \quad
        \overline{r} \mapsto \overline{\phi(r)}.
      \]
      Der Ring $\im (\pi \phi) \subseteq S/\pideal$ ist als Unterring eines Integritätsbereichs ebenfalls ein Integritätsbereich.
      Somit ist $R/\qideal \cong \im(\pi \phi)$ ein Integritätsbereich, also $\qideal$ ein Primideal.
    \item
      Die Aussage gilt nicht:
      Es sei etwa $\phi \colon \Integer \to \Rational$ die kanonische Inklusion.
      Dann ist $\mideal \coloneqq 0$ ein maximales Ideal in $\Rational$, aber $\phi^{-1}(0) = 0$ ist kein maximales Ideal in $\Integer$.
  \end{enumerate}
\end{solution}