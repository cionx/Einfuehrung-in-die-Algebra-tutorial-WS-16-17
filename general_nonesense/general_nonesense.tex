\documentclass[a4paper, 10pt, numbers=noenddot]{scrartcl}

\usepackage{../generalstyle}
\usepackage{specificstyle}

\title{Lösungen zu Zettel 1}
\author{Jendrik Stelzner}
\date{\today}

\begin{document}
\maketitle










\addtocounter{section}{1}










\section{}

Zur Belustigung bezeichnen wir die gegebene Abbildung mit
\begin{align*}
              \Xi
  \colon&\,   \{ \varphi  \colon R[T] \to S \mid \text{$\varphi$ ist ein Ringhomomorphismus} \}
  \\
        &\,   \phantom{\varphi} \to \{ \psi \colon R \to S        \mid \text{$\psi$ ist ein Ringhomomorphismus} \}
                                    \times S,
  \\
         &\,  \varphi
  \mapsto     (\varphi|_R, \varphi(T)).
\end{align*}





\subsection{}

Es sei $\varphi \colon R[T] \to S$ ein Ringhomomorphismus.
Es seien $\psi \coloneqq \varphi|_R$ und $s \coloneqq \varphi(T)$.
Für jedes $\sum_i a_i T^i \in R[T]$ gilt dann
\[
    \varphi\left( \sum_i a_i T^i \right)
  = \sum_i \varphi(a_i) \varphi(T)^i
  = \sum_i \psi(a_i) s^i.
\]
Deshalb ist $\varphi$ durch $\psi$ und $s$ schon eindeutig bestimmt.
Somit ist $\Xi$ injektiv.





\subsection{}

Falls $(\psi, s)$ im Bild von $\Xi$ liegt, so gibt es einen Ringhomomorphismus $\varphi \colon R[T] \to S$ mit $\psi = \varphi|_R$ und $s = \varphi(T)$.
Dann gilt
\[
    \psi(r) s
  = \varphi(r) \varphi(T)
  = \varphi(r T)
  = \varphi(T r)
  = \varphi(T) \varphi(r)
  = s \psi(r)
  \qquad
  \text{für alle $r \in R$},
\]
wobei sich die mittlere Gleichheit aus der Kommutativität von $R[T]$ ergibt.

Es sei nun andererseits $(\psi, s)$ ein Paar bestehend aus einem Ringhomomorphismus $\psi \colon R \to S$ und einem Element $s \in S$, so dass $\psi(r) s = s \psi(r)$ für alle $r \in R$.
Um zu zeigen, dass $(\psi, s)$ im Bild von $\Xi$ liegt, müssen wir zeigen, das es einen Ringhomomorphismus $\varphi \colon R[T] \to S$ gibt, so dass $\psi = \varphi|_R$ und $s = \varphi(T)$.
Aus dem vorherigen Aufgabenteil wissen wir, wie $\varphi$ aussehen muss.
Wir definieren deshalb die Abbildung
\[
  \varphi \colon R[T] \to S,
  \quad
  \sum_i a_i T^i \mapsto \sum_i \psi(a_i) s^i.
\]
Da $\varphi|_R = \psi$ und $\varphi(T) = s$ gilt es nur noch zeigen, dass $\varphi$ ein Ringhomomorphismus ist.

Für alle $f, g \in R[T]$ mit $f = \sum_i a_i T^i$ und $g = \sum_i b_i T^i$ gelten
\begin{align*}
      \varphi(f + g)
  &=  \varphi\left( \sum_i a_i T^i + \sum_i b_i T^i \right)
   =  \varphi\left( \sum_i (a_i + b_i) T^i \right)
   =  \sum_i (a_i + b_i) s^i
  \\
  &=  \sum_i a_i s^i + \sum_i b_i s^i
   =  \varphi\left( \sum_i a_i T^i \right) + \varphi\left( \sum_i b_i TWi \right)
   =  \varphi(f) + \varphi(g),
\end{align*}
und
\begin{align*}
    \varphi(f \cdot g)
  &= \varphi\left( \left( \sum_i a_i T^i \right) \cdot \left( \sum_i b_i T^i \right) \right)
   = \varphi\left( \sum_i \sum_{j + k = i} a_j b_k T^i \right)
  \\
  &= \sum_i \psi\left( \sum_{j + k = i} a_j b_k \right) s^i
   = \sum_i \sum_{j + k = i} \psi(a_j) \psi(b_k) s^i
  \\
  &= \left( \sum_i \psi(a_i) s^i \right) \cdot \left( \sum_i \psi(b_i) s^i \right)
   = \varphi\left( \sum_i a_i T^i \right) \cdot \varphi\left( \sum_i b_i T^i \right)
  \\
  &= \varphi(f) \cdot \varphi(g).
\end{align*}
Also ist $\varphi$ additiv und multiplikativ.
Da
\[
    \varphi(1)
  = \varphi(1 \cdot T^0)
  = \psi(1) s^0
  = 1 \cdot 1
  = 1
\]
ist $\varphi$ auch unitär.
Insgesamt zeigt dies, dass $\varphi$ ein Ringhomomorphismus ist.























\end{document}
