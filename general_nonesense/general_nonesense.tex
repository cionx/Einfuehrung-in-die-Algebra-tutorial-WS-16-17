\documentclass[a4paper, 10pt, numbers=noenddot]{scrartcl}

\usepackage{../generalstyle}
\usepackage{specificstyle}

\title{Lösungen zu Zettel 1}
\author{Jendrik Stelzner}
\date{\today}

\begin{document}
\maketitle










\section{}

Für je zwei Mengen $X$ und $Y$ sei $F(X,Y)$ die Menge der Funktionen $X \to Y$.
Wir bezeichnen die übliche Adjunktionsabbildung mit
\[
  \Phi \colon F(R \times A, A) \to F(R, F(A,A)),
  \quad
  h \mapsto (r \mapsto h(r, -))
\]
wobei wir für jede Funktion $h \colon R \times A \to A$ und jedes $r \in R$ mit $h(r,-)$ die Funktion
\[
  h(r, -) \colon A \to A,
  \quad
  a \mapsto h(r,a)
\]
bezeichnen.
Die Funktion $\Phi$ ist eine Bijektion, und ihr Inverses ist durch
\[
  \Psi \colon F(R, F(A,A)) \to F(R \times A, A),
  \quad
  h \mapsto ( (r,a) \mapsto h(r)(a) )
\]
gegeben.





\subsection{}

Es sei $\mu \in F(R \times A, A)$ und $f \coloneqq \Phi(f) \in F(R, F(A,A))$.
Es gilt zu zeigen, dass $\mu$ genau dann eine $R$-Modulstruktur auf $A$ ist, wenn $f$ ein Ringhomomorphismus nach $\End(A)$ ist.
Das $\mu$ eine $R$-Modulstruktur auf $A$ ist, bedeutet, dass die folgenden Bedingungen erfüllt sind:
\begin{enumerate}[label = (M\arabic*)]
  \item
    \label{enum: module structure is additiv in second argument}
    Für alle $r \in R$ und $a_1, a_2 \in A$ gilt $\mu(r, a_1 + a_2) = \mu(r, a_1) + \mu(r, a_2)$.
  \item
    Für alle $r_1, r_2 \in R$ und $a \in A$ gilt $\mu(r_1 + r_2, a) = \mu(r_1, a) + \mu(r_2, a)$.
  \item
    Für alle $r_1, r_2 \in R$ und $a \in A$ gilt $\mu(r_1 \cdot r_2, a) = \mu(r_1, \mu(r_2, a))$.
  \item
    Für alle $a \in A$ ist $\mu(1, a) = a$.
\end{enumerate}
Dass $f$ ein Ringhomomorphismus ist, bedeutet, dass die folgenden Bedingungen erfüllt sind:
\begin{enumerate}[label = (R\arabic*)]
  \item
    \label{enum: ring morphism into the endomorphisms is well defined}
    Für alle $r \in R$ gilt $f(r) \in \End(A)$, d.h.\ für alle $r \in R$ und $a_1, a_2 \in A$ gilt $f(r)(a_1 + a_2) = f(r)(a_1) + f(r)(a_2)$.
  \item
    Für alle $r \in R$ gilt $f(r_1, r_2) = f(r_1) + f(r_2)$, d.h.\ für alle $r_1, r_2 \in R$ und $a \in A$ gilt $f(r_1 + r_2)(a) = f(r_1)(a) + f(r_2)(a)$.
  \item
    Für alle $r \in R$ gilt $f(r_1 \cdot r_2) = f(r_1) \circ f(r_2)$, d.h.\ für alle $r_1, r_2 \in R$ und $a \in A$ gilt $f(r_1 \cdot r_2)(a) = f(r_1)( f(r_2)(a) )$.
  \item
    Es gilt $f(1) = \id_A$, d.h.\ für alle $a \in A$ gilt $f(1)(a) = a$.
\end{enumerate}
Da $\mu(r,a) = f(r)(a)$ sind die Bedingungen in gegebener Nummerierung paarweise äquivalent zueinander (bespielsweise ist \ref{enum: module structure is additiv in second argument} äquivalent zu \ref{enum: ring morphism into the endomorphisms is well defined}).
Also erfüllt $\mu$ alle vier Bedingungen genau dann, wenn $f$ alle vier Bedingungen erfüllt.
Also ist $\mu$ genau dann eine Modulstruktur, wenn $f$ ein Ringhomomorphismus nach $\End(A)$ ist.





\subsection{}

Wir bezeichnen die gegebene geordnete Basis $B$ von $V$ mit $B = (b_1, \dotsc, b_n)$.

Die $K$-Vektorraumstruktur auf $V$ entspricht einem Ringhomomorphismus
\[
  f \colon K \to \End(V),
  \quad
  K \mapsto (v \mapsto \lambda \cdot v).
\]
Das Bild von $f$ liegt bereits im Unterring $\End_K(V)$, denn für jedes $\lambda \in K$ ist die Abbildung $f(\lambda) \colon V \to V$, $v \mapsto \lambda \cdot v$ nicht nur additiv, sondern auch $K$-linear (denn für alle $\mu \in K$ ist $f(\lambda)(\mu v) = \lambda \mu v = \mu \lambda v = \mu f(\lambda)(v)$.)
Dashalb können wir $f$ als einen Ringhomomorphismus $K \to \End_K(V)$ auffassen.
Der angegebene Ringisomorphismus $g \colon \End_K(V) \to \Mat_n(K)$ ordnet jedem Endomorphismus $L \in \End_K(V)$ die darstellende Matrix $\basismatrix_B(L) \in \Mat_n(K)$ zu.
Für jedes $\lambda \in K$ ist $f(\lambda) = \lambda \id_V$ und somit
\[
    g(f(\lambda))
  = g(\lambda \id_V)
  = \basismatrix_B(\lambda \id_V)
  = \lambda \basismatrix_B(\id_V)
  = \lambda I_n.
\]
Die Komposition $K \xrightarrow{f} \End_K(V) \xrightarrow{g} \Mat_n(K)$ bildet als $\lambda \in K$ auf $\lambda I_n$ ab.





\subsection{}

Damit der Ausdruck $\End_{\End_R(A)}(A)$ Sinn ergibt, müssen wir zunächst eine $\End_R(A)$-Mo\-dul\-struk\-tur auf $A$ definieren.
Eine $\End_R(A)$-Modulstruktur auf $A$ zu definieren ist äquivalent dazu, einen Ringhomomorphismus $\End_R(A) \to \End(A)$ anzugeben.

Der Ringhomomorphismus, der hier betrachtet werden soll, ist die kanonische Inklusion $i \colon \End_R(A) \to \End(A)$, $f \mapsto f$.
(Dies wird in der Aufgabenstellung recht schlecht -- nämlich gar nicht -- angegeben.)
Die $\End_R(A)$-Modulstruktur auf $A$, die $i$ entspricht, ist durch
\[
    f \cdot a
  = i(f)(a)
  = f(a)
  \qquad
  \text{für alle $f \in \End_R(A)$, $a \in A$}
\]
gegeben.

Für $g \in \End(A)$ gilt genau dann $g \in \End_{\End_R(A)}$, wenn $g(f \cdot a) = f \cdot g(a)$ für alle $g \in G$, d.h.\ wenn $g(f(a)) = f(g(a))$ für alle $a \in A$.
Also ist
\[
    \End_{\End_R(A)}
  = \{ g \in \End(A) \mid \text{$gf = fg$ für alle $f \in \End_R(A)$} \}.
\]

Ist $g$ im Bild des zu $\mu$ gehörigen Ringhomomorphismus $R \to \End(A)$, $r \mapsto \mu(r, -)$, so gibt es ein $r \in R$ mit $g = \mu(r, -)$.
Für alle $a \in A$ ist dann $g(a) = \mu(r,-)(a) = \mu(r,a) = r \cdot a$, und für $f \in \End_R(A)$ ist deshalb
\[
    g(f(a))
  = r \cdot f(a)
  = f(r \cdot a)
  = f(g(a))
  \qquad
  \text{für alle $a \in A$}.
\]
Also ist dann $g \in \End_{\End_R(A)}(A)$.





\subsection{}

Wir bemerken zunächst, dass die Aussage im Fall $V = 0$ nicht gilt:
Dann wäre nämlich $\End(V) = 0$ und für $\End_{\End_K(V)}(V) \subseteq \End(V)$ somit auch $\End_{\End_K(V)}(V) = 0$.
Es gilt aber $K \ncong 0$.
Wir arbeiten daher im Folgenden unter der zusätzlichen Annahme, dass $V \neq 0$.

Nach dem vorherigen Aufgabenteil gilt $\End_{\End_K(V)}(V) \subseteq \End_K(V)$, da $K$ kommutativ ist.
Deshalb ist
\begin{equation}
  \label{eqn: desired ring as the center}
  \begin{aligned}
        \End_{\End_K(V)}(V)
    &=  \{
          f \in \End(V)
        \mid
          \text{$gf = fg$ für alle $g \in \End_K(V)$}
        \}
    \\
    &=  \{
          f \in \End_K(V)
        \mid
          \text{$gf = fg$ für alle $g \in \End_K(V)$}
        \}.
  \end{aligned}
\end{equation}
Unter dem Isomorphismus $\basismatrix_B \colon \End_K(V) \cong \Mat_n(K)$, $f \mapsto \basismatrix_B(f)$ entspricht der Unterring $\End_{\End_K(V)}(V) \subseteq \End_K(V)$ dem Unterring
\[
            \{
              A \in \Mat_n(K)
            \mid
              \text{$AB = BA$ für alle $B \in \Mat_n(K)$}
            \}
  \subseteq \Mat_n(K).
\]
Wir bezeichnen diesen Unterring mit $\Center(\Mat_n(K))$.

\begin{remark}
  Allgemein wird für einen Ring $R$ die Teilmenge
  \[
              \Center(R)
    \coloneqq \{ r \in R \mid \text{$rs = sr$ für alle $s \in R$} \}
  \]
  als das \emph{Zentrum} von $R$ bezeichnet.
  Es handelt sich hierbei um einen kommutativen Unterring von $R$.
  In \eqref{eqn: desired ring as the center} haben wir formuliert, dass $\End_{\End_K(V)}(V) = Z(\End_K(V))$.
\end{remark}

Wir haben nun das folgende kommutative Diagram:
\[
  \begin{tikzcd}
      {}
    & \Mat_n(K)
      \arrow[hookleftarrow]{r}
    & \Center(\Mat_n(K))
    \\
      K
      \arrow{ur}{\psi}
      \arrow[swap]{dr}{\phi}
    & {}
    & {}
    \\
      {}
    & \End_K(V)
      \arrow[swap]{uu}{\basismatrix_B}
      \arrow[hookleftarrow]{r}
    & \End_{\End_K(V)}(V)
      \arrow[swap]{uu}{\basismatrix_B}
  \end{tikzcd}
\]
Dabei ist $\phi \colon K \to \End_K(V)$, $\lambda \mapsto \lambda \id_V$ und $\psi \colon K \to \Mat_n(K)$, $\lambda \mapsto \lambda I$.

Da $\im \phi \subseteq \End_{\End_K(V)}$ ist auch $\im \psi \subseteq \Center(\Mat_n(K))$.
Die Bijektivität der Einschränkung $\phi \colon K \to \End_{\End_K(V)}$ ist äquivalent dazu, dass die Einschränkung $\psi \colon K \to \Center(\Mat_n(K))$ ein Isomorphismus ist.
Da $V \neq 0$ ist $n = \dim V \geq 1$ und $\psi$ deshalb injektiv.
Für die Surjektivität von $\psi$ müssen wir das folgende Lemma beweisen:

\begin{lemma}
  \label{lem: center of matrix rings}
  Es gilt $Z(\Mat_n(K)) = \{ \lambda I \mid \lambda \in K\}$.
\end{lemma}
\begin{proof}
  Es sei $D = (d_{ij}) \in Z(\Mat_n(K))$.
  Für alle $1 \leq i,j \leq n$ sei $E_{ij} \in \Mat_n(K)$ die Matrix, deren $(i,j)$-ter Eintrag $1$ ist, und deren Einträge sonst alle $0$ sind (die $j$-te Spalte von $E_{ij}$ ist also $e_i$, und alle anderen Spalten sind $0$).
  
  Wir zeigen zunächst, dass $D$ eine Diagonalmatrix ist.
  Hierfür fixieren wir ein $1 \leq i \leq n$.
  Die Matrix $D E_{ii}$ ergibt sich aus $D$, indem bis auf die $i$-te Spalte alle Spalten gelöscht werden, d.h.\ es gilt
  \[
      D E_{ii}
    = \begin{pmatrix}
        0       & \cdots  & 0 & d_{1i}    & 0       & \cdots  & 0  \\
        0       & \cdots  & 0 & d_{2i}    & 0       & \cdots  & 0  \\
        \vdots  & \ddots  & 0 & \vdots    & \vdots  & \ddots  & 0  \\
        0       & \cdots  & 0 & d_{n-1,i} & 0       & \cdots  & 0  \\
        0       & \cdots  & 0 & d_{ni}    & 0       & \cdots  & 0
      \end{pmatrix},
  \]
  wobei es sich bei der mittig gesetzen Spalte um die $i$-te handelt.
  Die Matrix $E_{ii} D$ ergibt sich auch $D$, indem bis auf die $i$-te Zeile alle Zeilen gelöscht werden, d.h.\ es gilt
  \[
      D E_{ii}
    = \begin{pmatrix}
        0       & 0       & \cdots  & 0         & 0       \\
        \vdots  & \vdots  & \ddots  & \vdots    & \vdots  \\
        0       & 0       & \cdots  & 0         & 0       \\
        d_{i1}  & d_{i2}  & \cdots  & d_{i,n-1} & d_{in}  \\
        0       & 0       & \cdots  & 0         & 0       \\
        \vdots  & \vdots  & \ddots  & \vdots    & \vdots  \\
        0       & 0       & \cdots  & 0         & 0
      \end{pmatrix},
  \]
  wobei es sich bei der mittig gesetzen Zeile um die $i$-te handelt.
  Da $D \in \Center(\Mat_n(K))$ ist $D E_{ii} = E_{ii} D$, die beiden obigen Matrizen stimmen also überein.
  Durch Vergleich der Einträge erhalten wir, dass $d_{ij} = 0$ für alle $j \neq i$ und $d_{ki} = 0$ für alle $k \neq i$.
  Da dies für jedes $1 \leq i \leq n$ gilt, erhalten wir insgesamt, dass $d_{ij} = 0$ für alle $i \neq j$.
  Somit ist $D$ eine Diagonalmatrix.
  
  Um zu zeigen, dass $D$ bereits eine Skalarmatrix ist (d.h.\ ein skalares Vielfaches der Einheitsmatrix) müssen wir noch zeigen, dass $d_{ii} = d_{jj}$ für alle $1 \leq i,j \leq n$.
  Hierfür bemerken wir, dass
  \[
    D E_{ij} = d_{ii} E_{ij}
    \quad\text{und}\quad
    E_{ij} D = d_{jj} E_{ij}
    \qquad
    \text{für alle $1 \leq i,j \leq n$},
  \]
  Da $D E_{ij} = E_{ij} D$ für alle $1 \leq i,j \leq n$ ergibt sich, dass $d_{ii} E_{ij} = d_{jj} E_{ij}$ für alle $1 \leq i,j \leq n$, und somit $d_{ii} = d_{jj}$.
\end{proof}


\begin{remark}
  Allgemein gilt für jeden Ring $R$, dass
  \[
      \Center(\Mat_n(R))
    = \left\{
          \begin{pmatrix}
            r &         &   \\
              & \ddots  &   \\
              &         & r
          \end{pmatrix}
        \,
        \middle|
        \,
          r \in \Center(R)
      \right\}.
  \]
  Hierfür muss $R$ weder kommutativ sein, noch eine $1$ haben.
  
  Für kommutative Ringe mit $1$ gilt der Beweis von Lemma~\ref{lem: center of matrix rings} unverändert (man muss nur $K$ durch $R$ ersetzen.)
  
  Für nicht-kommutative Ringe mit $1$ muss man noch einen weiteren Schritt hinzufügen, um einzusehen, dass die Skalare aus dem Zentrum $\Center(R)$ stammen müssen.
  
  Für Ringe $1$ lässt sich ``künstlich eine $1$ hinzufügen'', wodurch sich die Aussage auf den Fall von Ringen mit $1$ zurückführen lässt.
\end{remark}





\subsection{}

Es sei $R$ ein nicht-kommutativer Ring.
Als $R$-Modul betrachten wir $R$ selbst, d.h.\ wir betrachten $A \coloneqq R$, und die Multiplikation der $R$-Modulstruktur auf $A$ entspricht der Ringmultiplikation von $R$.

Da $R$ nicht kommutativ ist gibt es $r_1, r_2 \in R$ mit $r_1 r_2 \neq r_2 r_1$.
Die Abbildung
\[
  \lambda_{r_1} \colon A \to A,
  \quad
  a \mapsto r_1 \cdot a
\]
ist deshalb nicht $R$-linear, denn
\[
        \lambda_{r_1}(r_2 \cdot 1)
  =     \lambda_{r_1}(r_2)
  =     r_1 r_2
  \neq  r_2 r_1
  =     r_2 (r_1 \cdot 1)
  =     r_2 \lambda_{r_1}(1).
\]










\section{}

Zur Belustigung bezeichnen wir die gegebene Abbildung mit
\begin{align*}
              \Xi
  \colon&\,   \{ \varphi  \colon R[T] \to S \mid \text{$\varphi$ ist ein Ringhomomorphismus} \}
  \\
        &\,   \phantom{\varphi} \to \{ \psi \colon R \to S        \mid \text{$\psi$ ist ein Ringhomomorphismus} \}
                                    \times S,
  \\
         &\,  \varphi
  \mapsto     (\varphi|_R, \varphi(T)).
\end{align*}





\subsection{}

Es sei $\varphi \colon R[T] \to S$ ein Ringhomomorphismus.
Es seien $\psi \coloneqq \varphi|_R$ und $s \coloneqq \varphi(T)$.
Für jedes $\sum_i a_i T^i \in R[T]$ gilt dann
\[
    \varphi\left( \sum_i a_i T^i \right)
  = \sum_i \varphi(a_i) \varphi(T)^i
  = \sum_i \psi(a_i) s^i.
\]
Deshalb ist $\varphi$ durch $\psi$ und $s$ schon eindeutig bestimmt.
Somit ist $\Xi$ injektiv.





\subsection{}

Falls $(\psi, s)$ im Bild von $\Xi$ liegt, so gibt es einen Ringhomomorphismus $\varphi \colon R[T] \to S$ mit $\psi = \varphi|_R$ und $s = \varphi(T)$.
Dann gilt
\[
    \psi(r) s
  = \varphi(r) \varphi(T)
  = \varphi(r T)
  = \varphi(T r)
  = \varphi(T) \varphi(r)
  = s \psi(r)
  \qquad
  \text{für alle $r \in R$},
\]
wobei sich die mittlere Gleichheit aus der Kommutativität von $R[T]$ ergibt.

Umgekehrt sei $(\psi, s)$ ein Paar bestehend aus einem Ringhomomorphismus $\psi \colon R \to S$ und einem Element $s \in S$, so dass $\psi(r) s = s \psi(r)$ für alle $r \in R$.
Um zu zeigen, dass $(\psi, s)$ im Bild von $\Xi$ liegt, müssen wir zeigen, das es einen Ringhomomorphismus $\varphi \colon R[T] \to S$ gibt, so dass $\psi = \varphi|_R$ und $s = \varphi(T)$.
Aus dem vorherigen Aufgabenteil wissen wir, wie $\varphi$ aussehen muss.
Wir definieren deshalb die Abbildung
\[
  \varphi \colon R[T] \to S,
  \quad
  \sum_i a_i T^i \mapsto \sum_i \psi(a_i) s^i.
\]
Da $\varphi|_R = \psi$ und $\varphi(T) = s$ gilt es nur noch zeigen, dass $\varphi$ ein Ringhomomorphismus ist.

Für alle $f, g \in R[T]$ mit $f = \sum_i a_i T^i$ und $g = \sum_i b_i T^i$ gelten
\begin{align*}
      \varphi(f + g)
  &=  \varphi\left( \sum_i a_i T^i + \sum_i b_i T^i \right)
   =  \varphi\left( \sum_i (a_i + b_i) T^i \right)
   =  \sum_i (a_i + b_i) s^i
  \\
  &=  \sum_i a_i s^i + \sum_i b_i s^i
   =  \varphi\left( \sum_i a_i T^i \right) + \varphi\left( \sum_i b_i TWi \right)
   =  \varphi(f) + \varphi(g),
\end{align*}
und
\begin{align*}
    \varphi(f \cdot g)
  &= \varphi\left( \left( \sum_i a_i T^i \right) \cdot \left( \sum_i b_i T^i \right) \right)
   = \varphi\left( \sum_i \sum_{j + k = i} a_j b_k T^i \right)
  \\
  &= \sum_i \psi\left( \sum_{j + k = i} a_j b_k \right) s^i
   = \sum_i \sum_{j + k = i} \psi(a_j) \psi(b_k) s^i
  \\
  &= \left( \sum_i \psi(a_i) s^i \right) \cdot \left( \sum_i \psi(b_i) s^i \right)
   = \varphi\left( \sum_i a_i T^i \right) \cdot \varphi\left( \sum_i b_i T^i \right)
  \\
  &= \varphi(f) \cdot \varphi(g).
\end{align*}
Also ist $\varphi$ additiv und multiplikativ.
Da
\[
    \varphi(1)
  = \varphi(1 \cdot T^0)
  = \psi(1) s^0
  = 1 \cdot 1
  = 1
\]
ist $\varphi$ auch unitär.
Insgesamt zeigt dies, dass $\varphi$ ein Ringhomomorphismus ist.





\subsection{}

%Die $K$-Vektorraumstruktur auf $V$ entspricht einen Ringhomomorphismus $\phi \colon K \to \End(V)$, $\lambda \mapsto (v \mapsto \lambda \cdot v)$.
In den vorherigen Aufgabenteilen haben wir Bijektionen konstruiert:
\begin{align*}
  &\,
  \{
    \text{$K[T]$-Modulstrukturen $\mu \colon K[T] \times V \to V$}
  \}
  \\
  \longleftrightarrow&\,
  \{
    \text{Ringhomomorphismen $\phi \colon K[T] \to \End(V)$}
  \}
  \\
  \longleftrightarrow&\,
  \left\{
    (\psi, f)
  \,
  \middle|
  \,
    \begin{tabular}{c}
      $\psi \colon K \to \End(V)$ ist ein Ringhomomorphismus, \\
      $f \in \End(V)$ mit $f \psi(\lambda) = \psi(\lambda) f$ für alle $\lambda \in K$
    \end{tabular}
  \right\}
  \\
  \longleftrightarrow&\,
  \left\{
    (\nu, f)
  \,
  \middle|
  \,
    \begin{tabular}{c}
      $K$-Modulstrukturen $\nu \colon K \times V \to V$\\
      $f \in \End(V)$ mit $f(\nu(\lambda, -)) = \nu(\lambda, f(-))$ für alle $\lambda \in K$
    \end{tabular}
  \right\}
\end{align*}

Dass eine $K[T]$-Modulstruktur $\mu \colon K[T] \times V \to V$ die gegebene $K$-Vektorraumstruktur von $V$ erweitert, bedeutet, dass $\mu(\lambda, v) = \lambda \cdot v$ für alle $\lambda \in K$, $v \in V$.
Die obige Bijektionskette schränkt sich dadurch wie folgt ein:
\begin{align*}
  &\,
  \left\{
    \begin{tabular}{c}
      $K[T]$-Modulstrukturen $\mu \colon K[T] \times V \to V$, die
      \\
      die gegebene $K$-Vektorraumstruktur erweitern.
    \end{tabular}
  \right\}
  \\
  \longleftrightarrow&\,
  \left\{
    \begin{tabular}{c}
      Ringhomomorphismen $\phi \colon K[T] \to \End(V)$
      \\
      mit $\phi(\lambda)(v) = \lambda v$ für alle $\lambda \in K$, $v \in V$
    \end{tabular}
  \right\}
  \\
  \longleftrightarrow&\,
  \left\{
    (\psi, f)
  \,
  \middle|
  \,
    \begin{tabular}{c}
      $\psi \colon K \to \End(V)$ ist ein Ringhomomorphismus
      \\
      mit $\psi(\lambda)(v) = \lambda \cdot v$ für alle $\lambda \in K$, $v \in V$,
      \\
      $f \in \End(V)$ mit $f \psi(\lambda) = \psi(\lambda) f$ für alle $\lambda \in K$
    \end{tabular}
  \right\}
  \\
  \longleftrightarrow&\,
  \left\{
    (\nu, f)
  \,
  \middle|
  \,
    \begin{tabular}{c}
      $K$-Modulstrukturen $\nu \colon K \times V \to V$
      \\
      mit $\nu(\lambda,v) = \lambda \cdot v$ für alle $\lambda \in K$, $v \in V$,
      \\
      $f \in \End(V)$ mit $f(\nu(\lambda, -)) = \nu(\lambda, f(-))$ für alle $\lambda \in K$
    \end{tabular}
  \right\}
  \\
  \longleftrightarrow&\,
  \{
    f \in \End(V)
  \mid
    \text{$f(\lambda \cdot v) = \lambda \cdot f(v)$ für alle $\lambda \in K$, $v \in V$}
  \}
  \\
  =&\, 
  \End_K(V).
\end{align*}
Damit haben wir eine Bijektion zwischen denjenigen $K[T]$-Modulstrukturen auf $V$, welche die $K$-Vektorraumstruktur erweitern, und den $K$-linearen Endomorphismen von $V$.

Konkret sieht diese Bijektion so aus, dass für $f \in \End_K(V)$ die entsprechende $K[T]$-Modulstruktur $\mu \colon K[T] \times V \to V$ durch
\[
    \mu\left( \sum_i a_i T^i, v \right)
  = \sum_i a_i f^i(v)
  \qquad
  \text{für alle $\sum_i a_i T^i \in K[T]$, $v \in V$}
\]
gegeben ist.










\end{document}
