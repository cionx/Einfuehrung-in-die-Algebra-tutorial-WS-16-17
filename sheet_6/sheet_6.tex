\documentclass[a4paper,10pt,numbers=noenddot]{scrartcl}

\usepackage{../generalstyle}
\usepackage{specificstyle}

\title{Lösungen zu Zettel~6}
\author{Jendrik Stelzner}
\date{\today}

\begin{document}
\maketitle





\section{}

Es seien $\primes$ ein Repräsentantensystem der Assoziiertheitsklassen der Primelemente von $R$.

\begin{lemma}
  \label{lemma: divisability via prime factors}
  \begin{enumerate}
    \item
      Für alle $x, y \in R$ und $p \in \primes$ gilt $\nu_p(x y) = \nu_p(x) + \nu_p(y)$.
    \item
      Für alle $x, y \in R$ gilt genau dann $x \mid y$, wenn $\nu_p(x) \leq \nu_p(y)$ für alle $p \in \primes$.
  \end{enumerate}
\end{lemma}

\begin{proof}
  Es gibt Primfaktorzerlegungen $x  = u_1 \prod_{p \in \primes} p^{\nu_p(x)}$ und $y = u_2 \prod_{p \in \primes} p^{\nu_p(y)}$ mit $u_1, u_2 \in R^\times$.
  \begin{enumerate}
    \item
      Für das Produkt $x y$ ergibt sich eine Primfaktorzerlegung
      \[
          x y
        = u_1 \prod_{p \in \primes} p^{\nu_p(x)} \cdot u_2 \prod_{p \in \primes} p^{\nu_p(y)}
        = (u_1 u_2) \prod_{p \in \primes} p^{\nu_p(x) + \nu_p(y)},
      \]
      weshalb $\nu_p(x y) = \nu_p(x) + \nu_p(y)$ für alle $p \in \primes$.
    \item
      Gilt $x \mid y$, so gibt es $z \in R$ mit $y = x z$, weshalb
      \[
              \nu_p(y)
        =     \nu_p(x z)
        =     \nu_p(x) + \nu_p(z)
        \geq  \nu_p(x).
      \]
      Gilt andererseits $\nu_p(x) \leq \nu_p(y)$ für alle $p \in \primes$, so ist $z \coloneqq u_1^{-1} u_2 \prod_{p \in \primes} p^{\nu_p(y) - \nu_p(x)} \in R$ wohldefiniert, und wegen $xy = z$ gilt $x \mid y$.
    \qedhere
  \end{enumerate}
\end{proof}

Für das Element $z \coloneqq \prod_{p \in \primes} p^{\min(\nu_p(x), \nu_p(y))} \in R$ gilt $\nu_p(z) = \min( \nu_p(x), \nu_p(y) )$ für alle $p \in \primes$.
Für jedes $z' \in R$ gilt nach Lemma~\ref{lemma: divisability via prime factors}, dass
\begin{align*}
          z' \mid x,y
  \iff&\, \text{$\nu_p(z') \leq \nu_p(x), \nu_p(y)$ für alle $p \in \primes$}
  \\
  \iff&\, \text{$\nu_p(z') \leq \min(\nu_p(x), \nu_p(y))$ für alle $p \in \primes$}
  \\
  \iff&\, \text{$\nu_p(z') \leq \nu_p(z)$ für alle $p \in \primes$}
  \\
  \iff&\, z' \mid z.
\end{align*}
Somit ist $z$ ein größter gemeinsamer Teiler von $x$ und $y$.




\section{}

Für $k, n \in \Integer$ schreiben wir im Folgenden $[k]_n$ für die Restklasse von $k$ in $\Integer/n\Integer$.
Das Gleichungssystem
\[
  \left\{
    \begin{matrix}
      4x  & \equiv  & \phantom{1}5  & \mod \phantom{1}9,
    \\
      3x  & \equiv  &           10  & \mod           11,
    \end{matrix}
  \right.
\]
für $x \in \Integer$ ist mit dieser Notation äquivalent zu dem Gleichungssystem
\[
  \left\{
    \begin{matrix}
      [4]_{9\phantom{1}}  [x]_{9\phantom{1}}  & = & [5]_9,
    \\
      [3]_{11}            [x]_{11}            & = & [10]_{11}.
    \end{matrix}
  \right.
\]
Da $4$ und $9$ teilerfremd sind, ist $[4]_9 \in \Integer/9\Integer$ eine Einheit (siehe Übungszettel~4), und es gilt $[4]_9^{-1} = [7]_9$.
Für alle $x \in \Integer$ ist deshalb
\[
              [4]_9 [x]_9 = [5]_9
  \iff  [7]_9 [4]_9 [x]_9 = [7]_9 [5]_9
  \iff              [x]_9 = [8]_9.
\]
Analog ergibt sich, dass $[3]_{11} \in \Integer/11\Integer$ eine Einheit ist, und mit $[3]_{11}^{-1} = [4]_{11}$ ergibt sich für alle $x \in \Integer$, dass
\[
                 [3]_{11} [x]_{11}  = [10]_{11}
  \iff  [4]_{11} [3]_{11} [x]_{11}  = [4]_{11} [10]_{11}
  \iff                    [x]_{11}  = [7]_{11}.
\]
Es gilt also die Lösungen $x \in \Integer$ des Gleichungssystems
\[
  \left\{
    \begin{matrix}
      [x]_{9\phantom{1}}  & = & [8]_{9\phantom{1}},
    \\
      [x]_{11}            & = & [7]_{11},
    \end{matrix}
  \right.
\]
zu finden, also die Urbilder von $([8]_9, [7]_{11}) \in (\Integer/9\Integer) \times (\Integer/11\Integer)$ bezüglich des Ringhomomorphismus
\[
  \Integer \to (\Integer/9\Integer) \times (\Integer/11\Integer),
  \quad
  x \mapsto ([x]_9, [x]_{11}).
\]

Da $9$ und $11$ teilerfremd sind, gibt es nach dem chinesischen Restklassensatz einen Ringisomorphismus
\[
          \varphi
  \colon  \Integer/99\Integer \to     (\Integer/9\Integer) \times (\Integer/11\Integer),
  \quad   [x]_{99}            \mapsto ([x]_9, [x]_{11}),
\]
und da $5 \cdot 9 + (-4) \cdot 11 = 1$ ist $\varphi^{-1}$ durch
\begin{align*}
          \varphi^{-1}
  \colon  (\Integer/9\Integer) \times (\Integer/11\Integer) &\to      \Integer/99\Integer,
  \\
          ([x]_9, [y]_{11})                                 &\mapsto  [5 \cdot 9 \cdot y + (-4) \cdot 11 \cdot x]_{99}
                                                             =        [55 x + 45 y]_{99}
\end{align*}
gegeben.
Der Isomorphismus $\varphi$ bringt das folgende Diagram zum kommutieren:
\begin{equation}
  \label{eqn: commutative diagram}
  \begin{tikzcd}
      {}
    & \Integer
      \arrow[swap]{dl}{\pi_{99}}
      \arrow{dr}{(\pi_9, \pi_{11})}
    & {}
    \\
      \Integer/99\Integer
      \arrow{rr}{\varphi}
    & {}
   & (\Integer/9\Integer) \times (\Integer/11\Integer)
  \end{tikzcd}
\end{equation}
Dabei bezeichnet $\pi_n \colon \Integer \to \Integer/n\Integer$, $x \mapsto [x]_n$ für alle $n \in \Integer$ die kanonische Projektion.
Zusammen mit $\varphi^{-1}([8]_9, [7]_{11}) = [755]_{99} = [62]_{99}$ erhalten wir, dass
\[
    (\pi_{9}, \pi_{11})^{-1}([8]_{9}, [7]_{11})
  = \pi_{99}^{-1}\left( \varphi^{-1}([8]_{9}, [7]_{11}) \right)
  = \pi_{99}^{-1}( [62]_{99} )
  = 62 + 99 \Integer.
\]
Dies ist die gesuchte Lösungsmenge des gegebenen Gleichungssystems.

\begin{remark}
  \begin{enumerate}
    \item
      Aus der Kommutativität der Diagrams~\eqref{eqn: commutative diagram} ergibt bereits, dass die gesuchte Lösungsmenge von der Form $x_0 + 99\Integer$ ist, wobei $x_0$ eine spezielle Lösung ist.
      Um eine solche spezielle Lösung zu finden, haben wir die konkrete Berechnung von $\varphi^{-1}$ genutzt, um $\varphi^{-1}([8]_9, [7]_{11}) = [62]_{99}$ zu bestimmen.
      
      In dem konkreten Beispiel dieser Aufgabe ist es allerdings einfacher, eine spezielle Lösung zu bruteforcen, als $\varphi^{-1}$ zu berechnen:
      Möglichen Lösungen der Gleichung $[x]_9 = [8]_9$ sind nämlich
      \[
        8, 17, 26, 35, 44, 53, 62, 71, 80, 89, 98, \dotsc,
      \]
      und mögliche Lösungen der Gleichung $[x]_{11} = [7]_{11}$ sind
      \[
        7, 18, 29, 40, 51, 62, 73, 84, 95, \dotsc,
      \]
      woraus sich durch direkten Vergleich die gemeinsame Lösung $62$ ergibt.
    \item
      Auf Ähnliche Weise lassen sich auch die Inversen $[4]_9^{-1}$ und $[3]_{11}^{-1}$ schnell durch Bruteforcen bestimmen:
      Mögliche $z \in \Integer$ mit $[z]_9 = [1]_9$ sind
      \[
        1, 10, 19, 28, \dotsc,
      \]
      wobei $28$ ein Vielfaches von $4$ ist, nämlich das $7$-fache.
      Folglich ist $[4]_9^{-1} = [7]_9$.
      Mögliche $z \in \Integer$ mit $[z]_{11} = [1]_{11}$ sind
      \[
        1, 12, \dotsc,
      \]
      wobei $12$ ein Vielfaches von $3$ ist, nämlich das $4$-fache.
      Folglich ist $[3]_{11}^{-1} = [4]_{11}$.
      
      Für kleine $n$ kann dieses Verfahren zum Bestimmen von Inversen in $\Integer/n\Integer$ von Hand einfacher und schneller sein als die Maschinerie des euklidischer Algorithmus.
      (Inbesondere ist weniger Nachdenken und deutlich weniger Verwalten von Zwischenschritten erforderlich).
  \end{enumerate}
\end{remark}





\addtocounter{section}{3}





\section{}

Wir erinnern an das folgende Resultat aus der Vorlesung:

\begin{lemma}
  \label{lem: finitely generated and short exact sequences}
  Es sei $0 \to M \xrightarrow{f} P \xrightarrow{g} N \to 0$ eine kurze exakte Sequenz von $R$-Moduln.
  Sind $M$ und $N$ endlich erzeugt, so ist auch $P$ endlich erzeugt.
\end{lemma}



\subsection*{$M$ und $N$ sind noethersch $\implies$ $P$ ist noethersch}

Wir fixieren einen Untermodul $P' \subseteq P$.
Es seien $M' \coloneqq f^{-1}(P')$ und $N' \coloneqq g(P')$, sowie $f' \coloneqq f|_{M'} \colon M' \to P'$, $m \mapsto f(m)$ und $g' \colon g|_{P'} \colon P' \to N'$, $p \mapsto g(p)$.
Die Sequenz
\begin{equation}
  \label{eqn: restricted short exact sequence}
  0 \to M' \xrightarrow{f'} P' \xrightarrow{g'} N' \to 0
\end{equation}
ist exakt:
Die Injektivität von $f'$ folgt aus der von $f$, denn Restriktionen von Injektionen sind ebenfalls injektiv.
Die Surjektivität von $g'$ ergibt sich aus $\im g' = g'(P') = g(P') = N'$.
Die Exaktheit der Sequenz~\eqref{eqn: restricted short exact sequence} an $P'$ ergibt sich aus
\[
    \ker g'
  = P' \cap \ker g
  = P' \cap \im f
  = f(f^{-1}(P'))
  = f(M')
  = f'(M')
  = \im f'.
\]
Die Untermoduln $M' \subseteq M$ und $N' \subseteq N$ sind endlich erzeugt, da $M$ und $N$ noethersch sind.
Zusammen mit der Exaktheit der Sequenz \eqref{eqn: restricted short exact sequence} ergibt sich nach Lemma~\ref{lem: finitely generated and short exact sequences}, dass auch $P'$ endlich erzeugt ist.





\end{document}
