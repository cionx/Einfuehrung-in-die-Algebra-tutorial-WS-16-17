\documentclass[a4paper, 10pt, numbers=noenddot]{scrartcl}

\usepackage{../generalstyle}
\usepackage{specificstyle}

\title{Lösungen zu \\ Aufgabe 2, Zettel 3}
\author{Jendrik Stelzner}
\date{\today}

\begin{document}





\maketitle





\section*{Notationen}

Im Folgenden werden konvergente Reihen stets in der Form $\sum_{i=0}^\infty a_i s^i$ geschrieben.
Endliche Summen werden in der Form $\sum_i a_i$ oder $\sum_{i=0}^n a_i$ geschrieben; die zweite Notation werden wir nur nutzen, wenn wir konkrete Summationsgrenzen angeben.
Elemente des Potenzreihenrings $\powerseries{R}{T}$ fassen wir als formale Summen auf, und werden diese als $\sum_i a_i T^i$ schreiben.

Wir wählen diese Konvention, um zu unterscheiden, welche Summen unabhängig von den Topologien sind ($\sum_i$ und $\sum_{i=0}^n$), und bei welchen Summen es sich um Reihen handelt, die mithilfe von Topologien definiert sind ($\sum_{i=0}^\infty$).





\section{}
\label{section: properties of a metric}

Für alle $f, g \in \powerseries{R}{T}$ gilt
\[
        d_q(f,g) = 0
  \iff  q^{-\nu(f-g)} = 0
  \iff  \nu(f-g) = \infty
  \iff  f-g = 0
  \iff  f = g.
\]
Für jedes $h \in \powerseries{R}{T}$ und alle $n \geq 0$ gilt genau dann $h_n \neq 0$ wenn $-h_n \neq 0$, weshalb $\nu(h) = \nu(-h)$.
Für alle $f, g \in \powerseries{R}{T}$ gilt deshalb
\[
    d_q(f,g)
  = q^{-\nu(f-g)}
  = q^{-\nu(g-f)}
  = d_q(g,f).
\]
Zum Beweis der Dreiecksungleichung fixieren wir $f, g, h \in \powerseries{R}{T}$.
Es gilt zu zeigen, dass
\begin{equation}
  \label{eqn: triangle inequality}
        q^{-\nu(f-h)}
  \leq  q^{-\nu(f-g)} + q^{-\nu(g-h)}.
\end{equation}
Hierfür zeigen wir, dass bereits die stärkere Ungleichung
\begin{equation}
  \label{eqn: ultrametric}
        q^{-\nu(f-h)}
  \leq  \max\{ q^{-\nu(f-g)}, q^{-\nu(g-h)} \}
\end{equation}
gilt.
(Wir zeigen also, dass in \eqref{eqn: triangle inequality} bereits einer der beiden Summanden ausreicht.
Um welchen es sich dabei handelt hängt allerdings von $f$, $g$ und $h$ ab.)
Da
\begin{gather*}
    \max\{ q^{-\nu(f-g)}, q^{-\nu(g-h)} \}
  = q^{\max\{ -\nu(f-g), -\nu(g-h) \}}
  = q^{-\min\{ \nu(f-g), \nu(g-h) \}}
\shortintertext{gilt}
        q^{-\nu(f-h)} \leq  \max\{ q^{-\nu(f-g)}, q^{-\nu(g-h)} \}
  \iff  \nu(f-h)      \geq  \min\{ \nu(f-g), \nu(g-h) \}.
\end{gather*}
Diese letzte Ungleichung gilt, denn für alle $0 \leq n < \min\{ \nu(f-g), \nu(g-h) \}$ gilt $f_n = g_n$ und $g_n = h_n$, und somit auch $f_n = h_n$.

\begin{remark}
  Eine Metrik $d \colon X \times X \to [0,\infty)$ auf einer Menge $X$, die neben der gewöhnlichen Dreiecksungleichung
  \[
    d(x,z) \leq d(x,y) + d(y,z)
    \qquad
    \text{für alle $x, y, z \in X$}
  \]
  sogar schon die stärkere Ungleichung
  \[
    d(x,z) \leq \max\{ d(x,y), d(y,z) \}
    \qquad
    \text{für alle $x, y, z \in X$}
  \]
  erfüllt, heißt \emph{Ultrametrik}.
  Wir haben gezeigt, dass $d_q$ eine Ultrametrik auf $\powerseries{R}{T}$ ist.
\end{remark}


Wir merken noch an, dass die Metrik $d_q$ translationsinvariant ist, d.h.\ es gilt
\begin{equation}
  \label{eqn: translation invariance}
    d_q(f + h, g + h)
  = d_q(f, g)
  \qquad
  \text{für alle $f, g, h \in \powerseries{R}{T}$}.
\end{equation}





\section{}
\label{section: convergence of power series}

Für $f \in \powerseries{R}{T}$ und eine Folge $(f^{(i)})_i$ von Potenzreihen $f^{(i)} \in \powerseries{R}{T}$ gilt
\begin{align*}
      &\, \text{$f^{(i)} \to f$ für $i \to \infty$ bezüglich $d_q$} \\
  \iff&\, \text{$d_q(f^{(i)}, f) \to 0$ für $i \to \infty$} \\
  \iff&\, \text{$q^{-\nu(f^{(i)} - f)} \to 0$ für $i \to \infty$} \\
  \iff&\, \text{$-\nu(f^{(i)} - f) \to -\infty$ für $i \to \infty$} \\
  \iff&\, \text{$\nu(f^{(i)} - f) \to \infty$ für $i \to \infty$} \\
  \iff&\, \text{für jedes $n \geq 0$ gibt es ein $j \geq 0$ mit $\nu(f^{(i)} - f) \geq n$ für alle $i \geq j$}  \\
  \iff&\, \text{für jedes $n \geq 0$ gibt es ein $j \geq 0$ mit $f^{(i)}_m = f_m$ für alle $i \geq j$, $m < n$}  \\
  \iff&\, \text{für jedes $n \geq 0$ gibt es ein $j \geq 0$ mit $f^{(i)}_n = f_n$ für alle $i \geq j$}.
\end{align*}
Es gilt also $f^{(i)} \to f$ genau dann wenn für jedes $n \geq 0$ gilt, dass $f^{(i)}_n = f_n$ für $i$ groß genug.
(Man beachte, dass es von $n$ abhängt, wann $f^{(i)}_n$ konstant wird.
Insbesondere wird die Folge $f^{(i)}$ selbst nicht notwendigerweise konstant.)
Das zeigt insbesondere, dass die Folge $(f^{(i)})_i$ genau dann konvergiert, wenn für jedes $n \geq 0$ die Folge der Koeffizienten $(f^{(i)}_n)_i$ konstant wird.

Wir haben auch gezeigt, dass sich der Grenzwert $\lim_{i \to \infty} f^{(i)}$ dann koeffizientenweise bestimmen lässt.
%Im Folgenden versehen wir $R$ mit der diskreten Topologie, bzw.\ diskreten Metrik.
%Dann konvergiert eine Folge $(a_i)_i$ von Werten $a_i \in R$ genau dann gegen ein $a \in R$, wenn es ein $j \geq 0$ gibt, so dass $a_i = a$ für alle $i \geq j$.

Eine Reihe $\sum_{i = 0}^\infty f^{(i)}$ konvergiert per Definition genau dann, wenn die Folge $(g^{(j)})_j$ der Partialsummen $g^{(j)} \coloneqq \sum_{i=0}^j f^{(i)}$ konvergiert.
Wie bereits gezeigt ist dies äquivalent dazu, dass für jedes $n \geq 0$ die Koeffizientenfolge $(g^{(j)}_n)_j$ konstant wird.
Dies bedeutet gerade, dass es für jedes $n \geq 0$ ein $j \geq 0$ gibt, so dass $g^{(k_1)}_n = g^{(k_2)}_n$ für alle $k_1 \geq k_2 \geq j$;
da $g^{(k_1)}_n - g^{(k_2)}_n = \sum_{i = k_2 + 1}^{k_1} f^{(i)}_n$ ist dies äquivalent dazu, dass $f^{(i)}_n = 0$ für alle $i > j$.

Da sich der Grenzwert einer konvergenten Folge $(f^{(i)})_i$ von Potenzreihen $f^{(i)} \in \powerseries{R}{T}$ koeffizientenweise bestimmen lässt, ergibt sich, dass sich auch der Grenzwert einer konvergenten Reihe $\sum_{i=0}^\infty f^{(i)}$ von Potenzreihen $f^{(i)} \in \powerseries{R}{T}$ koeffizientenweise berechnen lässt, d.h.\ für alle $n \geq 0$ gilt $(\sum_{i=0}^\infty f^{(i)})_n = \sum_i f^{(i)}_n$.

Wir wollen den Leser an dieser Stelle darauf aufmerksam machen, dass sich Konvergenzverhalten einer Folge, bzw.\ Reihe in $\powerseries{R}{T}$ nicht von dem gewählten Parameter $q > 1$ abhängt.

\begin{remark}
  \label{remark: product topology on ring of power series}
  Versieht man den Ring $R$ mit der diskreten Topologie, bzw.\ der diskreten Metrik, so entspricht der topologische Raum $\powerseries{R}{T}$ zusammen mit den stetigen Projektionen $\pi_i \colon \powerseries{R}{T} \to R$, $f \mapsto f_i$ dem abzählbaren topologischen Produkt $\prod_{i \geq 0} R$.
  
  Insbesondere konvergiert dann eine Folge $(f^{(i)})_i$ von Potenzreihen $f^{(i)} \in \powerseries{R}{T}$ genau dann in $\powerseries{R}{T}$, wenn für jedes $n \geq 0$ die Koeffizientenfolge $(f^{(i)}_n)_i$ in $R$ konvergiert.
  Für den Grenzwert $\lim_{i \to \infty} f^{(i)}$ gilt dann $(\lim_{i \to \infty} f^{(i)})_n = \lim_{i \to \infty} f^{(i)}_n$ für alle $n \geq 0$.
\end{remark}





\section{}

Für alle $n, i \geq 0$ gilt $(T^i)_n = \delta_{in}$;
für fixiertes $n \geq 0$ ist deshalb $(T^i)_n = 0$ für alle $n > i$.
Wie im Aufgabenteil~\ref{section: convergence of power series} gesehen, ist deshalb $T^i \to 0$ for $i \to \infty$ bezüglich $d_q$.

Für $f \in \powerseries{R}{T}$ konvergiert die Reihe $\sum_{i=0}^\infty f_i T^i$ bezüglich $d_q$, denn für alle $n, i \geq 0$ gilt $(f_i T^i)_n = f_i \delta_{in}$, und für $i > n$ verschwindet dieser Term.
Durch koeffizientenweises Berechnen des Grenzwertes ergibt sich, dass
\[
    \left( \sum_{i=0}^\infty f_i T^i \right)_{\!\!\!n}
  = \sum_{i=0}^\infty \left( f_i T^i \right)_n
  = \sum_{i=0}^\infty f_i \delta_{in}
  = f_n
  \qquad
  \text{für alle $n \geq 0$},
\]
und somit $\sum_{i=0}^\infty f_i T^i = f$.





\section{}



\subsection*{Unabhängigkeit der Topologie vom Parameter $q$}

Es seien $q_1, q_2 > 0$.
Es gilt zu zeigen, dass eine Teilmenge $U \subseteq \powerseries{R}{T}$ genau dann offen bezüglich $d_{q_1}$ ist, wenn sie offen bezüglich $d_{q_2}$ ist.
Dies ist äquivalent dazu, dass eine Teilmenge $C \subseteq \powerseries{R}{T}$ genau dann abgeschlossen bezüglich $d_{q_1}$ ist, wenn sie abgeschlossen bezüglich $d_{q_2}$ ist.
Dies ist äquivalent dazu, dass eine Teilmenge $C \subseteq \powerseries{R}{T}$ genau dann folgenabgeschlossen bezüglich $d_{q_1}$ ist, wenn sie folgenabgeschlossen bezüglich $d_{q_2}$ ist.

In Aufgabenteil~\ref{section: convergence of power series} haben wir gesehen, dass das Konvergenzverhalten einer Folge $(f^{(i)})_i$ von Potenzreihen $f^{(i)} \in \powerseries{R}{T}$ bezüglich den Metriken $d_{q_1}$ und $d_{q_2}$ nicht von den Parametern $q_1$ und $q_2$ abhängt, d.h.\ für jedes $f \in \powerseries{R}{T}$ gilt genau dann $f^{(i)} \to f$ bezüglich $d_{q_1}$ wenn $f^{(i)} \to f$ bezüglich $d_{q_2}$ gilt.
Deshalb ist eine Teilmenge $C \subseteq \powerseries{R}{T}$ genau dann folgenabgeschlossen bezüglich $d_{q_1}$, wenn sie folgenabgeschlossen bezüglich $d_{q_2}$ ist.



\subsection*{Stetigkeit der Ringoperationen}

% TODO: Add the definition of a topological ring.

Es gilt zu zeigen, dass für je zwei konvergente Folgen $(f^{(i)})_i$ und $(g^{(i)})_i$ von Potenzreihen $f^{(i)}, g^{(i)} \in \powerseries{R}{T}$ auch die Folgen $(f^{(i)} + g^{(i)})_i$ und $(f^{(i)} \cdot g^{(i)})_i$ konvergieren, und dass
\begin{align*}
      \lim_{i \to \infty} \left( f^{(i)} + g^{(i)} \right)
  &=  \left( \lim_{i \to \infty} f^{(i)} \right) + \left( \lim_{i \to \infty} g^{(i)} \right)
\shortintertext{und}
      \lim_{i \to \infty} \left( f^{(i)} \cdot g^{(i)} \right)
  &=  \left( \lim_{i \to \infty} f^{(i)} \right) \cdot \left( \lim_{i \to \infty} g^{(i)} \right).
\end{align*}
Hierfür fixieren wir zwei solche konvergenten Folgen und setzen $f \coloneqq \lim_{i \to \infty} f^{(i)}$ und $g \coloneqq \lim_{i \to \infty} g^{(i)}$.

Es sei $n \geq 0$.
Aus $f^{(i)} \to f$ und $g^{(i)} \to g$ ergibt sich nach Aufgabenteil~\ref{section: convergence of power series}, dass es ein $j \geq 0$ gibt, so dass $f^{(i)}_n = f_n$ und $g^{(i)}_n = g_n$ für alle $i \geq j$.
Für alle $i \geq j$ ist
\[
    \left( f^{(i)} + g^{(i)} \right)_{\!n}
  = f^{(i)}_n + g^{(i)}_n
  = f_n + g_n
  = (f + g)_n.
\]
Aus der Beliebigkeit von $n \geq 0$ folgt nach Aufgabenteil~\ref{section: convergence of power series}, dass $f^{(i)} + g^{(i)} \to f + g$.

Für das Produkt gehen wir analog vor:
Es sei $n \geq 0$.
Da $f^{(i)} \to f$ und $g^{(i)} \to g$ ergibt sich nach Aufgabenteil~\ref{section: convergence of power series}, dass es ein $j \geq 0$ gibt, so dass $f^{(i)}_m = g^{(i)}_m$ für alle $i \geq j$ und $m = 0, \dotsc, n$.
Für alle $i \geq j$ ist damit
\[
    \left( f^{(i)} \cdot g^{(i)} \right)_{\!n}
  = \sum_{m = 0}^n f^{(i)}_m g^{(i)}_{n-m}
  = \sum_{m = 0}^n f_m g_{n-m}
  = (f \cdot g)_n.
\]
Wegen der Beliebigkeit von $n$ zeigt dies nach Aufgabenteil~\ref{section: convergence of power series}, dass $f^{(i)} \cdot g^{(i)} \to f \cdot g$.

Die Stetigkeit der Inversion ergibt sich ähnlich:
Es sei $(f^{(i)})_i$ eine Folge von Einheiten $f^{(i)} \in \powerseries{R}{T}^\times$  und $f \in \powerseries{R}{T}^\times$ mit $f^{(i)} \to f$ für $i \to \infty$.
Es sei $g \coloneqq f^{-1}$ und für alle $i \geq 0$ sei $g^{(i)} \coloneqq (f^{(i)})^{-1}$.
Es gilt zu zeigen, dass auch $g^{(i)} \to g$ für $i \to \infty$.

Wir fixieren ein $n \geq 0$.
Da $f^{(i)} \to f$ gibt es ein $j \geq 0$ mit $f^{(i)}_m = f_m$ für alle $i \geq j$ und $m = 0, \dotsc, n$.
Per Induktion über $m = 0, \dotsc, n$ zeigen wir, dass $g^{(i)}_m = g_m$ für alle $i \geq j$:
Für alle $i \geq j$ ist $g_0 = f_0^{-1} = (f^{(i)}_0)^{-1} = g^{(i)}_0$.
Gilt $g^{(i)}_m = g_m$ für alle $i \geq j$ und $0 \leq m < n$, so ergibt sich, dass
\[    
    g^{(i)}_{m+1}
  = - g^{(i)}_0 \sum_{l=0}^m g^{(i)}_l f^{(i)}_{m+1-l}
  = - g_0 \sum_{l=0}^m g_l f_{m+1-l}
  = g_{m+1}.
\]
Ingesamt zeigt dies, dass es für jedes $n \geq 0$ ein $j \geq 0$, so dass $g^{(i)}_m = g_m$ für alle $i \geq j$ und $m = 0, \dotsc, n$;
insbesondere ist $g^{(i)}_n = g_n$ für alle $i \geq j$.
Das zeigt, dass $g^{(i)} \to g$ für $i \to \infty$.

\begin{remark}
  Die Menge der Einheiten $\powerseries{R}{T}^\times$ ist als Teilmenge von $\powerseries{R}{T}$ sowohl offen als auch abgeschlossen:
  
  Es sei $(f^{(i)})_i$ eine Folge in $\powerseries{R}{T}^\times$ die gegen ein $f \in \powerseries{R}{T}$ konvergiert.
  Für alle $i \geq 0$ gilt $f^{(i)}_0 \in R^\times$, da $f^{(i)}$ invertierbar ist.
  Es gibt ein $j \geq 0$ mit $f_0 = f^{(i)}_0$ für alle $i \geq j$, also gilt auch $f_0 \in R^\times$.
  Somit gilt auch $f \in \powerseries{R}{T}^\times$.
  Das zeigt, dass $\powerseries{R}{T}^\times$ folgenabgeschlossen in $\powerseries{R}{T}$ ist, und somit abgeschlossen in $\powerseries{R}{T}$.
  
  Analog ergibt sich, dass auch $\powerseries{R}{T} \smallsetminus \powerseries{R}{T}^\times$ abgeschlossen ist, und $\powerseries{R}{T}^\times$ somit offen.
  
  Die Aussage lässt sich auch abstrakter einsehen:
  Versieht man $R$ wie in Bemerkung~\ref{remark: product topology on ring of power series} mit der diskreten Topologie, bzw.\ der diskreten Metrik, so ist die Projektion auf den konstanten Koeffizienten $\pi_0 \colon \powerseries{R}{T} \to R$, $f \mapsto f_0$ stetig.
  Für jede Menge von Koeffizienten $C \subseteq R$ sind dann die Urbilder $\pi_0^{-1}(C)$ und $\pi_0^{-1}(R \smallsetminus C) = \powerseries{R}{T} \smallsetminus \pi_0^{-1}(C)$ offen in $\powerseries{R}{T}$.
  Wählt man $C = R^\times$, so ergibt sich die Aussage.
\end{remark}





\section{}

Es genügt zu zeigen, dass $\powerseries{R}{T}$ bezüglich $d_q$ vollständig ist.
Die Menge der Cauchyfolgen stimmt dann mit der Menge der konvergenten Folgen überein, und diese ist unabhängig von $q$, da die durch $d_q$ induzierte Topologie unabhängig von $q$ ist.

Ist $(f^{(i)})_i$ eine Cauchyfolge in $\powerseries{R}{T}$ bezüglich $d_q$, so gilt inbesondere $d_q(f^{(i+1)}, f^{(i)}) \to 0$.
Wegen der Translationsinvarianz von $d_q$ (siehe \eqref{eqn: translation invariance}) gilt deshalb $d_q(f^{(i+1)} - f^{(i)}, 0) \to 0$, also $f^{(i+1)} - f^{(i)} \to 0$.
Nach Aufgabenteil~\ref{section: convergence of power series} gibt es deshalb für jedes $n \geq 0$ ein $j \geq 0$ mit $f^{(i+1)}_n - f^{(i)}_n = 0$ für alle $i \geq j$, also $f^{(i+1)}_n = f^{(i)}_n$ für alle $i \geq j$;
die Folge $(f^{(i)}_n)_i$ ist also konstant für $i \geq j$.
Nach Aufgabenteil~\ref{section: convergence of power series} zeigt dies, dass die Folge $(f^{(i)})_i$ konvergiert.





\section{}
\label{section: calculation of units}



\subsection*{Beweis der Kontraktionseigenschaft}

\begin{claim}
  \label{claim: rules for the degree}
  Für alle $f, g \in \powerseries{R}{T}$ gilt
  \[
    \nu(f g) \geq \nu(g)
    \quad\text{und}\quad
    \nu(T g) = \nu(g) + 1.
  \]
  Ist $f \in T \cdot \powerseries{R}{T}$, so gilt $\nu(f g) \geq \nu(g) + 1$.
\end{claim}
\begin{proof}
  Für alle $0 \leq n < \nu(g)$ gilt $g_n = 0$ und somit auch
  \[
      (f g)_n
    = \sum_{m=0}^n f_m \underbrace{g_{n-m}}_{=0}
    = 0.
  \]
  Deshalb ist $\nu(f g) \geq \nu(g)$.  
  (Tatsächlich ergibt sich mit der obigen Argumentation, dass $\nu(fg) \geq \nu(f) + \nu(g)$.
  Ist $R$ ein Integritätsbereich, so handelt es sich hierbei um eine Gleichheit.)
  Aus $(T g)_0 = 0$ und $(T g)_n = g_{n-1}$ für alle $n \geq 1$ ergibt sich, dass $\nu(T g) = \nu(g) + 1$.
  Ist $f \in T \cdot \powerseries{R}{T}$ so gibt es ein $f' \in \powerseries{R}{T}$ mit $f = T f'$.
  Deshalb gilt dann
  \[
          \nu(f g)
    =     \nu(T f' g)
    =     \nu(f' g) + 1
    \geq  \nu(g) + 1.
    \qedhere
  \]
\end{proof}

Für alle $f \in T \cdot \powerseries{R}{T}$ und $g \in \powerseries{R}{T}$ bezeichnen wir die gegebene Abbildung mit
\[
  \phi_{f,g} \colon \powerseries{R}{T} \to \powerseries{R}{T},
  \quad
  x \mapsto g - fx.
\]
Mit Behauptung~\ref{claim: rules for the degree} erhalten wir für alle $x_1, x_2 \in \powerseries{R}{T}$, dass
\begin{align*}
        d_q( \phi_{f,g}(x_1), \phi_{f,g}(x_2) )
  &=    d_q(g - f x_1, g - f x_2)
   =    q^{-\nu((g - f x_1) - (g - f x_2))}
  \\
  &=    q^{-\nu(f (x_2 - x_1))}
   =    q^{-(\nu(x_2 - x_1) + 1)}
   =    q^{-1} q^{-\nu(x_2 - x_1)}
  \\
  &=    q^{-1} d_q(x_2, x_1)
   =    q^{-1} d_q(x_1, x_2).
\end{align*}
Da $q > 1$ ist $0 < q^{-1} < 1$.
Damit haben wir gezeigt, dass $\phi_{f,g}$ bezüglich $d_q$ eine Kontraktion ist (mit möglicher Kontraktionskonstante $q^{-1}$).



\subsection*{Bestimmung der Einheiten}

\begin{claim}
  \label{claim: power series with constant term 1 are units}
  Ist $f \in \powerseries{R}{T}$ mit $f_0 = 1$, so gilt $f \in \powerseries{R}{T}^\times$.
\end{claim}
\begin{proof}
  Wir betrachten die abgeänderte Potenzreihe
  \[
              g        
    \coloneqq g - 1
    =         g - f_0
    =         \sum_{i \geq 1} f_i T^i
    \in       T \cdot \powerseries{R}{T}.
  \]
  Dann ist $\phi_{g, 1} \colon \powerseries{R}{T} \to \powerseries{R}{T}$ eine Kontraktion bezüglich $d_q$.
  Da $\powerseries{R}{T}$ bezüglich $d_q$ ein vollständiger metrischer Raum ist, können wir auf $\phi_{g,1}$ den Banachschen Fixpunktsatz anwenden.
  Somit erhalten wir einen (eindeutigen) Fixpunkt $x \in \powerseries{R}{T}$ von $\phi_{g,1}$.
  Es gilt nun $x = \phi_{g,1}(x) = 1 - g x$ und somit $x (1 + g) = 1$.
  Also ist $1 + g = f$ eine Einheit (mit $f^{-1} = x$).
\end{proof}

Ist $f \in \powerseries{R}{T}$ eine Einheit, so gibt es ein $g \in \powerseries{R}{T}$ mit $fg = 1$.
Für die konstanten Terme gilt $1 = (fg)_0 = f_0 g_0$ und somit $f_0 \in R^\times$ (mit $f_0^{-1} = g_0$).

Ist andererseits $f \in \powerseries{R}{T}$ mit $f_0 \in R^\times$, so lässt sich $f$ als $f = f_0 (f_0^{-1} f)$ schreiben.
Da $(f_0^{-1} f)_0 = f_0^{-1} f_0 = 1$ ist $f_0^{-1} f$ nach Behauptung~\ref{claim: power series with constant term 1 are units} eine Einheit in $\powerseries{R}{T}$.
Da auch $f_0 \in R^\times \subseteq \powerseries{R}{T}^\times$ ist $f = f_0 \cdot (f_0^{-1} f)$ das Produkt zweier Einheiten, und damit ebenfalls eine Einheit.





\section{}

% TODO: Add definition of denseness.

Für $f \in \powerseries{R}{T}$ gilt für die Folge $(f^{(i)})_i$ von Polynomen $f^{(i)} \coloneqq \sum_{j=0}^i f_j T^j \in R[T]$ nach Aufgabenteil~\ref{section: convergence of power series}, dass $f^{(i)} \to f$ für $i \to \infty$.
Also ist $R[T]$ dicht in $\powerseries{R}{T}$.





\section{}



\subsection*{Motivation}

Ein Ringhomomorphismus $\psi \colon R[T] \to S$ ist eindeutig durch die Einschränkung $\varphi \coloneqq \psi|_R$ und das Bild $s \coloneqq \psi(T)$ bestimmt:
Für ein beliebiges Polynom $\sum_i a_i T^i \in R[T]$ gilt dann, dass
\begin{equation}
  \label{eqn: formula for polynomials}
    \psi\left( \sum_i a_i T^i \right)
  = \sum_i \psi(a_i) s^i;
\end{equation}
da $a_i = 0$ für fast alle $i$ gilt, ist auch $\psi(a_i) = 0$ für fast alle $i$, und die Summe $\sum_i \psi(a_i) s^i$ somit wohldefiniert.
Umgekehrt liefert jedes Paar $(\varphi, s)$ bestehend aus einem Ringhomomorphismus $\varphi \colon R \to S$ und einen Element $s \in S$ durch \eqref{eqn: formula for polynomials} einen Ringhomomorphismus $\psi \colon R[T] \to S$, und die beiden Konstruktionen sind invers zueinander.

Es ist naheliegend dieses Ergebnis auf den Potenzreihenring $\powerseries{R}{T}$ verallgemeinern zu  wollen:
Ein Ringhomomorphismus $\psi \colon \powerseries{R}{T} \to S$ sollte mit $\varphi \coloneqq \psi|_R$ und $s \coloneqq \psi(T)$ durch $\psi(\sum_i a_i T^i) = \sum_i \varphi(a_i) s^i$ eindeutig bestimmt sein.

Die oben genutzte Bedingung, dass $a_i = 0$ für fast alle $i$, gilt nun aber nicht mehr.
Daher ergibt der Ausdruck $\sum_i \varphi(a_i) s^i$ im Allgemeinen keinen Sinn.
Man kann ihm aber Sinn verleihen, indem man fordert, dass $S$ ein topologischer Ring ist:  
Dann kann $\sum_i \varphi(a_i) s^i$ als eine Reihe verstanden werden.

Damit diese Reihe konvergiert muss die Wahl von $s$ allerdings noch passend eingeschränkt werden;
für $s = 1$ und $\sum_i a_i T^i = \sum_i T^i$ (also $a_i = 1$ für alle $i$) ergibt etwa die Summe $\sum_i 1$ im Allgemeinen keinen Sinn.
Außerdem sollte $\psi$ stetig sein, damit $\psi$ auch mit diesen unendlichen Summen verträglich ist.

Bevor wir uns ans Rechnen machen, wollen wir noch abkürzende Begriffe einführen:

\begin{definition}
  Es sei $a \colon \Natural^2 \to R$ eine Matrix.
  \begin{enumerate}
    \item
      Die Matrix $a$ heißt \emph{zeilenendlich}, wenn in jeder Zeile von $a$ fast alle Einträge verschwinden, d.h.\ für jedes $i \in \Natural$ gilt $a_{ij} = 0$ für fast alle $j \in \Natural$.
    \item
      Die Matrix $a$ heißt \emph{spaltenendlich}, wenn in jeder Spalte von $a$ fast alle Einträge verschwinden, d.h.\ für jedes $j \in \Natural$ gilt $a_{ij} = 0$ für fast alle $i \in \Natural$.
    \item
      Es sei $\varphi \colon R \to S$ ein Ringhomomorphismus und $s \in S$.
      Ist die Matrix $a$ zeilen- und spaltenendlich, so heißt sie \emph{$(\varphi,s)$-kompatibel}, bzw.\ \emph{kompatibel zu $(\varphi,s)$}, wenn die beiden Reihen $\sum_{i=0}^\infty( \sum_j \varphi(a_{ij}) s^i )$ und $\sum_{j=0}^\infty( \sum_i \varphi(a_{ij}) s^i )$ konvergieren und für die entsprechenden Grenzwerte $\sum_{i=0}^\infty( \sum_j \varphi(a_{ij}) s^i ) = \sum_{j=0}^\infty( \sum_i \varphi(a_{ij}) s^i )$ gilt.
  \end{enumerate}
\end{definition}

\begin{remark}
  Die Begriffe von Zeilen- und Spaltenendlichkeit einer Matrix sind weit verbreitet.
  Der Begriff von $(\varphi, s)$-Kompatiblitat hingegen wird hier der Nützlichkeit wegen ad-hoc eingefügt.
\end{remark}


\subsection*{Von Ringhomomorphismen $\psi \colon \powerseries{R}{T} \to S$ zu Paaren $(\varphi, s)$}

Es sei $\psi \colon \powerseries{R}{T} \to S$ ein stetiger Ringhomomorphismus.
Es seien $\varphi \coloneqq \psi|_R$ und $s \coloneqq \psi(T)$.
Dann ist $\varphi$ ein Ringhomomorphismus, und es gilt zu zeigen, jede zeilen- und spaltenendliche Matrix $a \colon \Natural^2 \to R$ kompatibel mit $(\varphi, s)$ ist.
Hierfür fixieren wir eine solche Matrix.

Für alle $i, j \in \Natural$ seien $f^{(i)} \coloneqq \sum_j a_{ij} T^i$ und $g^{(j)} \coloneqq \sum_i a_{ij} T^i$;
da $a$ zeilen- und spaltenendlich ist, sind beide Summen endlich.

\begin{claim}
  Die beiden Reihen $\sum_{i=0}^\infty f^{(i)}$ und $\sum_{j=0}^\infty g^{(j)}$ konvergieren, und für die beiden Grenzwerte gilt $\sum_{i=0}^\infty f^{(i)} = \sum_{j=0}^\infty g^{(j)}$.
\end{claim}

\begin{proof}
  Für alle $n \geq 0$ gilt $f^{(i)}_n = (\sum_j a_{ij}) \delta_{ni}$, und deshalb $f^{(i)}_n = 0$ alle $i \neq n$.
  Nach Aufgabenteil~\ref{section: convergence of power series} gilt deshalb $f^{(i)} \to 0$, weshalb die Reihe $\sum_{i=0}^\infty f^{(i)}$ konvergiert.
  Für den Grenzwert $f \coloneqq \sum_{i=0}^\infty f^{(i)}$ ergibt sich durch koeffizientenweises Berechnen, dass
  \[
    f_n
    = \sum_i f^{(i)}_n
    = \sum_i \left( \sum_j a_{ij} \right) \delta_{in}
    = \sum_j a_{nj}
    \qquad
    \text{für alle $n \geq 0$}.
  \]
  
  Für alle $n \geq 0$ gilt $g^{(j)}_n = a_{nj}$ für alle $j \geq 0$.
  Wegen der Spaltenendlichkeit von $a$ gilt für jedes $n \geq 0$, dass $g^{(j)}_n = a_{nj} = 0$ für fast alle $j \geq 0$.
  Nach Aufgabenteil~\ref{section: convergence of power series} ist deshalb $g^{(j)} \to 0$, weshalb die Reihe $\sum_{j=0}^\infty g^{(j)}$ konvergiert.
  Für den Grenzwert $g \coloneqq \sum_{j=0}^\infty g^{(j)}$ ergibt sich durch koeffizientenweises Berechnen, dass
  \[
    g_n = \sum_j g^{(j)}_n = \sum_j a_{nj}
    \qquad
    \text{für alle $n \geq 0$}.
  \]
  Für alle $n \geq 0$ gilt $f_n = g_n$ und somit $f = g$.
\end{proof}

Anwenden von $\psi$ auf die Reihe $\sum_{i=0}^\infty f^{(i)}$ ergibt wegen der Additivität und Stetigkeit von $\psi$, dass auch die Reihe $\sum_{i=0}^\infty \psi(f^{(i)})$ konvergiert.
Dabei gilt
\[
    \psi\left( \sum_{i=0}^\infty f^{(i)} \right)
  = \sum_{i=0}^\infty \psi\left( f^{(i)} \right)
  = \sum_{i=0}^\infty \psi\left( \sum_j a_{ij} T^i \right)
  = \sum_{i=0}^\infty \sum_j \varphi(a_{ij}) s^i.
\]
Analog ergibt sich, dass auch die Reihe $\sum_{j=0}^\infty \psi(g^{(i)})$ konvergiert, und dass
\[
    \psi\left( \sum_{j=0}^\infty g^{(j)} \right)
  = \sum_{j=0}^\infty \sum_i \varphi(a_{ij}) s^i.
\]
Also konvergieren die beiden Reihen $\sum_{i=0}^\infty \sum_j \varphi(a_{ij}) s^i$ und $\sum_{j=0}^\infty \sum_i \varphi(a_{ij}) s^i$, und aus der Gleichheit $\sum_{i=0}^\infty f^{(i)} = \sum_{j=0}^\infty g^{(j)}$ ergibt sich, dass
\[
    \sum_{i=0}^\infty \sum_j \varphi(a_{ij}) s^i
  = \psi\left( \sum_{i=0}^\infty f^{(i)} \right)
  = \psi\left( \sum_{j=0}^\infty g^{(j)} \right)
  = \sum_{j=0}^\infty \sum_i \varphi(a_{ij}) s^i.
\]


\subsection*{Von Paaren $(\varphi, s)$ zu Ringhomomorphismen $\psi \colon \powerseries{R}{T} \to S$}

Es sei nun $(\varphi, s)$ ein Paar, so dass jede zeilen- und spaltenendlich $a \colon \Natural^2 \to R$ kompatibel zu $(\varphi, s)$ ist.
Wir konstruieren in mehreren Schritten einen stetigen Ringhomomorphismus $\psi \colon \powerseries{R}{T} \to S$ mit $\psi|_R = \varphi$ und $\psi(T) = s$.


\subsubsection*{Schritt 0: Die Idee}
Wegen der Stetigkeit von $\psi$ und Dichtheit von $R[T] \subseteq \powerseries{R}{T}$ sollte $\psi$ durch die Einschränkung $\psi|_{R[T]}$ bereits eindeutig bestimmt sein; dass $S$ Hausdorff ist, ist hierfür eine technische Voraussetzung.
Die Einschränkung $\psi|_{R[T]} \colon R[T] \to S$ ist auf rein algebraisch Weise durch die Einschränkung $\psi|_R$ und das Bildelement $\psi(T)$ eindeutig bestimmt.

Für die Konstruktion wollen wir diese Argumenation umgekehren:
Wir erweitern das Paar $(\varphi, s)$ zu einem Ringhomomorphismus $\tilde{\psi} \colon R[T] \to S$, den wir eindeutig zu einer Abbildung $\psi \colon \powerseries{R}{T} \to S$ fortsetzen, indem wir beliebige Potenzreihen $f \in \powerseries{R}{T}$ durch Polynome $f^{(i)} \in R[T]$ approximieren. 
Anschließend überprüfen wir, dass $\psi$ ein stetiger Ringhomomorphismus ist.

Die angegebene technische Bedingung an $s \in S$ wird dabei auf zwei Weisen genutzt:
Zum einen sorgt die Konvergenz der Reihen dafür, dass gewissen Grenwerte existieren, was die genutzten Approximationsargumente erlaubt.
Die Gleichheit der beiden Reihen wird dafür genutzt diese Grenzwerte genauer zu bestimmen.


\subsubsection*{Schritt 1: Konstruktion von $\tilde{\psi} \colon R[T] \to S$}
Nach der universellen Eigenschaft des Polynomrings $R[T]$ entspricht das Paar $(\varphi, s)$ einem eindeutigen Ringhomomorphismus $\tilde{\psi} \colon R[T] \to S$ mit $\tilde{\psi}|_R = \varphi$ und $\tilde{\psi}(T) = s$, und dieser ist gegeben durch
\[
    \tilde{\psi}\left( \sum_i a_i T^i \right)
  = \sum_i \varphi(a_i) s^i
  \qquad
  \text{für alle $\sum_i a_i T^i \in R[T]$}.
\]


\subsubsection*{Schritt 2: Konvergiert $(f^{(i)})_i$ mit $f^{(i)} \in R[T]$ in $\powerseries{R}{T}$ so konvergiert $\tilde{\psi}(f^{(i)})_i$ in $S$}
Ist $(f^{(i)})_i$ eine in $\powerseries{R}{T}$ konvergente Folge von Polynomen $f^{(i)} \in R[T]$ so konvergiert auch die Bildfolge $( \tilde{\psi}(f^{(i)}) )_i$.

Wir definieren die Folge $( g^{(i)} )_i$ von Polynomen $g^{(i)} \in R[T]$ durch $g^{(0)} \coloneqq f^{(0)}$ und $g^{(i)} \coloneqq f^{(i)} - f^{(i-1)}$ für alle $i \geq 1$.
Dann ist $\sum_{j=0}^i g^{(j)} = f^{(i)}$ für alle $i \geq 0$, und somit auch
\[
    \tilde{\psi}\left( f^{(i)} \right)
  = \tilde{\psi}\left( \sum_{j=0}^i g^{(j)} \right)
  = \sum_{j=0}^i \tilde{\psi}\left( g^{(j)} \right)
  \qquad
  \text{für alle $i \geq 0$}.
\]
Es gilt also zu zeigen, dass die Reihe $\sum_{j=0}^\infty \tilde{\psi}(g^{(j)})$ konvergiert.

Da die Folge $f^{(i)}$ konvergiert, wissen wir, dass die Reihe $\sum_{j=0}^\infty g^{(j)}$ konvergiert.
Die Matrix $a \colon \Natural^2 \to R$, $(i,j) \mapsto g^{(j)}_i$ ist zeilen- und spaltenendlich:
Für jedes $j \geq 0$ ist $g^{(j)}$ ein Polynom und somit $g^{(j)}_i = 0$ für fast alle $i \geq 0$.
Da die Reihe $\sum_{j=0}^\infty g^{(j)}$ konvergiert, gilt nach Aufgabenteil~\ref{section: convergence of power series} für jedes $i \geq 0$, dass $g^{(j)}_i = 0$ für fast alle $j \geq 0$.

Da die Matrix $a$ zeilen- und spaltenendlich ist, konvergiert die Reihe $\sum_{j=0}^\infty \sum_i \varphi( g^{(j)}_i ) s^i$ nach Annahme.
Da
\[
    \sum_i \varphi\left( g^{(j)}_i \right) s^i
  = \tilde{\psi}\left( \sum_i g^{(j)}_i T^i \right)
  = \tilde{\psi}\left( g^{(j)} \right)
  \qquad
  \text{für alle $i \geq 0$}
\]
ist dies genau die Konvergenz der Reihe $\sum_{j=0}^\infty \tilde{\psi}(g^{(j)})$.


\subsubsection*{Schritt 3: Konvergiert $R[T] \ni f^{(i)} \to f$ in $\powerseries{R}{T}$ so gilt $\lim_{i \to \infty} \tilde{\psi}(f^{(i)}) = \sum_{i=0}^\infty f_i s^i$}
Ist $(f^{(i)})_i$ eine Folge von Polynomen $f^{(i)} \in R[T]$ mit $f^{(i)} \to f \in \powerseries{R}{T}$, so gilt für den Grenzwert $\lim_{i \to \infty} \tilde{\varphi}(f^{(i)})$, der nach dem vorherigen Schritt existiert, dass
\[
    \lim_{i \to \infty} \tilde{\varphi}\left( f^{(i)} \right)
  = \sum_{i=0}^\infty \varphi(f_i) s^i.
\]

Die Folge $(g^{(j)})_j$ von Polynomen $g^{(j)} \in R[T]$ sei definiert wie zuvor.
Wie oben gesehen gilt $f = \lim_{i \to \infty} f^{(i)} = \sum_{j=0}^\infty g^{(j)}$, und die Matrix $\Natural^2 \to R$, $(i,j) \mapsto g^{(j)}_i$ ist zeilen- und spaltenendlich.
Damit erhalten wir, dass
\begin{align*}
    \lim_{i \to \infty} \tilde{\psi}\left( f^{(i)} \right)
  &= \sum_{j=0}^\infty \tilde{\psi}\left( g^{(j)} \right)
   = \sum_{j=0}^\infty \tilde{\psi}\left( \sum_i g^{(j)}_i T^i \right)
   = \sum_{j=0}^\infty \sum_i \varphi\left( g^{(j)}_i \right) s^i
  \\
  &= \sum_{i=0}^\infty \sum_j \varphi\left( g^{(j)}_i \right) s^i
   = \sum_{i=0}^\infty \varphi\left( \sum_j g^{(j)}_i \right) s^i
   = \sum_{i=0}^\infty \varphi\left( \left( \sum_{j=0}^\infty g^{(j)} \right)_{\!\!\!i}\; \right) s^i
  \\
  &= \sum_{i=0}^\infty \varphi(f_i) s^i.
\end{align*}


\subsubsection*{Schritt 4: Definition von $\psi$}
Wir können nun $\psi$ definieren:
Es sei $f \in \powerseries{R}{T}$.
Wegen der Dichtheit von $R[T]$ in $\powerseries{R}{T}$ gibt es eine Folge $(f^{(i)})_i$ von Polynomen $f^{(i)} \in R[T]$ mit $f^{(i)} \to f$.
Nach den vorherigen Schritten konvergiert die Bildfolge $(\tilde{\psi}(f^{(i)}))_i$, und der Grenzwert $\lim_{i \to \infty} \tilde{\psi}(f^{(i)}) = \sum_{i=0}^\infty f_i s^i$ hängt nur von $f$ ab.
Daher ist
\[
            \psi(f)
  \coloneqq \lim_{i \to \infty} \tilde{\psi}\left( f^{(i)} \right)
  =         \sum_{i=0}^\infty f_i s^i
\]
wohldefiniert.
Wir merken auch noch an, dass $\psi|_{R[T]} = \tilde{\psi}$.


\subsubsection*{Schritt 5: Stetigkeit von $\psi$}
Es lässt sich zeigen, dass $\psi$ stetig ist;
ein Beweis hierfür wird noch hinzugefügt werden.

% TODO: Prove continuity of psi.

\begin{remark}
  Beim Beweis der Stetigkeit von $\psi$ erweist sich unter den gegegebenen Bedingungen als problematisch:
  
  Nimmt man zusätzlich an, dass $S$ metrisierbar ist, so lässt sich ein relativ kurzes Diagonalfolgenargument nutzen.
  Auch eine Argumentation über gleichmäßige Stetigkeit ist dann möglich.
  Inbesondere können in beiden Fällen Sätze aus der Analysis zitiert werden.
  
  Dass $S$ jedoch a priori nur ein weiter hausdorffscher topologischer Ring ist, macht das ganze deutlich schwieriger.
  Der einzige Beweis, den der Autor bisher gefunden hat, nutzt ein für diese Vorlesung übermäßiges Maß an point set topology, und nutzt (und beweist) ad-hoc einige Eigenschaften topologischer Gruppen.
\end{remark}

Wir wollen aber zumindest anmerken, dass $\psi$ die eindeutige stetige Fortsetzung von $\tilde{\psi}$ auf $\powerseries{R}{T}$ ist:
Ist $\psi' \colon \powerseries{R}{T} \to S$ eine stetige Fortsetzung von $\tilde{\psi}$, so gibt es für jedes $f \in \powerseries{R}{T}$ eine Folge $(f^{(i)})_i$ von Polynomen $f^{(i)} \in R[T]$ mit $f^{(i)} \to f$, weshalb
\[
    \psi'(f)
  = \psi'\left( \lim_{i \to \infty} f^{(i)} \right)
  = \lim_{i \to \infty} \psi'\left( f^{(i)} \right)
  = \lim_{i \to \infty} \tilde{\psi}\left( f^{(i)} \right)
  = \psi(f).
\]


\subsubsection*{Schritt 6: $\psi$ ist ein Ringhomomorphismus}
Es seien $f, g \in \powerseries{R}{T}$ und $(f^{(i)})_i, (g^{(i)})_i$ zwei Folgen von Polynomen $f^{(i)}, g^{(i)} \in R[T]$ mit $f^{(i)} \to f$ und $g^{(i)} \to g$.
Aus der Stetigkeit der Addition von $\powerseries{R}{T}$ folgt, dass auch $f^{(i)} + g^{(i)} \to f + g$.
Aus der Stetigkeit der Addition von $S$ und der Stetigkeit von $\psi$ ergibt sich damit, dass
\begin{align*}
        \psi(f) + \psi(g)
  &=    \left( \lim_{i \to \infty} \tilde{\psi}\left( f^{(i)} \right) \right)
      + \left( \lim_{i \to \infty} \tilde{\psi}\left( g^{(i)} \right) \right)
   =  \lim_{i \to \infty} \left( \tilde{\psi}\left( f^{(i)} \right) + \tilde{\psi}\left( g^{(i)} \right) \right)
  \\
  &=  \lim_{i \to \infty} \tilde{\psi}\left( f^{(i)} + g^{(i)} \right)
   =  \tilde{\psi}\left( \lim_{i \to \infty} \left( f^{(i)} + g^{(i)} \right) \right)
   =  \psi(f + g).
\end{align*}
Also ist $\psi$ additiv.
Analog ergibt sich, dass $\psi$ multiplikativ ist.
Außerdem gilt
\[
    \psi(1)
  = \psi\left( T^0 \right)
  = \psi\left( \sum_i \delta_{i,0} T^i \right)
  = \sum_i \varphi(\delta_{i,0}) s^i
  = \sum_i \delta_{i,0} s^i
  = s^0
  = 1.
\]


\subsubsection*{Schritt 7: Die beiden Konstruktionen sind invers zueinender}
Es sei $\psi \colon \powerseries{R}{T} \to S$ ein stetiger Ringhomomorphismus.
Es seien $\varphi \coloneqq \psi|_R$ und $s \coloneqq \psi(T)$.
Dann ist jede zeilen- und spaltenendliche Matrix $a \colon \Natural^2 \to R$ kompatibel mit $(\varphi, s)$.
Es sei $\tilde{\theta} \colon R[T] \to S$ der eindeutige Ringhomomorphismus mit $\tilde{\theta}|_R = \varphi$ und $\tilde{\theta}(T) = s$, und es sei $\theta \colon \powerseries{R}{T} \to S$ die eindeutige stetige Fortsetzung von $\tilde{\theta}$.
Die Einschränkung $\tilde{\psi} \coloneqq \psi|_{R[T]} \colon R[T] \to S$ ist ein Ringhomomorphismus, für den $\tilde{\psi}|_R = \psi|_{R[T]}|_R = \psi|_R = \varphi$ und $\tilde{\psi}(T) = \psi(T) = s$ gilt.
Deshalb gilt $\tilde{\psi} = \tilde{\theta}$ nach der Eindeutigkeit von $\tilde{\theta}$.
Damit ist $\psi$ eine stetige Fortsetzung von $\tilde{\psi} = \tilde{\theta}$ auf $\powerseries{R}{T}$; wegen der Eindeutigkeit dieser Fortsetzung ist $\psi = \theta$.

Es sei andererseits $(\varphi, s)$ ein Paar bestehend aus einem Ringhomomorphismus $\varphi \colon R \to S$ und einem Element $s \in S$, so dass jede zeilen- und spaltenendliche Matrix $\Natural^2 \to R$ kompatibel mit $(\varphi, s)$ ist.
Es sei $\tilde{\psi} \colon R[T] \to S$ der eindeutige Ringhomomorphismus mit $\tilde{\psi}|_R = \varphi$ und $\tilde{\varphi}(T) = s$, und es sei $\psi \colon \powerseries{R}{T} \to S$ die eindeutige stetige Fortsetzung von $\tilde{\psi}$.
Für das zu $\psi$ gehörige Paar $(\varphi', s')$ mit $\varphi' = \psi|_R$ und $s' = \psi(T)$ gilt $\varphi' = \psi|_R = \tilde{\psi}|_R = \varphi$ und $s' = \psi(T) = \tilde{\psi}(T) = s$, also $(\varphi', s') = (\varphi, s)$.





\section{}

Es sei $\iota \colon R \to \powerseries{R}{T}$, $r \mapsto r = r T^0$ die kanonische Inklusion und $f \in T \cdot \powerseries{R}{T}$.
Es gilt zu zeigen, dass das Paar $(\iota, f)$ im Sinne des vorherigen Aufgabenteiles einen wohldefinierten stetigen Ringendomorphismus
\[
  \psi \colon \powerseries{R}{T} \to \powerseries{R}{T},
  \quad
  \sum_i a_i T^i \mapsto \sum_{i=0}^\infty a_i f^i
\]
induziert.
(Auf dem Übungszettel wird die Inklusion $\iota$ unglücklicherweise mit $\id_R$ notiert.)
Hierfür müssen wir überprüfen, dass jede zeilen- und spaltenendliche Matrix $a \colon \Natural^2 \to R$ mit $(\iota, f)$ kompatibel ist.

Hierfür bemerken wir, dass aus $f \in T \cdot \powerseries{R}{T}$ folgt, dass $f^i \in T^i \cdot \powerseries{R}{T}$ für alle $i \geq 0$, denn wegen $f \in T \cdot \powerseries{R}{T}$ ist $f = T g$ für ein $g \in \powerseries{R}{T}$, und somit $f^i = T^i g^i$ für alle $i \geq 0$.

Für jedes $i \geq 0$ gilt $f^i \in T^i \cdot \powerseries{R}{T}$ und somit auch $\sum_j a_{ij} f^i \in T^i \cdot \powerseries{R}{T}$.
Deshalb gilt $(\sum_j a_{ij} f^i)_n = 0$ für alle $i > n$, also $\sum_j a_{ij} f^i \to 0$ für $i \to \infty$.
Nach Aufgabenteil~\ref{section: convergence of power series} konvergiert deshalb die Reihe $\sum_{i=0}^\infty (\sum_j a_{ij} f^i)$, und für alle $n \geq 0$ ergibt sich durch koeffizientenweises Berechnen des Grenzwertes, dass
\[
    \left( \sum_{i=0}^\infty \left( \sum_j a_{ij} f^i \right) \right)_{\!\!\!n}
  = \sum_i \left( \sum_j a_{ij} f^i \right)_{\!\!\!n}
  = \sum_i \sum_j a_{ij} (f^i)_n.
\]

Für die Konvergenz der Reihe $\sum_{j=0}^\infty (\sum_i a_{ij} f^i)$ gilt es zu zeigen, dass $\sum_i a_{ij} f^i \to 0$ für $j \to \infty$, dass es also für jedes $n \geq 0$ ein $J \geq 0$ mit $(\sum_i a_{ij} f^i)_n = 0$ für alle $j \geq J$ gibt.
Für jedes $j \geq 0$ gilt wie zuvor für alle $n \geq 0$, dass
\[
    \left( \sum_i a_{ij} f^i \right)_{\!\!\!n}
  = \sum_i a_{ij} (f^i)_n
  = \sum_{i=0}^n a_{ij} (f^i)_n.
\]
Wegen der Zeilenendlichkeit von $a$ gibt es für jedes $i = 0, \dotsc, n$ ein $J_i \geq 0$ mit $a_{ij} = 0$ für alle $j \geq J_i$.
Für $J \coloneqq \max_{i=1, \dotsc, n} J_i$ gilt dann $a_{ij} = 0$ für alle $i = 1, \dotsc, n$ und $j \geq J$.
Für jedes $j \geq J$ ist also $(\sum_i a_{ij} f^i)_n = \sum_{i=0}^n a_{ij} (f^i)_n = 0$.

Die Reihe $\sum_{j=0}^\infty (\sum_i a_{ij} f^i)$ konvergiert also, und für jedes $n \geq 0$ ergibt sich durch koeffizientenweises Berechnen des Grenzwertes, dass
\[
    \left( \sum_{j=0}^\infty \left( \sum_i a_{ij} f^i \right) \right)_{\!\!\!n} 
  = \sum_j \left( \sum_i a_{ij} f^i \right)_{\!\!\!n}
  = \sum_j \sum_i a_{ij} (f^i)_n.
\]
Wegen der Endlichkeit der Summen $\sum_i \sum_j a_{ij} (f^i)_n$ und $\sum_j \sum_i a_{ij} (f^i)_n$ ergibt sich für die beiden konvergenten Reihen $\sum_{i=0}^\infty \sum_j a_{ij} f^i$ und $\sum_{j=0}^\infty \sum_i a_{ij} f^i$, dass
\[
    \left( \sum_{i=0}^\infty \left( \sum_j a_{ij} f^i \right) \right)_{\!\!\!n}
  = \sum_i \sum_j a_{ij} (f^i)_n
  = \sum_j \sum_i a_{ij} (f^i)_n
  = \left( \sum_{j=0}^\infty \left( \sum_i a_{ij} f^i \right) \right)_{\!\!\!n}
\]
für alle $n \geq 0$, und somit $\sum_{i=0}^\infty \sum_j a_{ij} f^i = \sum_{j=0}^\infty \sum_i a_{ij} f^i$.





\section{}

Wir geben einen alternativen Beweis von Behauptung~\ref{claim: power series with constant term 1 are units}.
Die Charakterisierung der Einheitengruppe $\powerseries{R}{T}^\times$ ergibt sich dann wie im Rest von Aufgabenteil~\ref{section: calculation of units}.

Es sei $f \in \powerseries{R}{T}$ mit $f_0 = 1$.
Die Potenzreihe
\[
            g
  \coloneqq f-1
  =         \sum_{i \geq 1} f_i T^i
  \in       T \cdot \powerseries{R}{T}
\]
induziert nach dem vorherigen Aufgabenteil zu einen stetigen Ringendomorphismus
\[
  \psi \colon \powerseries{R}{T} \to \powerseries{R}{T},
  \quad
  \sum_i a_i T^i \mapsto \sum_{i=0}^\infty a_i g^i.
\]
Inbesondere gilt $f = 1 + g = \psi(1 + T)$.

Da $(1 + T)(\sum_i (-1)^i T^i) = 1$ gilt, ist $1 + T$ eine Einheit in $\powerseries{R}{T}$.
Deshalb ist auch $\psi(1 + T) = 1 + g = f$ eine Einheit in $\powerseries{R}{T}$ mit
\[
    f^{-1}
  = \psi(1 + T)^{-1}
  = \psi\left( (1 + T)^{-1} \right)
  = \psi\left( \sum_i (-1)^i T^i \right)
  = \sum_{i=0}^\infty (-1)^i g^i.
\]













\end{document}
