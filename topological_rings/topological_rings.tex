\documentclass[a4paper, 10pt, numbers=noenddot]{scrartcl}

\usepackage{../generalstyle}
\usepackage{specificstyle}

\title{Lösungen zu \\ Aufgabe 3, Zettel 2}
\author{Jendrik Stelzner}
\date{\today}

\begin{document}
\maketitle





\section{}
\label{section: properties of a metric}

Für alle $f, g \in \powerseries{R}{T}$ gilt
\[
        d_q(f,g) = 0
  \iff  q^{-\nu(f-g)} = 0
  \iff  \nu(f-g) = \infty
  \iff  f-g = 0
  \iff  f = g.
\]
Für jedes $h \in \powerseries{R}{T}$ und $i \geq 0$ ist genau dann $h_i \neq 0$ wenn $-h_i \neq 0$, weshalb $\nu(h) = \nu(-h)$.
Für alle $f, g \in \powerseries{R}{T}$ gilt deshalb
\[
    d_q(f,g)
  = q^{-\nu(f-g)}
  = q^{-\nu(g-f)}
  = d_q(g,f).
\]
Zum Beweis der Dreiecksungleichung fixieren wir $f, g, h \in \powerseries{R}{T}$.
Es gilt zu zeigen, dass
\begin{equation}
  \label{eqn: triangle inequality}
        q^{-\nu(f-h)}
  \leq  q^{-\nu(f-g)} + q^{-\nu(g-h)}.
\end{equation}
Hierfür zeigen wir, dass bereits
\[
        q^{-\nu(f-h)}
  \leq  \max\{ q^{-\nu(f-g)}, q^{-\nu(g-h)} \}.
\]
(Wir zeigen also, dass in \eqref{eqn: triangle inequality} bereits einer der beiden Summanden ausreicht.
Um welchen es sich dabei handelt hängt allerdings von $f$, $g$ und $h$ ab.)
Da
\begin{gather*}
    \max\{ q^{-\nu(f-g)}, q^{-\nu(g-h)} \}
  = q^{\max\{ -\nu(f-g), -\nu(g-h) \}}
  = q^{-\min\{ \nu(f-g), \nu(g-h) \}}
\shortintertext{gilt}
        q^{-\nu(f-h)} \leq  \max\{ q^{-\nu(f-g)}, q^{-\nu(g-h)} \}
  \iff  \nu(f-h)      \geq  \min\{ \nu(f-g), \nu(g-h) \}.
\end{gather*}
Diese letzte Ungleichung gilt, denn für alle $0 \leq i < \min\{ \nu(f-g), \nu(g-h) \}$ gilt $f_i = g_i$ und $g_i = h_i$, und somit auch $f_i = h_i$.

Wir merken noch an, dass die Metrik $d_q$ translationsinvariant ist, d.h.\ es gilt
\[
    d_q(f + h, g + h)
  = d_q(f, g)
  \qquad
  \text{für alle $f, g, h \in \powerseries{R}{T}$}.
\]





\section{}
\label{section: convergence of power series}

Für $f \in \powerseries{R}{T}$ und eine Folge $(f^{(i)})_i$ von Elementen $f^{(i)} \in \powerseries{R}{T}$ gilt
\begin{align*}
      &\, \text{$f^{(i)} \to f$ für $i \to \infty$ bezüglich $d_q$} \\
  \iff&\, \text{$d_q(f^{(i)}, f) \to 0$ für $i \to \infty$} \\
  \iff&\, \text{$q^{-\nu(f^{(i)} - f)} \to 0$ für $i \to \infty$} \\
  \iff&\, \text{$-\nu(f^{(i)} - f) \to -\infty$ für $i \to \infty$} \\
  \iff&\, \text{$\nu(f^{(i)} - f) \to \infty$ für $i \to \infty$} \\
  \iff&\, \text{für jedes $n \geq 0$ gibt es ein $j \geq 0$ mit $\nu(f^{(i)} - f) \geq n$ für alle $i \geq j$}  \\
  \iff&\, \text{für jedes $n \geq 0$ gibt es ein $j \geq 0$ mit $f^{(i)}_m = f_m$ für alle $m \leq n$, $i \geq j$}  \\
  \iff&\, \text{für jedes $n \geq 0$ gibt es ein $j \geq 0$ mit $f^{(i)}_n = f_n$ für alle $i \geq j$}.
\end{align*}
Es gilt also $f^{(i)} \to f$ genau dann wenn für jedes $n \geq 0$ gilt, dass $f^{(i)}_n = f_n$ für $i$ groß genug.
(Man beachte, dass es von $n$ abhängt, wann $f^{(i)}_n$ konstant wird.
Insbesondere wird die Folge $f^{(i)}$ selbst nicht notwendigerweise konstant.)
Das zeigt insbesondere, dass die Folge $(f^{(i)})_i$ genau dann konvergiert, wenn für jedes $n \geq 0$ die Folge der Koeffizienten $(f^{(i)}_n)_i$ konstant wird.
Wir haben aber auch gezeigt, dass sich der Grenzwert $\lim_{i \to \infty} f^{(i)}$ dann koeffizientenweise bestimmen lässt.

Eine Reihe $\sum_{i = 0}^\infty f^{(i)}$ konvergiert per Definition genau dann, wenn die Folge $(g^{(j)})_j$ der Partialsummen $g^{(j)} \coloneqq \sum_{i=0}^\infty f^{(i)}$ konvergiert.
Wie bereits gezeigt ist dies äquivalent dazu, dass für jedes $n \geq 0$ die Koeffizientenfolge $(g^{(j)}_n)_j$ konstant wird.
Dies bedeutet gerade, dass es für jedes $n \geq 0$ ein $k \geq 0$ gibt, so dass $g^{(j_1)}_n = g^{(j_2)}_n$ für alle $j_1 \geq j_2 \geq k$;
da $g^{(j_1)}_n - g^{(j_2)}_n = \sum_{i = j_2 + 1}^{j_1} f^{(i)}_n$ ist gilt dies genau dann, wenn $f^{(i)}_n = 0$ für alle $i > k$.

Außerdem zeigt die obige Argumentation, dass sich der Grenzwert der Reihe $\sum_{i=0}^\infty f^{(i)}$ dann koeffizientenweise berechnen lässt, d.h.\ für alle $n \geq 0$ gilt $(\sum_{i=0}^\infty f^{(i)})_n = \sum_{i=0}^\infty f^{(i)}_n$.

Wir wollen den Leser an dieser Stelle darauf aufmerksam machen, dass sich Konvergenzverhalten einer Folge, bzw.\ Reihe in $\powerseries{R}{T}$ nicht von dem gewählten Parameter $q > 1$ abhängt.

\begin{remark}
  Versieht man den Ring $R$ mit der diskreten Topologie, bzw.\ diskreten Metrik, so entspricht der topologische Raum $\powerseries{R}{T}$ zusammen mit den stetigen Projektionen $\pi_i \colon \powerseries{R}{T} \to R$, $f \mapsto f_i$ dem abzählbaren topologischen Produkt $\prod_{i \geq 0} R$.
\end{remark}





\section{}

Für alle $n, i \geq 0$ gilt $(T^i)_n = \delta_{in}$;
für fixiertes $n \geq 0$ ist deshalb $(T^i)_n = 0$ für alle $n > i$.
Wie im Aufgabenteil~\ref{section: convergence of power series} gesehen, ist deshalb $T^i \to 0$ for $i \to \infty$ bezüglich $d_q$.

Für $f \in \powerseries{R}{T}$ konvergiert die Reihe $\sum_{i=0}^\infty f_i T^i$, denn für jedes $n \geq 0$ gilt $(f_i T^i)_n = f_i \delta_{in}$, und für $i > n$ verschwindet dieser Term.
Durch koeffizientenweises Berechnen des Grenzwertes ergibt sich, dass
\[
    \left( \sum_{i=0}^\infty f_i T^i \right)_n
  = \sum_{i=0}^\infty (f_i T^i)_n
  = \sum_{i=0}^\infty f_i \delta_{in}
  = f_n
  \qquad
  \text{für alle $n \geq 0$},
\]
und somit $\sum_{i=0}^\infty f_i T^i = f$.





\section{}



\subsection*{Unabhängigkeit der Topologie vom Parameter $q$}


Es seien $q_1, q_2 > 0$.
Es gilt zu zeigen, dass eine Teilmenge $U \subseteq \powerseries{R}{T}$ genau dann offen bezüglich $d_{q_1}$ ist, wenn sie offen bezüglich $d_{q_2}$ ist.
Dies ist äquivalent dazu, dass eine Teilmenge $C \subseteq \powerseries{R}{T}$ genau dann abgeschlossen bezüglich $d_{q_1}$ ist, wenn sie abgeschlossen bezüglich $d_{q_2}$ ist.
Hierfür genügt es zu zeigen, dass eine Teilmenge $C \subseteq \powerseries{R}{T}$ genau dann folgenabgeschlossen bezüglich $d_{q_1}$ ist, wenn sie folgenabgeschlossen bezüglich $d_{q_2}$ ist.

In Aufgabenteil~\ref{section: convergence of power series} haben wir gesehen, dass das Konvergenzverhalten einer Folge $(f^{(i)})_i$ von Elementen $f^{(i)} \in \powerseries{R}{T}$ bezüglich den Metriken $d_{q_1}$ und $d_{q_2}$ nicht von den Parametern $q_1$ und $q_2$ abhängt, d.h.\ für jedes $f \in \powerseries{R}{T}$ gilt genau dann $f^{(i)} \to f$ bezüglich $d_{q_1}$ wenn $f^{(i)} \to f$ bezüglich $d_{q_2}$.
Deshalb ist $C \subseteq \powerseries{R}{T}$ genau dann folgenabgeschlossen bezüglich $d_{q_1}$, wenn es folgenabgeschlossen bezüglich $d_{q_2}$ ist.

Also ist die von $d_q$ erzeugte Topologie unabhängig von $q$.



\subsection*{Stetigkeit der Ringoperationen}

Es gilt zu zeigen, dass für je zwei konvergente Folgen $(f^{(i)})_i$ und $(g^{(i)})_i$ von Elementen $f^{(i)}, g^{(i)} \in \powerseries{R}{T}$ auch die Folgen $(f^{(i)} + g^{(i)})_i$ und $(f^{(i)} \cdot g^{(i)})_i$ konvergieren, und dass
\begin{align*}
      \lim_{i \to \infty} \left( f^{(i)} + g^{(i)} \right)
  &=  \left( \lim_{i \to \infty} f^{(i)} \right) + \left( \lim_{i \to \infty} g^{(i)} \right)
\shortintertext{und}
      \lim_{i \to \infty} \left( f^{(i)} \cdot g^{(i)} \right)
  &=  \left( \lim_{i \to \infty} f^{(i)} \right) \cdot \left( \lim_{i \to \infty} g^{(i)} \right)
\end{align*}
Hierfür fixieren wir zwei solche konvergenten Folgen und schreiben $f \coloneqq \lim_{i \to \infty} f^{(i)}$ und $g \coloneqq \lim_{i \to \infty} g^{(i)}$.

Es sei $n \geq 0$.
Aus $f^{(i)} \to f$ und $g^{(i)} \to g$ ergibt sich nach Aufgabenteil~\ref{section: convergence of power series}, dass es ein $j \geq 0$ gibt, so dass $f^{(i)}_n = f_n$ und $g^{(i)}_n = g_n$ für alle $i \geq j$.
Es sei $j \geq j_f, j_g$.
Für alle $i \geq j$ ist
\[
    (f^{(i)} + g^{(i)})_n
  = f^{(i)}_n + g^{(i)}_n
  = f_n + g_n
  = (f + g)_n.
\]
Aus der Beliebigkeit von $n \geq 0$ folgt nach Aufgabenteil~\ref{section: convergence of power series}, dass $f^{(i)} + g^{(i)} \to f + g$.

Für das Produkt gehen wir analog vor:
Es sei $n \geq 0$.
Da $f^{(i)} \to f$ und $g^{(i)} \to g$ ergibt sich nach Aufgabenteil~\ref{section: convergence of power series}, dass es ein $j \geq 0$ gibt, so dass $f^{(i)}_k = g^{(i)}_k$ für alle $k = 0, \dotsc, n$ und $i \geq j$.
Für alle $i \geq j$ ist damit auch
\[
    (f^{(i)} \cdot g^{(i)})_n
  = \sum_{k = 0}^n f^{(i)}_k g^{(i)}_{n-k}
  = \sum_{k = 0}^n f_k g_{n-k}
  = (f \cdot g)_n.
\]
Wegen der Beliebigkeit von $n$ zeigt dies nach Aufgabenteil~\ref{section: convergence of power series}, dass $f^{(i)} \cdot g^{(i)} \to f \cdot g$.


% TODO: Inversion hinzufügen




\section{}

Es genügt zu zeigen, dass $\powerseries{R}{T}$ bezüglich $d_q$ vollständig ist.
Die Menge der Cauchyfolgen stimmt dann mit der Menge der konvergenten Folgen überein, und diese ist unabhängig von $q$, da die Topologie unabhängig von $q$ ist.

Ist $(f^{(i)})_i$ eine Cauchyfolge in $\powerseries{R}{T}$ bezüglich $d_q$, so ist inbesondere $d_q(f^{(i+1)}, f^{(i)}) \to 0$ für $i \to \infty$.
Wegen der Translationsinvarianz von $d_q$ ist $d_q(f^{(i+1)} - f^{(i)}, 0) \to 0$, also $f^{(i+1)} - f^{(i)} \to 0$.
Nach Aufgabenteil~\ref{section: convergence of power series} gibt es deshalb für jedes $n \geq 0$ ein $j \geq 0$ mit $f^{(i+1)}_n - f^{(i)}_n = 0$ für alle $i \geq j$, also $f^{(i+1)}_n = f^{(i)}_n$ für alle $i \geq j$, weshalb die Folge $(f^{(i)}_n)_i$ für $i \geq j$ konstant ist.
Nach Aufgabenteil~\ref{section: convergence of power series} konvergiert die Folge $(f^{(i)})_i$ deshalb.













\end{document}
