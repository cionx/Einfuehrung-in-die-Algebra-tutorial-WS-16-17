\documentclass[a4paper,10pt,numbers=noenddot]{scrartcl}

\usepackage{../generalstyle}
\usepackage{specificstyle}

\title{Lösung zu Zettel~10, Aufgabe 4}
\author{Jendrik Stelzner}
\date{\today}

\begin{document}





\maketitle





\section{Allgemeines Vorgehen}





\begin{lemma}
  \label{lemma: quotients via diagonal matrices}
  Es sei $R$ ein Ring.
  \begin{enumerate}
    \item
      Es sei $(M_i)_{i \in I}$ eine Familie von $R$-Moduln, und für jedes $i \in I$ sei $N_i \subseteq M_i$ ein Untermodul.
      Dann ist $\bigoplus_{i \in I} N_i \subseteq \bigoplus_{i \in I} M_i$ ein Untermodul, und die Abbildung
      \[
            \left. \left( \bigoplus_{i \in I} M_i \right) \middle/ \left( \bigoplus_{i \in I} N_i \right) \right.
        \to \bigoplus_{i \in I} (M_i / N_i),
        \qquad
        [(m_i)_{i \in I}] \mapsto ([m_i])_{i \in I}
      \]
      ist ein wohldefinierter Isomorphismus von $R$-Moduln.
    \item
      Es sei $\phi \colon F \to G$ ein Modulhomomorphismus zwischen freien $R$-Moduln endliches Rangs;
      es gebe eine Basis $\basis{B} = (b_1, \dotsc, b_n)$ von $F$ und eine Basis $\basis{C} = (c_1, \dotsc, c_m)$ von $G$, so dass $\phi$ bezüglich dieser Basen durch eine Matrix $A \in \Mat(m \times n, R)$ der Form
      \[
          A
        = \begin{pmatrix}
            a_1     &         &         & 0       & \cdots  & 0       \\
                    & \ddots  &         & \vdots  & \ddots  & \vdots  \\
                    &         & a_r     & 0       & \cdots  & 0       \\
            0       & \cdots  & 0       & 0       & \cdots  & 0       \\
            \vdots  & \ddots  & \vdots  & \vdots  & \ddots  & \vdots  \\
            0       & \cdots  & 0       & 0       & \cdots  & 0
          \end{pmatrix}
      \]
      dargestellt wird.
      Dann gilt $G / \im \phi \cong R/(a_1) \oplus \dotsb \oplus R/(a_r) \oplus R^{m-r}$.
  \end{enumerate}
\end{lemma}

\begin{proof}
  \begin{enumerate}
    \item
      Es ist klar, dass $\bigoplus_{i \in I} N_i \subseteq \bigoplus_{i \in I} M_i$ ein Untermodul ist.
      Die Abbildung
      \[
                \varphi
        \colon  \bigoplus_{i \in I} M_i \to \bigoplus_{i \in I} (M_i / N_i),
        \quad
                (m_i)_{i \in I} \mapsto ([m_i])_{i \in I}
      \]
      ist ein surjektiver Homomorphismus von $R$-Moduln, und induziert daher einen Isomorphismus von $R$-Moduln
      \[
                \overline{\varphi}
        \colon  \left( \bigoplus_{i \in I} M_i \right) / \ker \varphi \to \bigoplus_{i \in I} (M_i / N_i),
        \quad
                [(m_i)_{i \in I}] \mapsto ([m_i])_{i \in I}.
      \]
      Für $(m_i)_{i \in I} \in \bigoplus_{i \in I} M_i$ gilt
      \[
              (m_i)_{i \in I} \in \ker \varphi
        \iff  \text{$[m_i] = 0$ für alle $i \in I$}
        \iff  \text{$m_i \in N_i$ für alle $i \in I$},
      \]
      weshalb $\ker \varphi = \bigoplus_{i \in I} N_i$.
    \item
      Es gelten
      \[
          G
        = R c_1 \oplus \dotsb \oplus R c_m
      \quad\text{und}\quad
          \im \phi
        = (a_1) c_1 \oplus \dotsb \oplus (a_r) c_r
      \]
      und deshalb
      \begin{align*}
                G / \im \phi
        &=      (R c_1 \oplus \dotsb \oplus R c_m) / \left( (a_1) c_1 \oplus \dotsb \oplus (a_r) c_r \right)
        \\
        &\cong  ( \underbrace{R \oplus \dotsb \oplus R}_{m} )
                / ( (a_1) \oplus \dotsb \oplus (a_r) \oplus \underbrace{0 \oplus \dotsb \oplus 0}_{m-r} )
        \\
        &\cong  R/(a_1) \oplus \dotsb \oplus R/(a_r) \oplus \underbrace{R/0 \oplus \dotsb \oplus R/0}_{m-r}
        \\
        &\cong  R/(a_1) \oplus \dotsb \oplus R/(a_r) \oplus R^{m-r}.
      \qedhere
      \end{align*}
  \end{enumerate}
\end{proof}

Es sei nun $R$ ein euklidischer Ring und $\phi \colon R^n \to R^m$ ein Homomorphismus von $R$-Moduln.
Um den Quotienten $R^m / \im \phi$ bis auf Isomorphie zu bestimmen, lässt sich wie folgt vorgehen:
\begin{enumerate}
  \item
    Bezüglich der Standardbasen von $R^n$ und $R^m$ wird der Homomorphismus $\phi$ durch eine Matrix $A \in \Mat(m \times n, R)$ dargestellt.
    (Die Matrix $A$ ist dadurch gegeben, dass $\phi(x) = Ax$ für alle $x \in R^n$ gilt.)
  \item
    Durch elementare Zeilen- und Spaltenumformungen entsteht aus der Matrix $A$ eine Matrix $A' \in \Mat(m \times n, R)$ in Smith-Normalform.
    D.h.\ $A'$ ist von der Form
    \[
        A'
      = \begin{pmatrix}
          a_1     &         &         & 0       & \cdots  & 0       \\
                  & \ddots  &         & \vdots  & \ddots  & \vdots  \\
                  &         & a_r     & 0       & \cdots  & 0       \\
          0       & \cdots  & 0       & 0       & \cdots  & 0       \\
          \vdots  & \ddots  & \vdots  & \vdots  & \ddots  & \vdots  \\
          0       & \cdots  & 0       & 0       & \cdots  & 0
        \end{pmatrix}
    \]
    mit $a_i \mid a_{i+1}$ für alle $i = 1, \dotsc, r-1$.
  \item
    Es gibt eine Basis $\basis{B}$ von $R^n$ und eine Basis $\basis{C}$ von $R^m$, so dass $\phi$ bezüglich dieser Basen durch die Matrix $A'$ dargestellt wird.
    Nach Lemma~\ref{lemma: quotients via diagonal matrices} ist deshalb
    \[
            R^m / \im \phi
      \cong R/(a_1) \oplus \dotsb \oplus R/(a_r) \oplus R^{m-r}.
    \]
\end{enumerate}


\begin{remark}
  \begin{enumerate}
    \item
      Die Bedingung $a_i \mid a_{i+1}$ ist für die Berechnung des Quotienten nicht notwendig.
      Die Matrix $A'$ muss also nicht in Smith-Normalform sein, sondern nur in passender Diagonalgestalt.
    \item
      Die obige Berechnungsmethode zeigt, dass für eine Matrix $A \in \Mat(m \times n, R)$ der Quotient $R^m / A R^n$ bis auf Isomorphie nur von der Smith-Normalform von $A$ abhängt.
  \end{enumerate}
\end{remark}


Wir betrachten nun die folgenden Matrizen mit ganzzahligen Einträgen:
\[
    A
  = \begin{pmatrix}
      0 & 1 & 2 \\
      3 & 4 & 5 \\
      6 & 7 & 8
    \end{pmatrix},
  \quad
    B
  = \begin{pmatrix}
      1 & 2 & 3 \\
      4 & 5 & 6 \\
      7 & 8 & 9
    \end{pmatrix},
  \quad
    C
  = \begin{pmatrix}
      2 & 3 &  4  \\
      5 & 6 &  7  \\
      8 & 9 & 10
    \end{pmatrix},
  \quad
    D
  = \begin{pmatrix}
      -10 & 12  \\
      -16 & 18
    \end{pmatrix}
\]
Für diese Matrizen ergeben sich über $\Integer$ die folgenden Smith-Normalformen (die konkrenten Berechnungen finden sich im nächsten Abschnitt):
\[
    A'
  = B'
  = C'
  = \begin{pmatrix}
      1 &   &   \\
        & 3 &   \\
        &   & 0
    \end{pmatrix},
  \quad
    D'
  = \begin{pmatrix}
      2 &    \\
        & 6
    \end{pmatrix}
\]
Damit ergibt sich, dass
\[
        \Integer^3 / A \Integer^3
  \cong \Integer/(1) \oplus \Integer/(3) \oplus \Integer^1
  \cong \Integer/3 \oplus \Integer
\]
und ebenso $\Integer^3 / B \Integer^3 \cong \Integer/3 \oplus \Integer$ und $\Integer^3 / C \Integer^3 \cong \Integer/3 \oplus \Integer$.
Außerdem ergibt sich, dass
\[
        \Integer^2 / D \Integer^2
  \cong \Integer/2 \oplus \Integer/6
  \cong \Integer/2 \oplus \Integer/2 \oplus \Integer/3.
\]
Dabei nutzen wir den Chinesischen Restklassensatz, um einen Iso\-mor\-phis\-mus von Ringen, und damit insbesondere von abelschen Gruppen, $\Integer/6 \cong \Integer/2 \oplus \Integer/3$ zu erhalten.





\section{Berechnung der Smith-Normalformen}

Im Folgenden bringen wir die Matrizen $A$, $B$, $C$ und $D$ durch elementare Zeilen- und Spaltenumformungen in Smith-Normalform.

\begin{align*}
  A:
  &\,
  \begin{pmatrix}
    0 & 1 & 2 \\
    3 & 4 & 5 \\
    6 & 7 & 8
  \end{pmatrix}
  \to
  \begin{pmatrix*}[r]
    0 & 1 &  0  \\
    3 & 4 & -3  \\
    6 & 7 & -6
  \end{pmatrix*}
  \to
  \begin{pmatrix*}[r]
    0 & 1 &  0  \\
    3 & 0 & -3  \\
    6 & 0 & -6
  \end{pmatrix*}
  \\
  \to&\,
  \begin{pmatrix*}[r]
    1 & 0 &  0  \\
    0 & 3 & -3  \\
    0 & 6 & -6
  \end{pmatrix*}
  \to
  \begin{pmatrix*}[r]
    1 & 0 & 0 \\
    0 & 3 & 0 \\
    0 & 6 & 0
  \end{pmatrix*}
  \to
  \begin{pmatrix*}[r]
    1 & 0 & 0 \\
    0 & 3 & 0 \\
    0 & 0 & 0
  \end{pmatrix*}
\end{align*}

\begin{align*}
  B:
  &\,
  \begin{pmatrix*}[r]
    1 & 2 & 3 \\
    4 & 5 & 6 \\
    7 & 8 & 9
  \end{pmatrix*}
  \to
  \begin{pmatrix*}[r]
    1 &  0 &   0  \\
    4 & -3 &  -6  \\
    7 & -6 & -12
  \end{pmatrix*}
  \to
  \begin{pmatrix*}[r]
    1 &  0 &   0  \\
    0 & -3 &  -6  \\
    0 & -6 & -12
  \end{pmatrix*}
  \\
  &\,
  \to
  \begin{pmatrix*}[r]
    1 &  0 & 0  \\
    0 & -3 & 0  \\
    0 & -6 & 0
  \end{pmatrix*}
  \to
  \begin{pmatrix*}[r]
    1 &  0 & 0  \\
    0 & -3 & 0  \\
    0 &  0 & 0
  \end{pmatrix*}
  \to
  \begin{pmatrix*}[r]
    1 & 0 & 0 \\
    0 & 3 & 0 \\
    0 & 0 & 0
  \end{pmatrix*}
\end{align*}

\begin{align*}
  C:
  &\,
  \begin{pmatrix*}[r]
    2 & 3 &  4  \\
    5 & 6 &  7  \\
    8 & 9 & 10
  \end{pmatrix*}
  \to
  \begin{pmatrix*}[r]
    2 & 1 &  4  \\
    5 & 1 &  7  \\
    8 & 1 & 10
  \end{pmatrix*}
  \to
  \begin{pmatrix*}[r]
    0 & 1 & 0 \\
    3 & 1 & 3 \\
    6 & 1 & 6
  \end{pmatrix*}
  \\
  &\,
  \to
  \begin{pmatrix*}[r]
    0 & 1 & 0 \\
    3 & 0 & 3 \\
    6 & 0 & 6
  \end{pmatrix*}
  \to
  \begin{pmatrix*}[r]
    1 & 0 & 0 \\
    0 & 3 & 3 \\
    0 & 6 & 6
  \end{pmatrix*}
  \to
  \begin{pmatrix*}[r]
    1 & 0 & 0 \\
    0 & 3 & 0 \\
    0 & 6 & 0
  \end{pmatrix*}
  \to
  \begin{pmatrix*}[r]
    1 & 0 & 0 \\
    0 & 3 & 0 \\
    0 & 0 & 0
  \end{pmatrix*}
\end{align*}

\begin{align*}
  D:
  &\
  \begin{pmatrix}
    -10 & 12  \\
    -16 & 18
  \end{pmatrix}
  \to
  \begin{pmatrix}
    -10 & 2 \\
    -16 & 2
  \end{pmatrix}
  \to
  \begin{pmatrix}
    6 & 2 \\
    0 & 2
  \end{pmatrix}
  \to
  \begin{pmatrix}
    6 & 0 \\
    0 & 2
  \end{pmatrix}
  \to
  \begin{pmatrix}
    2 & 0 \\
    0 & 6
  \end{pmatrix}
\end{align*}




\end{document}