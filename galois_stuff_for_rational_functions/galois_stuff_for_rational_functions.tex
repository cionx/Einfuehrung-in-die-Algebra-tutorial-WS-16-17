\documentclass[a4paper,10pt,numbers=noenddot]{scrartcl}

\usepackage{../generalstyle}
\usepackage{specificstyle}

\title{Lösung zu Zettel~11, \\ Aufgaben 3 und 4}
\author{Jendrik Stelzner}
\date{\today}

\begin{document}





\maketitle



\section{Aufgabe 3}

Die $K$-linearen Körperhomomorphismen $K(t) \to K(t)$ sind nach der universellen Eigeschaft des Quotientenkörpers genau die eindeutigen Fortsetzungen der $K$-linearen, injektiven Ringhomomorphismen $K[T] \to K(t)$.

Die $K$-linearen Ringhomomorphismen $K[T] \to K(t)$ sind genau die Einsetzhomomorphismen von Elementen aus $K(t)$, wobei jedes Element einen anderen Einsetzhomomorphismus liefert; die injektiven Einsetzhomomorphismen entsprechen dabei genau den Elementen aus $K(t)$, die transzendent über $K$ sind.
Nach Aufgabe~2 ist ein Element $q \in K(t)$ genau dann transzendent, wenn $q \notin K$.

Ingesamt erhalten wir, dass die $K$-linearen Körperhomomorphismen $K(t) \to K(t)$ genau die Abbildungen der Form $p \mapsto p(q)$ für $q \in K(t)$ mit $q \notin K$ sind.

\begin{claim}
  Für ein Element $q \in K(t)$ gilt genau dann $q \notin K$, wenn $q = (at+b)/(ct+d)$ für ein
  \[
    \begin{pmatrix}
      a & b
      \\
      c & d
    \end{pmatrix}
    \in \GL_2(K).
  \]
\end{claim}
\begin{proof}
  
\end{proof}


Ist $q \in K(t)$ mit $q \notin K$, so ist das Bild des Einsetzhomomorphismus
\[
  \varphi \colon K(t) \to K(t),
  \quad
  p \mapsto p(q)
\]
genau $K(q)$.
Ist $q = f/g$ mit teilerfremden $f, g \in K[T]$, so gilt nach Aufgabe~2, dass
\[
    [K(t) : \im \varphi]
  = [K(t) : K(q)]
  = \max\{ \deg f, \deg g \}.
\]
Folglich ist $\varphi$ genau dann surjektiv, und damit bijektiv (denn Körperhomomorphismen sind immer injektiv), wenn $[K(t) : \im \varphi] = 1$, wenn also $f = at+b$ und $g = ct+d$ mit $a \neq 0$ oder $c \neq 0$.

Für ein Element $(at+b)/(ct+d)$ mit $c \neq 0$ oder $d \neq 0$ gilt
\begin{align*}
      &\, \frac{at+b}{ct+d} \in K
  \\
  \iff&\, \frac{at+b}{ct+d} = \lambda       \quad \text{für ein $\lambda \in K$}
  \\
  \iff&\, at + b = c \lambda t + d \lambda  \quad \text{für ein $\lambda \in K$}
  \\
  \iff&\, \left\{
            \begin{matrix}
              a &=& c \lambda
              \\
              b &=& d \lambda
            \end{matrix}
          \right.                           \quad \text{für ein $\lambda \in K$}
  \\
  \iff&\, \text{$\vect{a \\ b}$ und $\vect{c \\ d}$ sind linear abhängig}
  \\
  \iff&\, \begin{pmatrix}
            a & b
            \\
            c & d
          \end{pmatrix}
          \notin \GL_2(K).
\end{align*}

Insgesamt erhalten wir somit, dass die $K$-linearen Körperautomorphismen $K(t) \to K(t)$ genau die Einsetzhomomorphismen von Elementen $q \in K(t)$ der Form $q = (at+b)/(ct+d)$ mit
\[
  \begin{pmatrix}
    a & b
    \\
    c & d
  \end{pmatrix}
  \in
  \GL_2(K)
\]
sind.
Wir haben also eine surjektive Abbildung
\[
          \Phi
  \colon  \GL_2(K)
  \to     \Gal(K(t) / K),
  \quad
          \begin{pmatrix}
            a & b
            \\
            c & d
          \end{pmatrix}
          \mapsto
          \left(
                    p
            \mapsto p\left( \frac{at+b}{ct+d} \right)
          \right).
\]
Diese Abbildung ist ein Gruppenantihomomorphismus, d.h.\ für alle $S_1, S_2 \in \GL_2(K)$ gilt $\Phi(S_1 S_2) = \Phi(S_2) \Phi(S_1)$.
Sind nämlich
\[
    S_1
  = \begin{pmatrix}
      a & b
      \\
      c & d
    \end{pmatrix}
  \quad
  \text{und}
  \quad
    S_2
  = \begin{pmatrix}
      a'  & b'
      \\
      c'  & d'
    \end{pmatrix},
\]
so gilt
\begin{align*}
      \Phi(S_1)\left( \Phi(S_2)(t) \right)  
  &=  \Phi(S_1)( \Phi(S_2)(t) )
  =   \Phi(S_1)\left( \frac{a' t + b'}{c' t + d'} \right)
  =   \frac{a' \Phi(S_1)(t) + b'}{c' \Phi(S_1 ) + d'}
  \\
  &=  \frac{a' \frac{a t + b}{c t + d} + b'}{c' \frac{a t + b}{c t + d} + d'}
  =   \frac{a'(at+b) + b'(ct+d)}{c'(at+b) + d'(ct+d)}
  =   \frac{(a'a + b'c)t + a'b + b'd}{(c'a + d'c)t  + (c'b + d'd)}
  \\
  &=  \Phi\left(
            \begin{pmatrix}
              a'a + b'c & a'b + b'd
              \\
              c'a + d'c & c'b + d'd
            \end{pmatrix}
          \right)(t)
  =   \Phi(S_2 S_1)(t).
\end{align*}

Es gilt $\ker \varphi = K^\times I$, denn
\begin{align*}
        \begin{pmatrix}
          a & b
          \\
          c & d
        \end{pmatrix}
        \in
        \ker \varphi
  &\iff
        \varphi
        \left(
          \begin{pmatrix}
            a & b
            \\
            c & d
          \end{pmatrix}
        \right)
        =
        \id
  \\
  &\iff
        \varphi
        \left(
          \begin{pmatrix}
            a & b
            \\
            c & d
          \end{pmatrix}
        \right)
        (t)
        =
        t
  \\
  &\iff
        \frac{at+b}{ct+d} = t
  \\
  &\iff
        at + b = ct^2 + dt
  \\
  &\iff
        a = d, b = c = 0.
\end{align*}
Somit induziert $\varphi$ einen Antiisomorphismus von Gruppen
\[
  \PGL_2(K) \to \Gal(K(t)/K),
  \quad
  \overline{
  \begin{pmatrix}
    a & b
    \\
    c & d
  \end{pmatrix}
  }
  \mapsto
  \varphi
  \left(
    \begin{pmatrix}
      a & b
      \\
      c & d
    \end{pmatrix}
  \right).
\]





\section{}

Es sei
\[
            C_n(K)
  \coloneqq \{ \omega \in K \mid \omega^2 = 1 \}
\]
die Gruppe der $n$-ten Einheitswurzeln in $K$;
es handelt sich um eine Untergruppe von $K^\times$.

Für alle $\omega \in C_n(K)$ ist das Element $\omega t \in K(t)$ transzendent, da $\omega t \notin K$.
Somit gibt es für jedes $\omega \in C_n(K)$ einen Körperhomomorphismus
\[
          \sigma_\omega
  \colon  K(t) \to K(t),
  \quad   p(t) \mapsto p(\omega t).
\]
Es gilt $\omega_1 = \id_{K(t)}$, und für alle $\omega_1, \omega_2 \in C_n(K)$ gilt $\sigma_{\omega_1} \circ \sigma_{\omega_2} = \sigma_{\omega_1 \omega_2}$.
Daraus folgt, dass $\{\sigma_\omega \mid \omega \in C_n(K)\}$ ein Untergruppe von $\Aut(K(t))$ ist, und die Abbildung
\[
          \Phi
  \colon  C_n(K) \to \Aut(K(t)),
  \quad   \omega \mapsto \sigma_\omega
\]
ein Gruppenhomomorphismus ist.
Da $\phi(\omega)(t) = \sigma_\omega(t) = \omega t$ ist $\phi$ injektiv.
Wir zeigen, dass $\im \phi = \Gal(K(t) / K(t^n))$;
damit liefert die Zuordnung $\omega \mapsto \sigma_\omega$ einen Isomorphismus $C_n(K) \to \Gal(K(t) / K(t^n))$.

Es gilt $\im \Phi \subseteq \Gal(K(t) / K(t^n))$,
denn für jedes $\omega \in C_n(K)$ gilt
\[
    \sigma_\omega(t^n)
  = \sigma_\omega(t)^n
  = (\omega t)^n
  = \omega^n t^n
  = t^n
\]
und somit $\sigma_\omega|_{K(t^n)} = \id_{K(t^n)}$.

Die Surjektivität von $\Phi$ nach $\Gal(K(t) / K(t^n))$ ergibt sich aus der folgenden Aussage:

\begin{claim}
  Ist $P \in K(t)$ mit $P^n = t^n$, so ist $P = \omega t$ mit $\omega \in C_n(K)$.
\end{claim}
\begin{proof}
  Es sei $\mathcal{P} \subseteq K[T]$ ein Repräsentantensystem der Primelemente von $K[T]$.
  Wir können o.B.d.A.\ davon ausgehen, dass $t \in \mathcal{P}$.
  Dann gibt es eine eindeutige Darstellung
  \[
    P = \omega \prod_{p \in \mathcal{P}} p^{\nu_p}
  \]
  mit $\omega \in K^\times$ und $\nu_p \in \Integer$ für alle $p \in \mathcal{P}$, wobei $\nu_p = 0$ für fast alle $p \in \mathcal{P}$.
  Es gilt
  \[
      t^n
    = P^n
    = \omega^n \prod_{p \in \mathcal{P}} p^{n \nu_p}
    = \omega^n t^{n \nu_t} \prod_{p \in \mathcal{P} \setminus \{t\}} p^{n \nu_p}
  \]
  und aus der Eindeutigkeit der Darstellung von $t^n$ ergibt sich, dass $\omega^n = 1$, $\nu_t = 1$ und $\nu_p = 0$ für alle $ \in \mathcal{P} \setminus \{t\}$.
  Also ist $P = \omega t$ mit $\omega \in C_n(K)$.
\end{proof}

Die Galoisgruppe der Körpererweiterung $K(t)/K(t^n)$ ist also die Gruppe der $n$-ten Einheitswurzeln.
Nach Aufgabe~2 gilt $[K(t) : K(t^n)] = n$ (das Minimalpolynom von $t$ über $K(t^n)$ ist $X - t^n \in K(t^n)[X]$),
also st $K(t)/K(t^n)$ genau dann galoisch, wenn $|C_n(K)| = n$.
Da $C_n(K)$ die Nullstellenmenge des Polynoms $X^n - 1 \in K[X]$ ist, gilt dies genau dann, wenn das Polynom $X^n - 1 \in K[X]$ vollständig in Linearfaktoren zerfällt, und alle Nullstellen paarweise verschieden sind, d.h.\ wenn das Polynom zerfällt und seperabel ist.

Gilt $p \coloneqq \ringchar(K) > 0$ und $n = p^r$, so gilt
\[
    X^n - 1
  = X^{p^r} - 1^{p^r}
  = (X - 1)^{p_r}.
\]
In diesem Fall ist also $\Gal(K(t) / K(t^n)) \cong C_n(K) \cong 1$.














\end{document}
