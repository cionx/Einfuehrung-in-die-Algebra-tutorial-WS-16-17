\documentclass[a4paper,10pt,numbers=noenddot]{scrartcl}

\usepackage{../generalstyle}
\usepackage{specificstyle}

\title{Lösung zu Zettel~11, \\ Aufgaben 3 und 4}
\author{Jendrik Stelzner}
\date{\today}

\begin{document}





\maketitle





\addtocounter{section}{2}





\section{}

Die $K$-linearen Körperhomomorphismen $K(t) \to K(t)$ sind nach der universellen Eigeschaft des Quotientenkörpers genau die eindeutigen Fortsetzungen der $K$-linearen, injektiven Ringhomomorphismen $K[T] \to K(t)$.

Die $K$-linearen Ringhomomorphismen $K[T] \to K(t)$ sind genau die Einsetzhomomorphismen von Elementen aus $K(t)$, wobei jedes Element einen anderen Einsetzhomomorphismus liefert.
Die injektiven Einsetzhomomorphismen entsprechen dabei genau den Elementen aus $K(t)$, die transzendent über $K$ sind.
Dabei ist nach Aufgabe~2 ein Element $q \in K(t)$ genau dann transzendent über $K$, wenn $q \notin K$.

Ingesamt erhalten wir, dass die $K$-linearen Körperhomomorphismen $K(t) \to K(t)$ genau die Einsetzhomomorphismen $p \mapsto p(q)$ für $q \in K(t)$ mit $q \notin K$ sind.

\begin{claim}
  \begin{enumerate}
    \item
      Ist $q \in K(t)$ mit $q \notin K$, so ist der entsprechende Einsetzhomomorphismus $\varphi \colon K(t) \to K(t)$, $p \mapsto p(q)$ genau dann ein Automorphismus, wenn $q$ von der Form $q = (at+b)/(ct+d)$ mit $a \neq 0$ oder $c \neq 0$ ist.
    \item
      Für ein beliebiges Element $(at+b)/(ct+d) \in K(t)$ gilt genau dann $(at+b)/(ct+d) \notin K$, wenn
      \[
        \begin{pmatrix}
          a & b
          \\
          c & d
        \end{pmatrix}
        \in \GL_2(K).
      \]
  \end{enumerate}
\end{claim}
\begin{proof}
  \begin{enumerate}
    \item
      Da Körperhomomorphismen immer injektiv sind, ist $\varphi$ genau dann ein Automorphismus, wenn $\varphi$ surjektiv ist.
      Dabei gilt $\im \varphi = K(q)$.
      Ist $q = f/g$ mit teilerfremden $f, g \in K[T]$, so gilt nach Aufgabe~2, dass
      \[
          [K(t) : \im \varphi]
        = [K(t) : K(q)]
        = \max\{ \deg f, \deg g \}.
      \]
      Folglich ist $\varphi$ genau dann surjektiv, also $[K(t) : \im \varphi] = 1$, wenn $f = at+b$ und $g = ct+d$ mit $a \neq 0$ oder $c \neq 0$.
    \item
      Es gilt $c \neq 0$ oder $d \neq 0$, und deshalb
      \begin{align*}
            &\, \frac{at+b}{ct+d} \in K
        \\
        \iff&\, \frac{at+b}{ct+d} = \lambda       \quad \text{für ein $\lambda \in K$}
        \\
        \iff&\, at + b = c \lambda t + d \lambda  \quad \text{für ein $\lambda \in K$}
        \\
        \iff&\, \left\{
                  \begin{matrix}
                    a &=& c \lambda
                    \\
                    b &=& d \lambda
                  \end{matrix}
                \right.                           \quad \text{für ein $\lambda \in K$}
        \\
        \iff&\, \text{$\vect{a \\ b}$ und $\vect{c \\ d}$ sind linear abhängig}
        \\
        \iff&\, \begin{pmatrix}
                  a & b
                  \\
                  c & d
                \end{pmatrix}
                \notin \GL_2(K).
      \end{align*}
  \end{enumerate}
\end{proof}

Insgesamt erhalten wir somit, dass die $K$-linearen Körperautomorphismen $K(t) \to K(t)$ genau die Einsetzhomomorphismen von Elementen $q \in K(t)$ der Form $q = (at+b)/(ct+d)$ mit
\[
  \begin{pmatrix}
    a & b
    \\
    c & d
  \end{pmatrix}
  \in
  \GL_2(K)
\]
sind.
Wir haben also eine surjektive Abbildung
\[
          \Phi
  \colon  \GL_2(K)
  \to     \Gal(K(t) / K),
  \quad
          \begin{pmatrix}
            a & b
            \\
            c & d
          \end{pmatrix}
          \mapsto
          \left(
                    p
            \mapsto p\left( \frac{at+b}{ct+d} \right)
          \right).
\]
Diese Abbildung ist ein Gruppenantimorphismus, d.h.\ es gilt
\[
  \Phi(S_1) \Phi(S_2) = \Phi(S_2 S_1)
  \quad
  \text{für alle $S_1, S_2 \in \GL_2(K)$}.
\]
Sind nämlich
\[
    S_1
  = \begin{pmatrix}
      a & b
      \\
      c & d
    \end{pmatrix}
  \quad
  \text{und}
  \quad
    S_2
  = \begin{pmatrix}
      a'  & b'
      \\
      c'  & d'
    \end{pmatrix},
\]
so gilt
\begin{align*}
      (\Phi(S_1) \Phi(S_2))(t)
  &=  \Phi(S_1)( \Phi(S_2)(t) )
  =   \Phi(S_1)\left( \frac{a' t + b'}{c' t + d'} \right)
  =   \frac{a' \Phi(S_1)(t) + b'}{c' \Phi(S_1 )(t) + d'}
  \\
  &=  \frac{a' \frac{a t + b}{c t + d} + b'}{c' \frac{a t + b}{c t + d} + d'}
  =   \frac{a'(at+b) + b'(ct+d)}{c'(at+b) + d'(ct+d)}
  =   \frac{(a'a + b'c)t + (a'b + b'd)}{(c'a + d'c)t  + (c'b + d'd)}
  \\
  &=  \Phi\left(
            \begin{pmatrix}
              a'a + b'c & a'b + b'd
              \\
              c'a + d'c & c'b + d'd
            \end{pmatrix}
          \right)(t)
  =   \Phi(S_2 S_1)(t).
\end{align*}

Es gilt $\ker \Phi = K^\times \cdot I$, denn
\begin{align*}
        \begin{pmatrix}
          a & b
          \\
          c & d
        \end{pmatrix}
        \in
        \ker \Phi
  &\iff
        \Phi
        \left(
          \begin{pmatrix}
            a & b
            \\
            c & d
          \end{pmatrix}
        \right)
        =
        \id
  \iff
        \Phi
        \left(
          \begin{pmatrix}
            a & b
            \\
            c & d
          \end{pmatrix}
        \right)
        (t)
        =
        t
  \\
  &\iff
        \frac{at+b}{ct+d} = t
  \iff
        at + b = ct^2 + dt
  \iff
        a = d, b = c = 0.
\end{align*}
Somit induziert $\Phi$ einen Antiisomorphismus von Gruppen
\[
  \overline{\Phi}
  \colon
  \PGL_2(K) \to \Gal(K(t)/K),
  \quad
  \overline{
  \begin{pmatrix}
    a & b
    \\
    c & d
  \end{pmatrix}
  }
  \mapsto
  \Phi
  \left(
    \begin{pmatrix}
      a & b
      \\
      c & d
    \end{pmatrix}
  \right).
\]

\begin{remark}
  Dass $\Phi$ ein Gruppenantimorphismus, aber kein Gruppenhomomorphismus ist, lässt sich dadurch reparieren, dass man $\Phi$ durch die Abbildung
  \[
            \Psi
    \colon  \GL_2(K) \to \Gal(K(t)/K),
    \quad   S \mapsto \Phi(S)^{-1} = \Phi(S^{-1})
  \]
  ersetzt.
  Für jede Gruppe $G$ ist die Abbildung $(-)^{-1} \colon G \to G$, $g \mapsto g^{-1}$ ein Antiisomorphismus, also ist $\Phi$ als Komposition zweier Gruppenantimorphismen ein Gruppenhomomorphismus.
  Dabei gilt $\ker \Psi = \ker \Phi$, weshalb $\Psi$ einen Isomorphismus von Gruppen
  \[
    \overline{\Psi}
    \colon
    \PGL_2(K) \to \Gal(K(t)/K),
    \quad
    \overline{
    \begin{pmatrix}
      a & b
      \\
      c & d
    \end{pmatrix}
    }
    \mapsto
    \Psi
    \left(
      \begin{pmatrix}
        a & b
        \\
        c & d
      \end{pmatrix}
    \right).
  \]
  induziert.
  Dieser ist konkret durch
  \begin{align*}
    \overline{\Psi}
    \left(
      \overline{
        \begin{pmatrix}
          a & b
          \\
          c & d
        \end{pmatrix}
      }
    \right)
    (p)
    &=
    \Phi
    \left(
      \begin{pmatrix}
        a & b
        \\
        c & d
      \end{pmatrix}^{-1}
    \right)
    (p)
    =
    \Phi
    \left(
      \frac{1}{ad - bc}
      \begin{pmatrix*}[r]
         d  & -b
        \\
        -c  &  a
      \end{pmatrix*}
    \right)
    (p)
    \\
    &=
    \Phi
    \left(
      \begin{pmatrix*}[r]
         d  & -b
        \\
        -c  &  a
      \end{pmatrix*}
    \right)
    (p)
    =
    p\left( \frac{\phantom{-}dt - b}{-ct + a} \right)
  \end{align*}
  gegeben.
\end{remark}





\section{}

Es sei
\[
            C_n(K)
  \coloneqq \{ \omega \in K \mid \omega^n = 1 \}
\]
die Gruppe der $n$-ten Einheitswurzeln in $K$;
es handelt sich um eine Untergruppe von $K^\times$.
Nach Aufgabe~3 ergibt sich für jedes $\omega \in C_n(K)$ ein $K$-linearer Körperautomorphismus
\[
          \sigma_\omega
  \colon  K(t) \to K(t),
  \quad   p(t) \mapsto p(\omega t).
\]
Dabei gilt $\sigma_{\omega_1} \circ \sigma_{\omega_2} = \sigma_{\omega_1 \omega_2}$ für alle $\omega_1, \omega_2 \in C_n(K)$, weshalb die Abbildung
\[
          \Phi
  \colon  C_n(K) \to \Gal(K(t)/K),
  \quad   \omega \mapsto \sigma_\omega
\]
ein Gruppenhomomorphismus ist.
Da $\Phi(\omega)(t) = \sigma_\omega(t) = \omega t$ ist $\Phi$ injektiv, und wir zeigen, dass $\im \Phi = \Gal(K(t) / K(t^n))$;
damit liefert die Zuordnung $\omega \mapsto \sigma_\omega$ einen Isomorphismus $C_n(K) \to \Gal(K(t) / K(t^n))$.

Es gilt $\im \Phi \subseteq \Gal(K(t) / K(t^n))$,
denn für jedes $\omega \in C_n(K)$ gilt
\[
    \sigma_\omega(t^n)
  = \sigma_\omega(t)^n
  = (\omega t)^n
  = \omega^n t^n
  = t^n
\]
und somit $\sigma_\omega|_{K(t^n)} = \id_{K(t^n)}$.

Ist andererseits $\sigma \in \Gal(K(t)/K(t^n)) \subseteq \Gal(K(t)/K)$, so gibt es nach Aufgabe~3 ein eindeutiges Element $q \in K(t)$ mit $q \notin K$, so dass $\sigma(p) = p(q)$ für alle $p \in K(t)$.
Wegen der $K(t^n)$-Linearität von $\sigma$ muss dabei
\[
    q^n
  = \sigma(t)^n
  = \sigma(t^n)
  = t^n.
\]
Die Surjektivität von $\Phi$ nach $\Gal(K(t)/K(t^n))$ folgt deshalb aus der folgenden Aussage:

\begin{claim}
  Ist $q \in K(t)$ mit $q^n = t^n$, so ist $q = \omega t$ mit $\omega \in C_n(K)$.
\end{claim}
\begin{proof}
  Es sei $\mathcal{P} \subseteq K[T]$ ein Repräsentantensystem der Primelemente von $K[T]$.
  Wir können o.B.d.A.\ davon ausgehen, dass $t \in \mathcal{P}$.
  Dann gibt es eine eindeutige Darstellung
  \[
    q = \omega \prod_{p \in \mathcal{P}} p^{\nu_p}
  \]
  mit $\omega \in K^\times$ und $\nu_p \in \Integer$ für alle $p \in \mathcal{P}$, wobei $\nu_p = 0$ für fast alle $p \in \mathcal{P}$.
  Es gilt
  \[
      t^n
    = q^n
    = \omega^n \prod_{p \in \mathcal{P}} p^{n \nu_p}
    = \omega^n t^{n \nu_t} \prod_{p \in \mathcal{P} \setminus \{t\}} p^{n \nu_p},
  \]
  und aus der Eindeutigkeit dieser Darstellung für $t^n$ ergibt sich, dass $\omega^n = 1$, $\nu_t = 1$ und $\nu_p = 0$ für alle $ \in \mathcal{P} \setminus \{t\}$.
  Also ist $q = \omega t$ mit $\omega \in C_n(K)$.
\end{proof}

Die Galoisgruppe der Körpererweiterung $K(t)/K(t^n)$ ist also die Gruppe der $n$-ten Einheitswurzeln.
Nach Aufgabe~2 gilt $[K(t) : K(t^n)] = n$ (und das Minimalpolynom von $t$ über $K(t^n)$ ist $X - t^n \in K(t^n)[X]$),
also ist $K(t)/K(t^n)$ genau dann galoisch, wenn $|C_n(K)| = n$.
Da $C_n(K)$ die Nullstellenmenge des Polynoms $X^n - 1 \in K[X]$ ist, gilt dies genau dann, wenn das Polynom $X^n - 1 \in K[X]$ vollständig in Linearfaktoren zerfällt, und alle Nullstellen paarweise verschieden sind, d.h.\ wenn das Polynom zerfällt und seperabel ist.

Gilt $p \coloneqq \ringchar(K) > 0$ und $n = p^r$, so gilt
\[
    X^n - 1
  = X^{p^r} - 1^{p^r}
  = (X - 1)^{p_r}.
\]
In diesem Fall ist also $\Gal(K(t) / K(t^n)) \cong C_n(K) \cong 1$.














\end{document}
